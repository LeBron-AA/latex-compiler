\documentclass[12pt]{article}
\usepackage[utf8]{inputenc}
\usepackage{amsmath}
\usepackage{amssymb}
\usepackage{enumerate}
\usepackage{enumitem}
\usepackage{geometry}
\usepackage{hyperref}
\usepackage{setspace}
\usepackage{titlesec}


% ----- Custom counters and commands -----
\newcounter{unit}[section]
\renewcommand{\theunit}{\arabic{unit}}

\newcounter{chapter}[subsection]
\renewcommand{\thechapter}{\arabic{chapter}}

\renewcommand{\thesubsubsection}{\theunit.\arabic{subsubsection}}

\newcommand{\chapter}[1]{
    \refstepcounter{chapter}
    \subsection*{\Large{\S \thechapter. #1}}
    \addcontentsline{toc}{subsection}{\thechapter. #1}
}

\newcommand{\unit}[1]{
    \refstepcounter{unit}
    \section*{\Huge{\Roman{unit} #1}}
    \addcontentsline{toc}{section}{\Roman{unit} #1}
}

\newcommand{\longrightsquigarrow}{%
  \mathrel{\tikz[baseline=-0.5ex] \draw[->,decorate,decoration={snake,amplitude=.7mm,segment length=2.5mm,post length=1mm}] (0,0) -- (1.7,0);}}

\newcommand{\dem}{
    \noindent \underline{\textbf{Demostración:}}
}

\titleformat{\subsubsection}
  {\normalfont\large\bfseries} % mismo tamaño que \subsection
  {\thesubsubsection}{1em}{}

% ----------------------------------------

\begin{document}

\vfill
\begin{center}
    \Huge{Álgebra Lineal y Geometría I} \\[3ex] 
    \Large Apuntes elementales introductorios\\[3ex]
    \large{Javier Ortín Rodenas}
\end{center}
\vspace{2cm}
\onehalfspacing

Estos apuntes han sido confeccionados por el alumno un año después de haber cursado
la asignatura a la que hacen referencia, pero no recrean los apuntes originales del profesorado.
Este documento ha sido escrito desde cero sin consultal el material del equipo docente y sirve
como un ejercicio del autor para reflexionar sobre los conocimientos abordados entonces.

\newpage
\tableofcontents
\newpage

\unit{Espacios vectoriales}

\chapter{Cuerpos}

\subsubsection{Definición de operación binaria interna}
\hspace{3mm}
Sea $K$ un conjunto no vacío, una operación binaria interna es una función que toma dos
argumentos de entrada en $K$ y tiene salida también en $K$. Es decir, diremos que $*$ es una
operación binaria interna en $K$ si puede ser definida como:
\begin{align*}
    * : K \times K \longrightarrow K &&
    k_1, k_2 \rightsquigarrow k_1 * k_2
\end{align*}

\subsubsection{Definición de cuerpo}
\hspace{3mm}
Sea $K$ un conjunto no vacío. Sean $+$ (suma) y $\cdot*$ (multiplicación) dos operaciones
binarias internas en $K$. Diremos que $(K,+,\cdot)$ tiene estructura de cuerpo si cumple
todos los siguientes requisitos:
\begin{enumerate}[label=\roman*)]
    \item Existe un elemento neutro para la suma; es decir, al sumarlo a cualquier elemento de $K$ no altera el resultado: \vspace{-4mm}
        $$\exists \hspace{1mm} 0_K \in K : k + 0_K = k \hspace{3mm} \forall k \in K$$

    \item Todo elemento de $K$ tiene un opuesto para la suma: \vspace{-4mm}
        $$\forall k \in K \hspace{3mm} \exists  \hspace{1mm}-k \in K
        \hspace{2mm} : \hspace{2mm} k + (-k) = 0_K$$

    \item Existe un elemento neutro para la multiplicación: \vspace{-4mm}
        $$\exists \hspace{1mm} 1_K \in K : k \cdot 1_K = k \hspace{3mm} \forall k \in K$$

    \item Para todo elemento de $K$ salvo $0_K$ existe un inverso para la multiplicación:
        $$\forall k \in K\backslash\{0_K\} \hspace{3mm} \exists \hspace{1mm} k^{-1}
        : k \cdot k^{-1} = 1_K$$

    \item La suma y la multiplicación son conmutativas:
        \begin{align*}
            \forall k_1, k_2 \in K \text{ se verifica: } &&
            k_1 + k_2 = k_2 + k_1 &&
            k_1 \cdot k_2 = k_2 = k_1
        \end{align*}
    
    \item La multiplicación es distributiva respecto de la suma:
        $$\forall k_1,k_2,k_3 \in K : \hspace{2mm}
        k_1 \cdot (k_2 + k_3) =
        k_1 \cdot k_2 + k_1 \cdot k_3 $$
\end{enumerate}

En vista de esta definición, es claro que ($\mathbb{R}, +, \cdot \hspace{1mm})$
forma un cuerpo para la suma y multiplicación usuales para los números reales.
Estas mismas operaciones definen también un cuerpo para los números complejos $\mathbb{C}$.
Veamos algunos ejemplos y contraejemplos de esta definición.


\subsubsection{Números racionales}
\hspace{3mm}
Consideramos la suma y el producto usual en el conjunto de los números racionales $\mathbb{Q}$.
Cabe preguntarse si tiene estructura de cuerpo. Veamos que en efecto es así:

\vspace{2mm}
Por definición, $\mathbb{Q}$ está formado por divisiones de números enteros, luego podemos expresarlo
de la siguiente forma:
$$\mathbb{Q} = \left\{\frac{p}{q} : p,q \in \mathbb{Z}\right\}$$

Sea $a \in \mathbb{Q}$ cualquiera, podemos reescribir $a = \frac{p}{q}$ para $p, q \in \mathbb{Z}$.
Para hallar el opuesto a $a$ basta tomar $-a = \frac{-p}{q}$ pues se cumple:
$$a + (-a) = \frac{p}{q} + \frac{-p}{q} = \frac{p-p}{q} = 0$$

Además, es claro que $-a \in \mathbb{Q}$ pues se cumple:
$$a = \frac{p}{q} \in \mathbb{Q} \rightarrow
 p,q \in \mathbb{Z} \Rightarrow -p,q \in \mathbb{Z}
 \Rightarrow -a = \frac{-p}{q} \in \mathbb{Q}$$

Supongamos ahora que $a \neq 0$; de este modo, $p \neq 0$.
Así, bastaría tomar $a^{-1} = \frac{q}{p}$ inverso de $a$ bien
definido.
$$ a \cdot a^{-1} = \frac{p}{q} \cdot \frac{q}{p}
 = \frac{p \cdot q}{p \cdot q} = 1 \hspace{6mm}
 \text{ Es en efecto su inverso}$$
Además, al ser $a$ un número racional, $p$ y $q$ han de ser números enteros
luego $a^{-1}$ ha de ser también racional por definición.

\vspace{2mm}
Por todo lo anterior, $\mathbb{Q}$ cuenta con opuestos e inversos dentro del propio
conjunto. La conmutatividad y la distributivad son propiedades triviales
del producto y la suma usual. Concluimos que tiene estructura de cuerpo.

\subsubsection{Núeros naturales y enteros}
\hspace{3mm}
Los números naturales no forman un cuerpo, pues no tienen opuesto para la suma.
Por ejemplo, el opuesto del natural $5$ es el número $-5$, que no pertenece
a $\mathbb{N}$.

\vspace{2mm}
Aunque sí existen los opuestos para la suma en $\mathbb{Z}$, no 
es posible hallar inversos para el producto usual. Por ejemplo, el inverso
del entero 2 es $\frac{1}{2}$, que no pertenece al conjunto de los números enteros.

\subsubsection{Conjuntos de matrices}
\textbf{Ejercicio:} Halla un conjunto de matrices que tenga estructura de cuerpo

\vspace{2mm} \hspace{3mm}
El conjunto de matrices cualesquiera de números reales no es un cuerpo, pues no es posible
sumar matrices de distinta dimensión (entendiendo por suma de matrices la suma término a término).
Por tanto, si buscamos encontrar un conjunto de matrices que sí sea un cuerpo va a ser necesario
considerar un conjunto en el que todas ellas tengan una misma dimensión dada.

\vspace{2mm}
No obstante, esto no es suficiente para que el conjunto sea un cuerpo, pues la multiplicación usual de matrices
exige que el número de columnas de la primera matriz sea igual al número de columnas de la segunda. Al imponer también esta condición,
es claro que el conjunto de matrices que buscamos ha de ser (como máximo) el de las matrices cuadradas de un cierto tamaño cualquiera.

\vspace{2mm}
Sea $n \in \mathbb{N}$ cualquiera, introducimos la siguiente notación:
$$M_n(\mathbb{R}) = \{A : A\text{ es una matriz } n \times n \text{ con coeficientes en } \mathbb{R}\}$$
Este conjunto no tiene estructura de cuerpo pues no todas las matrices cuadradas son invertibles. Para que así fuera,
tendríamos que exigir que fuesen invertibles; es decir, que tengan determinante no nulo. Nótese que
es necesario hacer una excepción para la matriz nula y poder tener así un elemento neutro de la suma.

\vspace{2mm}
Incluso habiendo exigido todas estas restricciones contamos aún con problemas. La multiplicación de matrices puede
no ser conmutativa. Veamos un contraejemplo:
\begin{align*}
    A =
    \begin{pmatrix}
        1 & 1\\
        1 & 2
    \end{pmatrix} &&
    B = 
    \begin{pmatrix}
        2 & 0\\
        0 & 3
    \end{pmatrix}&&
    AB = 
    \begin{pmatrix}
        2 & 3\\
        2 & 6
    \end{pmatrix} &&
    BA = 
    \begin{pmatrix}
        2 & 2\\
        3 & 6
    \end{pmatrix}
\end{align*}

Un ejemplo de matrices que conmutan y son cerradas para el producto y la suma son las matrices diagonales. No obstante,
si queremos que el conjunto tenga también determinante no nulo pueden ocurrir complicaciones:
\begin{align*}
    \begin{pmatrix}2&0\\0&1\end{pmatrix} +
    \begin{pmatrix}5&0\\0&-1 \end{pmatrix} =
    \begin{pmatrix}7&0\\0&0\end{pmatrix}
\end{align*}
Las dos matrices sumando tienen determinante no nulo, pero no es el caso de la tercera.
Esto ocurre al aparecer filas nulas en un resultado que no es la matriz nula (elemento neutro de la suma).
Para solventar este problema, podemos exigir que todos los escalares de la diagonal deban de valer lo mismo.
De este modo, si apareciese alguna fila nula al sumar o multiplicar matrices, el resultado ha de ser necesariamente
la matriz nula (que sí está "permitida").

\newpage
Por todo lo anterior, concluimos que el conjunto de matrices diagonales $n\times n$ con coeficientes reales idénticos tiene estructura de cuerpo:
$$ \left\{\lambda \hspace{1mm} \text{Id}_n : \hspace{2mm}\lambda \in \mathbb{R} \right\}
= \left\{\begin{pmatrix}
    \lambda \hspace{3mm} \ldots \hspace{3mm} 0 \\
    \vdots \hspace{3mm} \ddots \hspace{3mm} \vdots \\
    0 \hspace{3mm} \ldots \hspace{3mm} \lambda
\end{pmatrix}\hspace{4mm} : \lambda \in \mathbb{R} \right\}$$
De hecho, como hemos comprobado anteriormente que $\mathbb{Q}$ es un cuerpo, podemos considerar también
las matrices con coeficientes diagonales idénticos en $\mathbb{Q}$ (o en cualquier otro cuerpo $K$).

\vspace{2mm}
\subsubsection{Proposición: Unicidad del elemento neutro}
\hspace{3mm}
Sea $(K, +, \cdot \hspace{1mm})$ un cuerpo cualquiera. Los elementos neutros $0_K$ y $1_K$ son únicos.

\vspace{2mm}
\dem Supongamos que existe más de uno para ver que son iguales

\vspace{2mm}
Veamos que el elemeneto neutro de la suma es único (el procedimiento para la multiplicación es análogo).
Supongamos que existen dos elementos neutros de la suma en $K$: $0_k$ y $\hat{0}_K$.

\vspace{2mm}
Por ser $\hat{0}_K$ elemento neutro, $0_K = 0_K + \hat{0}_K$

\vspace{2mm}
Como $0_K$ es también elemento neutro, $\hat{0}_K = \hat{0}_K + 0_K$

\vspace{2mm}
Finalmente, como la suma ha de ser conmutativa pues $K$ es un cuerpo, se tiene: \vspace{-3mm}
$$0_K = 0_K + \hat{0}_K = \hat{0}_K + 0_K = \hat{0}_K$$

\noindent
Por todo lo anterior, concluimos que el elemento neutro es único.

\newpage
\chapter{Introducción a los espacios vectoriales}
\subsubsection{Definición de espacio vectorial}
\hspace{3mm}
El concepto de espacio vectorial depende del de cuerpo, pues todo espacio vectorial
está asociado a un cuerpo determinado. Sea $(K, + \hspace{1mm}\cdot \hspace{1mm})$
un cuerpo cualquiera, sea $V$ un conjunto no vacío. Para que $V$ sea un $K$-espacio vectorial
ha de ser cerrado respecto de la suma de vectores y respecto del producto por escalares; es decir:
\label{requisitos-espacio-vectorial}
\begin{enumerate}[label=\roman*)]
    \item Cerrado para la suma de vectores: $\forall v, w \in V \hspace{1mm} : v + w \in V$
    \item Cerrado para el producto por escalares: $\forall v \in V \hspace{1mm} \forall \lambda \in K
      \hspace{1mm} \lambda v \in V$
\end{enumerate}
Como alternativa, basta comprobar únicamente que se cumpla la siguiente condición:
\begin{enumerate}[label=$\hat{\roman*)}$]
    \item $\forall v,w \in V \hspace{3mm}
        \forall \alpha, \beta \in K \hspace{1mm} :
        \hspace{2mm}\alpha v + \beta w \in V$
\end{enumerate}

\vspace{2mm}
Finalmente, las operaciones de suma de vectores y producto por escalares han de cumplir los ocho siguientes axiomas:
\begin{enumerate}[label = E.\arabic*)]
    \item La suma es conmutativa: \hspace{2mm}
        $v + w = w + v \hspace{4mm} \forall v,w \in V$
    \item La suma es asociativa: \hspace{2mm}
        $u + (v + w) = (u + v) + w \hspace{4mm} \forall u,v,w \in V$
    \item Existencia del elemento neutro: \hspace{2mm}
        $\exists 0_V \in V : v + 0_V = v \hspace{4mm} \forall v \in V$
    \item Existencia de los opuestos: \hspace{2mm}
        $\forall v \in V \hspace{2mm} \exists -v \in V : 
            v + (-v) = 0_V$
    \item El producto por escalares es asociativo: \hspace{1mm}
        $\alpha \cdot (\beta \cdot v) = (\alpha \cdot  \beta) \cdot v
            \hspace{2mm} \forall \alpha, \beta \in K
            \hspace{2mm} \forall v \in V$
    \item Elemento neutro del producto por escalares: \hspace{2mm}
        $1_K v = v \hspace{3mm} \forall v \in V$   
    \item Distributividad respecto de la suma vectorial:\\
        $\lambda(v + w) = \lambda v + \lambda w \hspace{3mm}
            \forall \lambda \in K \hspace{3mm} \forall v,w \in V$
    \item Distributividad respecto de la suma escalar: \\
        $(\alpha + \beta)v = \alpha v + \beta v \hspace{3mm}
            \forall \alpha, \beta \in K \hspace{3mm} \forall v \in V$
\end{enumerate}

En los ejercicios prácticos de esta asignatura los ocho axiomas suelen cumplirse trivialmente, por lo que
basta comprobar que $V$ sea cerrado para la suma y para el producto por escalares, además de ser un
conjunto no vacío.

\vspace{2mm}
Nótese que con los contenidos vistos hasta ahora los vectores no se multiplican entre sí, sino que únicamente
se suman vectores cuantificados por escalares (estos últimos actúan como "pesos" o "coeficientes"
de los primeros).

\vspace{2mm}
Por las definiciones de cuerpo y de espacio vectorial es evidente que 
todo cuerpo es un espacio vectorial sobre sí mismo. Veamos otros ejemplos no tan triviales.

\vspace{2mm}

\vspace{4mm}
\subsubsection{Matrices como espacio vectorial}
\hspace{3mm}
Sea $K$ un cuerpo, sean $n, m \in \mathbb{N}$ cualesquiera, el conjunto de matrices $n \times m$ con coeficientes
en $K$ es un $K$-espacio vectorial. Introducimos la siguiente notación: \vspace{-2mm}
$$M_{n,m}(K) = \{A : A\text{ es un matriz } n\times m \text{ con coeficientes en }K\}$$
En este caso, los escalares serían los elementos de $K$, mientras que los vectores serían las matrices
de $M_{n,m}(K)$. Veamos que es en efecto un espacio vectorial.

\vspace{2mm}
En primer lugar, consideramos la suma de matrices como la suma término a término. Al haber fijado
la dimensión $n \times m$, esta operación está bien definida. Además, la suma de matrices $n\times m$
da luegar también a una matriz $n \times m$. Finalemnte, como $K$ es un cuerpo, la suma de elementos de $K$
es también un elemento de $K$. Por tanto, el resultado de sumar dos matrices de $M_{n,m}(K)$ es también una matriz
de $M_{n,m}(K)$.

\vspace{2mm}
De manera simlar, el producto por escalares no afecta a la dimensión del resultado, y al ser
$K$ un cuerpo el producto es una operación binaria interna. De este modo, $M_{n,m}(K)$ es cerrado
respecto del producto por escalares de $K$. Por todo lo anterior, $M_{n,m}(K)$ es un $K$-espacio vectorial.

\newpage
\subsubsection{Polinomios como espacio vectorial}
\label{polinomios-espacio-vectorial}
\hspace{3mm}
Sea $K$ un cuerpo, sea $K[X]$ el conjunto de polinomios en la variable $X$ con coeficientes en $K$, veamos que es un $K$-espacio vectorial.
Sean $p, q \in P[K]$, supongamos que $p$ es de grado $n$ y $q$ de grado $m$ para ciertos $n,m\in \mathbb{N}$. Por tanto,
podemos expresarlos de la siguiente forma para ciertos $a_i, b_i \in K$: \vspace{-3mm}
\begin{flalign*}
    &&p &= a_n X^n + a_{n-1} X^{n-1} + \ldots + a_1X + a_0  &&&&&&&&&&&&&\\
    &&q &= b_m X^m + b_{m-1} X^{m-1} + \ldots + b_1X + b_0
\end{flalign*}
En este caso, realizaremos una única comprobación. Sean $\alpha, \beta \in K$,
definimos $N := \max\{n,m\}$. De este modo: \vspace{-3mm}
\begin{flalign*}
    \alpha p& + \beta q = (\alpha a_N + \beta b_N)X^N + \ldots (\alpha a_1 + \beta b_1)X + (\alpha a_0 + \beta b_0) = \\
     &=  c_N X^N + \ldots + c_1 X + c_0  \hspace{4mm}\text{ para } c_i = \alpha a_i + \beta b_i
\end{flalign*}
Como $K$ es un cuerpo y $\alpha, \beta, a_i, b_i \in K$, los $c_i$ también pertenecen a $K$.
Por tanto, tenemos que $\hspace{1mm}\alpha p + \beta q \in K[X]\hspace{1mm}$ luego $K[X]$ es un $K$-espacio vectorial.
Cabe preguntarse si existen subconjuntos de $K[X]$ que sean también espacios vectoriales. 

\vspace{2mm} Sea $n \in \mathbb{N}$ cualquiera, consideremos el conjunto de polinomios de $K[X]$ de grado
mayor o igual que $n$. Este conjunto no es un espacio vectorial, pues no es cerrado para la suma. Por ejemplo,
para $n = 3$ y $K = \mathbb{R}$ tenemos: \vspace{-3mm}
$$(X^4 + 2X^3 + 3X) + (-X^4 -2X^3 + 2) =  3X + 2
\hspace{3mm} \text{ que tiene grado } 1 < 3$$ 
Además, este conjunto no cumple con el axioma E.3) de la definición de espacio vectorial, pues
el elemento neutro de la suma es el polinomio nulo, que no pertenece a este conjunto al tener grado 0.

\vspace{2mm}
El cojnunto de polinomios de $K[X]$ con grado menor o igual que $n$ sí es un espacio vectorial, y se denota
$K_n[X]$. La demostración es idéntica a la vista para $K[X]$ notando que los polinomios $p$ y $q$ serían ambos
de grado menor o igual que $n$ luego bastaría tomar $N = n$.

\newpage
\subsubsection{Funciones como espacio vectorial}
\hspace{3mm}
Sea $A$ un conjunto no vacío cualquiera. Sea $K$ un cuerpo, sea $V$ un $K$-espacio vectorial. Consideramos el conjunto de funciones
$\{f:A \longrightarrow V\}$.Veamos que es un $K$-espacio vectorial.

\vspace{2mm}
La suma de dos funciones $f,g: A \longrightarrow V$ es también una función de este conjunto, pues se cumple: \vspace{-3mm}
$$\forall x \in A : f(x), g(x) \in V \Rightarrow f(x) + g(x) = (f+g)(x) \in V$$
Del mismo modo, sea $\lambda \in K$ cualquiera, por ser $V$ un $K$-espacio vectorial se tiene: \vspace{-3mm}
$$\forall x \in A : f(x) \in A \Rightarrow \lambda \hspace{1mm} f(x) = (\lambda f)(x) \in V$$
Es claro que es un espacio vectorial al ser cerrado para la suma y para el producto por escalares. Además, el subconjunto
$\{f : A \longrightarrow V \text{ tal que }f \text{ es continua}\}$ es también un espacio vectorial, pues la suma de dos funciones
continuas es también una función continua, y lo mismo ocurre con el producto por escalares.

\vspace{6mm}
\chapter{Sistemas de vectores}
\subsubsection{Definición de sistema}
\hspace{3mm}
A partir de ahora nos referiremos a los conjuntos de vectores de un espacio vectorial como "familias" o "sistemas" de vectores.

\subsubsection{Sistemas libres y ligados}
\hspace{3mm}
Diremos que un sistema de vectores es "libre" si todos los vectores que lo forman son linealmente
independientes entre sí. Esto quiere decir que ningún vector del sistema puede ser expresado como
combinación lineal de los demás.

\newpage
Por el contrario, si el sistema no es libre diremos que es "ligado". En tal caso, todos los elementos
del sistema pueden ser expresados como combinación lineal de los demás (basta expresar uno en función del
resto y despejar en tal expresión para obtener la del resto).

\vspace{2mm}
Sea $K$ un cuerpo, sea $V$ un $K$-espacio vectorial, sea $\{v_1, \ldots, v_n\}$ un sistema
de vectores de $V$. Supongamos que tal sistema no es libre. Así, existe $i \in \{1,\ldots,n\}$
para el que se cumple: \vspace{-3mm}
\begin{flalign}
    v_i = \lambda_1 v_1 + \ldots + \lambda_{i-1}v_{i-1} + \lambda_{i+1}v_{i+1} + \ldots + \lambda_n v_n
\end{flalign}
Nótese que si $v_i \neq 0_V$ necesariamente ha de cumplirse $\lambda_j \neq 0 \hspace{1mm}$
para cierto índice $j \in \{1, \ldots, n\}\backslash\{i\}$. De este modo, despejando: \vspace{-3mm}
\begin{flalign}
    0_V = v_i + (-v_i) = \lambda_1 v_1 + \ldots + \lambda_{i-1}v_{i-1} -v_i + \lambda_{i+1}v_{i+1} + \ldots + \lambda_n v_n 
\end{flalign}
En el caso $v_i \neq 0_V$ tenemos que es posible escribir el $0_V$ como una combinación lineal del sistema
donde haya al menos un coeficiente no nulo (sería el $\lambda_j$ mencionado anteriormente). Podemos pasar de la ecuación (1)
a la (2) y viceversa sin más que despejar. Por tanto, si podemos expresar el $0_V$ como combinación lineal del sistema con al menos
un coeficiente no nulo sera porque hay un vector no nulo que es combinación de los demás (las implicaciones son bidireccionales).

\vspace{2mm}
Sea $v \in V$ cualquiera, veamos que $0_K v = 0_V$: \vspace{-3mm}
$$0_V = v + (-v) = 1_K v + 1_K(-v) = 1_K v + (-1_K)v = (1_K-1_K)v = 0_K v$$ \\[-5ex]
Por tanto, cualquier sistema con el $0_V$ en él va a ser ligado, pues siempre va a
poder ser expresado como combinación lineal del resto de vectores al tomar $0_K$ como
coeficiente para el resto de vectores. El sistema $\{0_V\}$ también se considera ligado.

\vspace{2mm} \noindent
Por todo lo anterior, concluimos que las dos siguientes afirmaciones son equivalentes:
\begin{enumerate}[label=\roman*)]
    \item $\{v_1, \ldots, v_n\}$ es un sistema libre
    \item $0_V = \lambda_1 v_1 + \ldots + \lambda_n v_n
        \Rightarrow \lambda_i = 0_K \hspace{2mm} \forall i \in \{1,\ldots,n\}$
\end{enumerate}

\subsubsection{Clausura de una familia, sistema generador}
\hspace{3mm}
Sea $K$ un cuerpo, sea $V$ un $K$-espacio vectorial. Sea $\{v_1, \ldots, v_n\} $ un
sistema de vectores de $V$. Se define la clausura de $\{v_1, \ldots, v_n\} $ en $K$ como:
\vspace{-3mm}
\begin{flalign*}
    K \left<v_1, \ldots, v_n\right> := \big\{\lambda_1 v_1 + \ldots + \lambda_n v_n : \lambda_i \in K \hspace{2mm}\forall i \in \{1, \ldots, n\}\big\}
\end{flalign*}
Diremos que $\{v_1, \ldots, v_n\} $ es sistema generador de $V$ si su clausura es el propio $V$. Además, veamos
que la clausura de un sistema es un espacio vectorial. De hecho, es el menor subespacio vectorial que contiene a dicha familia.

\dem

En primer lugar, \hyperref[requisitos-espacio-vectorial]{veamos que es un espacio vectorial}. Sean $u, w \in K\left<v_1, \ldots, v_n \right>$,
podemos escribirlos de la siguiente forma: \vspace{-3mm}
\begin{align*}
    u = \lambda_1 v_1 + \ldots + \lambda_n v_n &&
    w = \gamma_1 v_1 + \ldots + \gamma_n v_n
\end{align*}
Sean $\alpha, \beta \in K$, veamos que $\alpha u + \beta w \in K\left<v_1, \ldots, v_n \right>$:
\vspace{-3mm}
\begin{flalign*}
    \alpha u &+ \beta w = \alpha \lambda_1 v_1 + \ldots + \alpha \lambda_n v_n
     + \beta \gamma_1 v_1 + \ldots + \beta \gamma_n v_n = \\ 
    &= (\alpha \lambda_1 + \beta \gamma_1)v_1 + \ldots + (\alpha \lambda_n + \beta \gamma_n)v_n
\end{flalign*} \\[-4ex]
Al ser $K$ un cuerpo, como $\alpha,\beta,\lambda_i,\beta_i \in K \Rightarrow (\alpha \lambda_i + \beta \gamma_i) \in K \hspace{2mm}\forall i \in \{1, \ldots, n\}$.
De este modo, es claro que $\alpha u + \beta w \in K\left<v_1, \ldots, v_n \right>$, luego la clausura es un espacio vectorial.
Queda ver que es el menor espacio vectorial que contiene a $\{v_1, \ldots, v_n\} $.

\vspace{2mm}
Supongamos que exite $T$ un $K$-espacio vectorial tal que $\{v_1, \ldots, v_n\} \subseteq T$. Veamos que la clausura
de los $v_i$ está contenida en $T$. Sea $u \in K\left<v_1, \ldots, v_n \right>$ cualquiera, podemos expresarlo como
$\displaystyle{u = \sum_{i=1}^n} \lambda_i v_i$.

Como $T$ es un espacio vcectorial y $\{v_1, \ldots, v_n\} \subseteq T$ por hipótesis, tenemos que $\forall i \in \{1, \ldots, n\} $
se tiene $\lambda_i v_i \in T$ (pues es cerrado respecto del producto por escalares). Además, como es cerrado respecto de la suma de vectores
se tiene $u = \lambda_1 v_1 + \ldots + \lambda_n v_v \in T$. Hemos llegado al resultado que buscábamos.

\newpage
\chapter{Bases y dimensiones}
\subsubsection{Definición de base}
\hspace{3mm}
Sea $K$ un cuerpo, sea $V$ un $K$-espacio vectorial. Diremos que una familia de vectores de $V$ es una "base" de $V$
si es simultáneamente un sistema generador y libre.

\vspace{2mm}
Cabe preguntarse qué propiedades tienen las bases. El objeto de esta sección es justificar por qué es tan importante el concepto
de base en el algebra lineal.


\subsubsection{Tamaño de las bases}
\hspace{3mm}
Sea $K$ un cuerpo, sea $V$ un $K$-espacio vectorial, veamos que todas las bases de $V$ tienen el mismo tamaño.
Para ello, basta demostar el siguiente resultado. Sea $\{v_1, \ldots, v_n\} $ una familia libre de vectores, sea
$\{w_1, \ldots, w_m\} $ un sistema generador de $V$; entonces, $n \leq m$.

\vspace{2mm}
\dem Veamos que es así por reducción al absurdo.

\vspace{2mm} 
Supongamos $n > m$. Por ser $\{w_1, \ldots, w_m\} $ sistema generador de $V$, su clausura es el propio $V$.
De este modo, $v_i \in K\left<w_1, \ldots, w_m \right> \hspace{2mm} \forall i \in \{1, \ldots, n\}$. En particualar,
podemos escribir $v_1 = \lambda_1^1 w_1 + \ldots + \lambda_m^1 w_m$. Al ser la familia de los $v_i$ libre por hipótesis,
ninguno de ellos es el $0_V$; por tanto, debe haber cierto $\lambda_j^1 \neq 0$. Al escoger este escalar no nulo podemos tomar
su inverso y poder despejar de la siguiente forma: \vspace{-3mm}
$$w_j = (-\lambda_j^1)^{-1} \big(-v_1 + \lambda_1^1w_1 +
\ldots + \lambda_{j-1}^1 w_{j-1} + \lambda_{j+1}^1 w_{j+1} + \ldots + \lambda_m^1 w_m\big)$$

En vista de esta expresión, es claro que $w_j$ pertenece a la clausura de la familia
$\{v_1\}\cup\big\{w_i : i \in \{1, \ldots, m\}\backslash\{j\} \big\}$. Por tanto, la clausura de
$w_j$ estará contenida en la de esta familia. Además, por definición de clausura, es claro que: 
$$K\left<F\right> \subseteq K\left<G\right> \Rightarrow
    K\left<F \cup H\right> \subseteq K\left<G \cup H\right>
\text{ para }F,G,H \text{ sistemas de vectores}$$
\newpage
Aplicándolo a este caso en particular, tenemos que: \vspace{-3mm}
$$V = K\left<w_1, \ldots, w_m\right> \subseteq
K\left<v_1, w_1, \ldots w_{j-1}, w_{j+1}, \ldots, w_m\right>$$
Pero al ser estar este último sistema de vectores contenido en el espacio vectorial $V$, su clausura
ha de estar contenida en el propio $V$. Por tanto, $\{v_1, w_1, \ldots w_{j-1}, w_{j+1}, \ldots, w_m\} $ es sistema generador de $V$.

\vspace{2mm}
Observemos que este procedimiento que hemos realizado con $v_1$ puede ser repetido para
$v_2, \ldots, v_m$. Como $v_2 \in V = K\left<v_1, w_1, \ldots w_{j-1}, w_{j+1}, \ldots, w_m\right>$,
podemos expresarlo como combinación lineal de vectores de este sistema. Además, recordemos que $\{v_i\}_{i=1}^n$
es una familia de vectores libres, luego $v_2 \neq 0_V$. Por tanto, al menos uno de los escalares de la combinación
lineal va a ser no nulo. Más aún, podemos asegurar que será un escalar asociado a cierto $w_k$
(con $k \in \{1, \ldots, m\}\backslash\{j\}  $ fijo), pues
de lo contrario $\{v_1, v_2\}$ no sería un sistema libre.
Al reemplazar $w_k$ por $v_2$ seguimos teniendo un sistema generador de $V$ (tal y como hicimos con $v_1$).

\vspace{2mm}
Reiterando, llegamos a que $\{v_1, \ldots, v_m\}$ es un sistema generador de $V$. Como hemos supuesto (hipótesis
del absurdo) $n > m$, tenemos que los vectores $\{v_{m+1}, \ldots, v_n\}$ no pertenecen a la clausura de $\{v_1, \ldots, v_m\}$,
pues de lo contrario $\{v_1, \ldots, v_n\}$ no sería un sistema libre. Esto supone una contradicción, pues $\{v_1, \ldots, v_m\}$
es un sistema generador, luego necesariamente ha de cumplirse: \vspace{-3mm}
$$\{v_{m+1}, \ldots, v_n\} \subseteq K\left<v_1, \ldots, v_m\right> = V \Rightarrow \{v_1, \ldots, v_n\} \text{ no es libre}$$
Se contradice la hipótesis, por lo que hemos llegado a una contradicción. Por todo lo anterior, concluimos que $n < m$.

\subsubsection{Concepto de dimensión}
\hspace{3mm}
En vista del resultado anterior, es evidente que todas las bases de un determinado espacio vectorial han de tener el mismo
número de vectores. Es precisamente este número a lo que denominamos "dimensión" de un espacio vectorial.

\newpage
En esta asignatura trabajaremos con espacios vectoriales de dimensión finita, aunque no todos son así. Veamos un ejemplo de espacio
vectorial de dimensión infinita para ilustrar por qué podría ocurrir esto (no se ahondará en espacios vectoriales de dimensión infinita
más allá de este caso).

\subsubsection{Base del espacio de polinomios}
\hspace{3mm}
Sea $K$ un cuerpo, \hyperref[polinomios-espacio-vectorial]{ya hemos comprobado} que los
polinomios en la variable $X$ con coeficientes en $K$ forman un $K$-espacio vectorial. Es sencillo
ver que podemos crear un sistema libre finito arbitrariamente grande, pero que no llegue a ser nunca 
sistema generador de $K[X]$.

\vspace{2mm}
Sea $n \in \mathbb{N}$ cualquiera pero fijo, podemos tomar el sistema $\{1,X,X^2, \ldots, X^n\}$.
Es evidente que este sistema es libre, pues ningún vector de él puede ser expresado como combinación
lineal del resto. No obstante, no es sistema generador de $K[X]$, peus el polinomio $X^{n+1}$ no
pertenece a la clausura de esta familia, pero sí a $K[X]$. Finalemnte, como el $n$ escogido es arbitrario,
podemos hacerlo tan grande como queramos, pero la familia resultante no será nunca un sistema generador.

\vspace{2mm}
Al restringirnos a $K_n[X]$ sí es posible dar una base, pues bastaría considerar el sistema
$\{1,X,\ldots,X^n\}$. Como consecuencia, $K_n[X]$ es un $K$-espacio vectorial de dimensión $n+1$.
\end{document}