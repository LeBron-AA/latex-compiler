\documentclass[12pt,a4paper]{article}

% Paquetes básicos
\usepackage[table,x11names]{xcolor}
\usepackage[utf8]{inputenc}
\usepackage{mathtools, amssymb, amsthm}
\usepackage{changepage}
\usepackage{geometry}
\usepackage[colorlinks=true]{hyperref}
\usepackage{enumitem}
\usepackage{etoolbox}
\usepackage{graphicx}
\usepackage{setspace}
\usepackage{tcolorbox}
\tcbuselibrary{skins, breakable}
\usepackage{titlesec}
\usepackage{tikz} % core TikZ
\usetikzlibrary{matrix}
\usepackage{pgfplots}
\usepackage{float}
\pgfplotsset{compat=newest}
\usepgfplotslibrary{fillbetween}

% Margins
%\geometry{left=3cm,right=3cm,top=2.5cm,bottom=2.5cm}

% Custom operators
\newcommand{\card}{\operatorname{card}}
\newcommand{\muae}{\overset{\mu-a.e.}{=}}

% Implication box setup
\tcbset{Implication-number/.style={
  enhanced,
  boxsep=2pt,
  colback=white,
  frame hidden,
  sharp corners,
  left=2pt, right=2pt, top=1pt, bottom=1pt,
  underlay={
    \draw[line width=0.5pt] (frame.south west) -- ([xshift=-133mm]frame.south east); % línea horizontal
    \draw[line width=0.5pt] ([xshift=-133mm]frame.north east) -- ([xshift=-133mm]frame.south east); % línea vertical
  }
}}

\tcbset{Implication-number-ds/.style={
  enhanced,
  boxsep=2pt,
  colback=white,
  frame hidden,
  sharp corners,
  left=2pt, right=2pt, top=1pt, bottom=1pt,
  underlay={
    \draw[line width=0.5pt] ([yshift=5mm]frame.south west) -- ([xshift=-133mm, yshift=5mm]frame.south east); % línea horizontal
    \draw[line width=0.5pt] ([xshift=-133mm, yshift=-3mm]frame.north east) -- ([xshift=-133mm, yshift=5mm]frame.south east); % línea vertical
  }
}}

\tcbset{Subset-contingency/.style={
  enhanced,
  boxsep=2pt,
  colback=white,
  frame hidden,
  sharp corners,
  left=2pt, right=2pt, top=1pt, bottom=1pt,
  underlay={
    \draw[line width=0.5pt] (frame.south west) -- ([xshift=-140mm]frame.south east); % línea horizontal
    \draw[line width=0.5pt] ([xshift=-140mm]frame.north east) -- ([xshift=-140mm]frame.south east); % línea vertical
  }
}}

\tcbset{Indent-subset-contingency/.style={
  enhanced,
  boxsep=2pt,
  colback=white,
  frame hidden,
  sharp corners,
  left=2pt, right=2pt, top=1pt, bottom=1pt,
  underlay={
    \draw[line width=0.5pt] (frame.south west) -- ([xshift=-140mm+0.07\textwidth]frame.south east); % línea horizontal
    \draw[line width=0.5pt] ([xshift=-140mm+0.07\textwidth]frame.north east) -- ([xshift=-140mm+0.07\textwidth]frame.south east); % línea vertical
  }
}}

% Useful commands
\renewcommand{\contentsname}{Contenidos}

\newcommand{\R}{\mathbb{R}}
\newcommand{\N}{\mathbb{N}}
\newcommand{\Z}{\mathbb{Z}}
\newcommand{\Q}{\mathbb{Q}}
\newcommand{\C}{\mathbb{C}}

\newcommand{\smallcup}{\mathop{\cup}\limits}
\newcommand{\smallcap}{\mathop{\cap}\limits}
\newcommand{\smallsum}{\mathop{\sum}\limits}
\newcommand{\smallprod}{\mathop{\prod}\limits}

\newcommand{\linf}[1]{\displaystyle{\mathop{\underline{\lim}}_{#1}}}
\newcommand{\mlim}[1]{\displaystyle{\lim_{#1}}}

%Integral de Lebesgue con patas
\newcommand{\lbint}{\mathop{\int_{\!\!\!\!\!|\!}^{\!|\!}}}

%Espacios \mathcal{L}_p
\newcommand{\elep}[2]{\mathcal{L}_{#1}(#2)}

% ----- Custom counters and counter commands -----
% Custom counter hierarchy
\newcounter{unit}[section]
\newcounter{chapter}[unit]
\makeatletter
\@addtoreset{subsubsection}{chapter}
\makeatother

\renewcommand{\theunit}{\arabic{unit}}
\renewcommand{\thechapter}{\arabic{chapter}}
\renewcommand{\thesubsubsection}{\arabic{subsubsection}.}

% Custom content hierarchy behavior
\newcommand{\chapter}[1]{
    \refstepcounter{chapter}
    \subsection*{\Large{\S \thechapter. #1}}
    \addcontentsline{toc}{subsection}{\thechapter. #1}
}
\newcommand{\unit}[1]{
    \refstepcounter{unit}
    \section*{\Huge{\Roman{unit} #1}}
    \addcontentsline{toc}{section}{\Roman{unit} #1}
}

\newcommand{\result}[1]{%
  \subsubsection{#1}%
  \label{result:\thesubsubsection}
}
  
\titleformat{\subsubsection}
    {\normalfont\large\bfseries} % mismo tamaño que \subsection
    {\thesubsubsection}{1em}{}
  
%---- Custom proof commands-----
\newcommand{\dem}{
    \noindent \underline{\textbf{Demostración:}}
}
\newcommand{\nota}{
    \noindent \underline{\textbf{Nota:}}
}
% ----------------------------------------
\hbadness=10000
\vbadness=10000
\hfuzz=100pt
\vfuzz=100pt
% ---------------------------------------
\title{Algoritmia}
\author{Práctica 1.2}
\date{15/2/2026}


\begin{document}

\maketitle
\hypersetup{linkcolor=black}
\vspace{4mm}
\tableofcontents
\hypersetup{linkcolor=Ivory4}
\newpage

\result{Bucle 1}
\hspace{3mm} El método \texttt{bucle1} tiene dos partes anidadas. La primera de ellas recorre $n^2$ elementos, multiplicando el contador de índice por 3 en cada paso. Así, la complejidad teórica esperada es de $O(\log n^2) = O(\log n)$. La segunda abarca $2n$ elementos saltando de 3 en 3, luego la complejidad de esta parte es $O(n)$. En total, \texttt{bucle1} es $O(n\log n)$.

\begin{table}[H]
\centering
\renewcommand{\arraystretch}{1.2}
\begin{tabular}{|r|r|r|r|}
\hline
\rowcolor{LightSteelBlue2}
\textit{n} & \textit{T empírico (ms)} & \textit{Repeticiones} & \textit{T teórico (ms)} \\
\hline
100     & 51        & 10.000 & - \\ \hline
200     & 99        & 10.000 & 117,35 \\ \hline
400     & 192       & 10.000 & 223,90 \\ \hline
800     & 458       & 10.000 & 428,42 \\ \hline
1.600   & 993       & 10.000 & 1.010,98 \\ \hline
3.200   & 2.111     & 10.000 & 2.172,59 \\ \hline
6.400   & 4.793     & 10.000 & 4.584,59 \\ \hline
12.800  & 10.285    & 10.000 & 10.344,15 \\ \hline
25.600  & 23.169    & 10.000 & 22.077,64 \\ \hline
51.200  & 49.114    & 10.000 & 49.502,33 \\ \hline
\end{tabular}
\end{table}


\begin{center}
  \includegraphics[
    width=0.9\textwidth,
    trim=0 3cm 0 3cm,
    clip
  ]{media/Bucle1.pdf}
  \newline
  Comparación de tiempos empíricos y teóricos de \texttt{bucle1}
\end{center}

\newpage
\result{Bucle 2}
\hspace{3mm} El método \texttt{bucle2} contiene un bucle externo que divide su variable de control entre 3 en cada iteración hasta llegar a 1, lo que supone $O(\log n)$. En cada una de esas iteraciones se ejecutan dos bucles anidados: el primero recorre $n$ elementos y el segundo aproximadamente $n/2$, por lo que juntos aportan una complejidad $O(n^2)$. En consecuencia, la complejidad total de \texttt{bucle2} es $O(n^2\log n)$.

\begin{table}[H]
\centering
\renewcommand{\arraystretch}{1.2}
\begin{tabular}{|r|r|r|r|r|r|}
\hline
\rowcolor{LightSteelBlue2}
\textit{n} & \textit{$T_\text{emp}$ total (ms)} & \textit{Repeticiones} & \textit{$T_\text{teo}$ total (ms)} & \textit{$\overline{T_\text{emp}}$ (ms)} & \textit{$\overline{T_\text{teo}}$ (ms)} \\
\hline
100     & 54       & 10.000 & -           & 0,0054    & - \\ \hline
200     & 151      & 10.000 & 248,5112    & 0,0151    & 0,0249 \\ \hline
400     & 694      & 10.000 & 683,0177    & 0,0694    & 0,0683 \\ \hline
800     & 3.159    & 10.000 & 3.097,1530  & 0,3159    & 0,3097 \\ \hline
1.600   & 10.200   & 10.000 & 13.946,2643 & 1,0200    & 1,3946 \\ \hline
3.200   & 40.955   & 10.000 & 44.633,1972 & 4,0955    & 4,4633 \\ \hline
6.400   & 156      & 10     & 177,8892    & 15,6000   & 17,7889 \\ \hline
12.800  & 663      & 10     & 673,3520    & 66,3000   & 67,3352 \\ \hline
25.600  & 2.897    & 10     & 2.846,3732  & 289,7000  & 284,6373 \\ \hline
51.200  & 11.410   & 10     & 12.379,3216 & 1.141,0000 & 1.237,9322 \\ \hline
\end{tabular}
\end{table}
Tiempos empíricos y teóricos totales (tiempo de todas las repeticiones) y medios (entre el número de repeticiones)

\begin{center}
  \includegraphics[
    width=0.9\textwidth,
    trim=0 3cm 0 3cm,
    clip
  ]{media/Bucle2.pdf}
  \newline
  Las repeticiones, luego  hemos de graficar usando los tiempos medios
\end{center}

\newpage
\result{Bucle 3}
\hspace{3mm} El método \texttt{bucle3} tiene su bucle más externo que recorre $2n$ elementos, por lo que su complejidad es $O(n)$. En cada iteración, un segundo bucle recorre desde el valor actual (\texttt{i}) hasta 0 decrementando de 2 en 2, lo que supone una complejidad $O(n)$. Además, existe un tercer bucle que duplica su variable hasta alcanzar $n$, aportando una capa más de complejidad $O(\log n)$. Así, se obtiene una complejidad total $O(n^2\log n)$.

\begin{table}[H]
\centering
\renewcommand{\arraystretch}{1.2}
\begin{tabular}{|r|r|r|r|r|r|}
\hline
\rowcolor{LightSteelBlue2}
\textit{n} & \textit{$T_\text{emp}$ total (ms)} & \textit{Repeticiones} & \textit{$T_\text{teo}$ total (ms)} & \textit{$\overline{T_\text{emp}}$ (ms)} & \textit{$\overline{T_\text{teo}}$ (ms)} \\
\hline
100      & 82      & 100   & -           & 0,82       & - \\ \hline
200      & 350     & 100   & 377,3689    & 3,50       & 3,7737 \\ \hline
400      & 1.498   & 100   & 1.583,1536  & 14,98      & 15,8315 \\ \hline
800      & 6.422   & 100   & 6.685,2091  & 64,22      & 66,8521 \\ \hline
1.600    & 26.802  & 100   & 28.351,6648 & 268,02     & 283,5166 \\ \hline
3.200    & 11.298  & 10    & 11.728,0289 & 1.129,80   & 1.172,8029 \\ \hline
6.400    & 47.576  & 10    & 49.073,1884 & 4.757,60   & 4.907,3188 \\ \hline
12.800   & 19.801  & 1     & 20.535,5104 & 19.801,00  & 20.535,5104 \\ \hline
25.600   & 83.367  & 1     & 85.009,1037 & 83.367,00  & 85.009,1037 \\ \hline
51.200   & Fdt     & -     & -           & -          & - \\ \hline
\end{tabular}
\end{table}

\begin{center}
  \includegraphics[
    width=0.9\textwidth,
    trim=0 3cm 0 3cm,
    clip
  ]{media/Bucle3.pdf}
  \newline
  Una vez más, comparamos las medias en igualdad de unidades
\end{center}

\newpage
\result{Bucle 4}
\hspace{3mm} El método \texttt{bucle4} consta de tres bucles anidados donde el primero recorre $n$ elementos. El segundo bucle, en la iteración $i$-ésima del primero recorre $i$ elementos. Análogamente, el tercer bucle recorre $j$ (índice del segundo) elementos. En total, la complejidad del método es de $O(n^3)$.

\begin{table}[H]
\centering
\renewcommand{\arraystretch}{1.2}
\begin{tabular}{|r|r|r|r|r|r|}
\hline
\rowcolor{LightSteelBlue2}
\textit{n} & \textit{$T_\text{emp}$ total (ms)} & \textit{Repeticiones} & \textit{$T_\text{teo}$ total (ms)} & \textit{$\overline{T_\text{emp}}$ (ms)} & \textit{$\overline{T_\text{teo}}$ (ms)} \\
\hline
100      & 70       & 100   & -           & 0,70       & - \\ \hline
200      & 478      & 100   & 560,00      & 4,78       & 5,60 \\ \hline
400      & 3.560    & 100   & 3.824,00    & 35,60      & 38,24 \\ \hline
800      & 28.111   & 100   & 28.480,00   & 281,11     & 284,80 \\ \hline
1.600    & 2.325    & 1     & 2.248,88    & 2.325,00   & 2.248,88 \\ \hline
3.200    & 18.583   & 1     & 18.600,00   & 18.583,00  & 18.600,00 \\ \hline
6.400    & 149.392  & 1     & 148.664,00  & 149.392,00 & 148.664,00 \\ \hline
12.800   & FdT      & -     & -           & -          & - \\ \hline
\end{tabular}
\end{table}

\begin{center}
  \includegraphics[
    width=\textwidth,
    trim=0 3cm 0 3cm,
    clip
  ]{media/Bucle4.pdf}
\end{center}

\newpage
\result{Comparativas entre bucles}
\hspace{3mm} Cociente \texttt{bucle1} y \texttt{bucle2}
\begin{table}[H]
\centering
\renewcommand{\arraystretch}{1.2}
\begin{tabular}{|r|r|r|r|}
\hline
\rowcolor{LightSteelBlue2}
\textit{n} & \textit{$T_1$ (ms)} & \textit{$T_2$ (ms)} & \textit{$T_1 / T_2$} \\
\hline
100      & 0,0051     & 0,0054     & 0,9444 \\ \hline
200      & 0,0099     & 0,0151     & 0,6556 \\ \hline
400      & 0,0192     & 0,0694     & 0,2767 \\ \hline
800      & 0,0458     & 0,3159     & 0,1450 \\ \hline
1.600    & 0,0993     & 1,0200     & 0,0974 \\ \hline
3.200    & 0,2111     & 4,0955     & 0,0515 \\ \hline
6.400    & 0,4793     & 15,6000    & 0,0307 \\ \hline
12.800   & 1,0285     & 66,3000    & 0,0155 \\ \hline
25.600   & 2,3169     & 289,7000   & 0,0080 \\ \hline
51.200   & 4,9114     & 1.141,0000 & 0,0043 \\ \hline
\end{tabular}
\end{table}
Los cocientes van decreciendo claramente en cada iteración, en línea con el resultado que esperábamos por sus complejidades teóricas.


\vspace{6mm} Cociente \texttt{bucle3} vs \texttt{bucle1}
\begin{table}[H]
\centering
\renewcommand{\arraystretch}{1.2}
\begin{tabular}{|r|r|r|r|}
\hline
\rowcolor{LightSteelBlue2}
\textit{n} & \textit{$T_3$ (ms)} & \textit{$T_2$ (ms)} & \textit{$T_3 / T_2$} \\
\hline
100      & 0,82      & 0,0054    & 151,8519 \\ \hline
200      & 3,50      & 0,0151    & 231,7881 \\ \hline
400      & 14,98     & 0,0694    & 215,8501 \\ \hline
800      & 64,22     & 0,3159    & 203,2922 \\ \hline
1.600    & 268,02    & 1,0200    & 262,7647 \\ \hline
3.200    & 1.129,8   & 4,0955    & 275,8638 \\ \hline
6.400    & 4.757,6   & 15,6000   & 304,9744 \\ \hline
12.800   & 19.801    & 66,3000   & 298,6576 \\ \hline
25.600   & 83.367    & 289,7000  & 287,7701 \\ \hline
51.200   & -         & 1.141,0000 & - \\ \hline
\end{tabular}
\end{table}
Los cocientes se estabilizan para $n$ grande, en línea con lo que esperábamos (mismas complejidades teóricas).

\newpage
\result{Bucle 5}
\hspace{3mm} Para lograr un bucle de complejidad $O(n^2 \log^2 n)$ de manera sencilla, hemos optado por usar dos bucles de complejidad $O(n)$ y otros  dos de complejidad $O(\log n)$, todos ellos anidados.

\begin{table}[H]
\centering
\renewcommand{\arraystretch}{1.2}
\begin{tabular}{|r|r|r|r|r|r|}
\hline
\rowcolor{LightSteelBlue2}
\textit{n} & \textit{$T_\text{emp}$ total (ms)} & \textit{Repeticiones} & \textit{$T_\text{teo}$ total (ms)} & \textit{$\overline{T_\text{emp}}$ (ms)} & \textit{$\overline{T_\text{teo}}$ (ms)} \\
\hline
100      & 699       & 100   & -           & 6,99       & - \\ \hline
200      & 3.181     & 100   & 3.701,0226  & 31,81      & 37,0102 \\ \hline
400      & 15.338    & 100   & 16.270,980  & 153,38     & 162,7098 \\ \hline
800      & 848       & 1     & 763,6865    & 848,00     & 763,6865 \\ \hline
1.600    & 3.073     & 1     & 4.131,9246  & 3.073,00   & 4.131,9246 \\ \hline
3.200    & 17.538    & 1     & 14.710,1879 & 17.538,00  & 14.710,1879 \\ \hline
6.400    & 72.779    & 1     & 82.719,0401 & 72.779,00  & 82.719,0401 \\ \hline
12.800   & FdT       & -     & -           & -          & - \\ \hline
\end{tabular}
\end{table}


\begin{center}
  \includegraphics[
    width=0.9\textwidth,
    trim=0 3cm 0 3cm,
    clip
  ]{media/Bucle5.pdf}
\end{center}

\newpage
\result{Bucle 6}
\hspace{3mm} Para lograr un bucle de complejidad $O(n^3 \log n)$ usamos tres bucles de complejidad $O(n)$ y otro de complejidad $\log n$, todos ellos anidados.

\begin{table}[H]
\centering
\renewcommand{\arraystretch}{1.2}
\begin{tabular}{|r|r|r|r|}
\hline
\rowcolor{LightSteelBlue2}
\textit{n} & \textit{$T_\text{emp}$ (ms)} & \textit{Repeticiones} & \textit{$T_\text{teo}$ (ms)} \\
\hline
100      & 66        & 1     & - \\ \hline
200      & 429       & 1     & 607,4719 \\ \hline
400      & 3.880     & 1     & 3.880,9880 \\ \hline
800      & 33.469    & 1     & 34.630,9899 \\ \hline
1.600    & FdT       & -     & - \\ \hline
\end{tabular}
\end{table}


\begin{center}
  \includegraphics[
    width=0.9\textwidth,
    trim=0 3cm 0 3cm,
    clip
  ]{media/Bucle6.pdf}
\end{center}


\newpage
\result{Bucle 7}
\hspace{3mm} Para lograr un bucle de complejidad $O(n^4)$ podemos usar 4 bucles anidados de complejidad $O(n)$.

\begin{table}[H]
\centering
\renewcommand{\arraystretch}{1.2}
\begin{tabular}{|r|r|r|r|}
\hline
\rowcolor{LightSteelBlue2}
\textit{n} & \textit{$T_\text{emp}$ (ms)} & \textit{Repeticiones} & \textit{$T_\text{teo}$ (ms)} \\
\hline
100      & 85       & 1     & - \\ \hline
200      & 1.151    & 1     & 1.360 \\ \hline
400      & 17.339   & 1     & 18.416 \\ \hline
800      & 261.464  & 1     & 277.424 \\ \hline
1.600    & FdT      & -     & - \\ \hline
\end{tabular}
\end{table}


\begin{center}
  \includegraphics[
    width=0.9\textwidth,
    trim=0 3cm 0 3cm,
    clip
  ]{media/Bucle7.pdf}
\end{center}

\newpage
\result{Python vs Java (Bucle4)}
\begin{table}[H]
\centering
\renewcommand{\arraystretch}{1.2}
\begin{tabular}{|r|r|r|r|r|r|}
\hline
\rowcolor{LightSteelBlue2}
\textit{n} & \textit{$T_\text{Python}$ (ms)} & \textit{$T_\text{Java total}$ (ms)} & \textit{Repeticiones Java} & \textit{$T_\text{Java medio}$ (ms)} & \textit{$T_\text{Py} / T_\text{J}$} \\
\hline
200      & 113       & 468       & 100   & 4,68       & 24,1453 \\ \hline
400      & 800       & 234       & 1     & 234,00     & 3,4188 \\ \hline
800      & 7.985     & 1.569     & 1     & 1.569,00   & 5,0892 \\ \hline
1.600    & 69.665    & 14.186    & 1     & 14.186,00  & 4,9108 \\ \hline
3.200    & FdT       & 110.528   & 1     & 110.528,00 & - \\ \hline
6.400    & FdT       & FdT       & -     & -          & - \\ \hline
\end{tabular}
\end{table}
El cociente de tiempos se va estabilizando.

\vspace{6mm}
\result{Condiciones de medida}
\hspace{3mm} Todas las medidas en Java han sido realizadas con Eclipse IDE usando la opción \texttt{-Xint} como argumento para la JVM.

\vspace{2mm} Los apartados 1 al 5 de esta memoria han usado las siguientes especificaciones:
[CPU 12th Gen Intel(R) Core(TM) i5-12400 @ 2,5GHz + 16GB RAM]
, y el resto
[CPU: Intel Core i7-6700HQ CPU @ 2,60GHz + 16GB RAM].
\end{document} 