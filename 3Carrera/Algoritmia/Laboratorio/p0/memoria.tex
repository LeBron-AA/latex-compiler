\documentclass[12pt,a4paper]{article}

% Paquetes básicos
\usepackage[table,x11names]{xcolor}
\usepackage[utf8]{inputenc}
\usepackage{mathtools, amssymb, amsthm}
\usepackage{changepage}
\usepackage{geometry}
\usepackage[colorlinks=true]{hyperref}
\usepackage{enumitem}
\usepackage{etoolbox}
\usepackage{graphicx}
\usepackage{setspace}
\usepackage{tcolorbox}
\tcbuselibrary{skins, breakable}
\usepackage{titlesec}
\usepackage{tikz} % core TikZ
\usetikzlibrary{matrix}
\usepackage{pgfplots}
\usepackage{float}

\pgfplotsset{compat=newest}
\usepgfplotslibrary{fillbetween}

% Margins
%\geometry{left=3cm,right=3cm,top=2.5cm,bottom=2.5cm}

% Custom operators
\newcommand{\card}{\operatorname{card}}
\newcommand{\muae}{\overset{\mu-a.e.}{=}}

% Implication box setup
\tcbset{Implication-number/.style={
  enhanced,
  boxsep=2pt,
  colback=white,
  frame hidden,
  sharp corners,
  left=2pt, right=2pt, top=1pt, bottom=1pt,
  underlay={
    \draw[line width=0.5pt] (frame.south west) -- ([xshift=-133mm]frame.south east); % línea horizontal
    \draw[line width=0.5pt] ([xshift=-133mm]frame.north east) -- ([xshift=-133mm]frame.south east); % línea vertical
  }
}}

\tcbset{Implication-number-ds/.style={
  enhanced,
  boxsep=2pt,
  colback=white,
  frame hidden,
  sharp corners,
  left=2pt, right=2pt, top=1pt, bottom=1pt,
  underlay={
    \draw[line width=0.5pt] ([yshift=5mm]frame.south west) -- ([xshift=-133mm, yshift=5mm]frame.south east); % línea horizontal
    \draw[line width=0.5pt] ([xshift=-133mm, yshift=-3mm]frame.north east) -- ([xshift=-133mm, yshift=5mm]frame.south east); % línea vertical
  }
}}

\tcbset{Subset-contingency/.style={
  enhanced,
  boxsep=2pt,
  colback=white,
  frame hidden,
  sharp corners,
  left=2pt, right=2pt, top=1pt, bottom=1pt,
  underlay={
    \draw[line width=0.5pt] (frame.south west) -- ([xshift=-140mm]frame.south east); % línea horizontal
    \draw[line width=0.5pt] ([xshift=-140mm]frame.north east) -- ([xshift=-140mm]frame.south east); % línea vertical
  }
}}

\tcbset{Indent-subset-contingency/.style={
  enhanced,
  boxsep=2pt,
  colback=white,
  frame hidden,
  sharp corners,
  left=2pt, right=2pt, top=1pt, bottom=1pt,
  underlay={
    \draw[line width=0.5pt] (frame.south west) -- ([xshift=-140mm+0.07\textwidth]frame.south east); % línea horizontal
    \draw[line width=0.5pt] ([xshift=-140mm+0.07\textwidth]frame.north east) -- ([xshift=-140mm+0.07\textwidth]frame.south east); % línea vertical
  }
}}

% Useful commands
\renewcommand{\contentsname}{Contenidos}

\newcommand{\R}{\mathbb{R}}
\newcommand{\N}{\mathbb{N}}
\newcommand{\Z}{\mathbb{Z}}
\newcommand{\Q}{\mathbb{Q}}
\newcommand{\C}{\mathbb{C}}

\newcommand{\smallcup}{\mathop{\cup}\limits}
\newcommand{\smallcap}{\mathop{\cap}\limits}
\newcommand{\smallsum}{\mathop{\sum}\limits}
\newcommand{\smallprod}{\mathop{\prod}\limits}

\newcommand{\linf}[1]{\displaystyle{\mathop{\underline{\lim}}_{#1}}}
\newcommand{\mlim}[1]{\displaystyle{\lim_{#1}}}

%Integral de Lebesgue con patas
\newcommand{\lbint}{\mathop{\int_{\!\!\!\!\!|\!}^{\!|\!}}}

%Espacios \mathcal{L}_p
\newcommand{\elep}[2]{\mathcal{L}_{#1}(#2)}

% ----- Custom counters and counter commands -----
% Custom counter hierarchy
\newcounter{unit}[section]
\newcounter{chapter}[unit]
\makeatletter
\@addtoreset{subsubsection}{chapter}
\makeatother

\renewcommand{\theunit}{\arabic{unit}}
\renewcommand{\thechapter}{\arabic{chapter}}
\renewcommand{\thesubsubsection}{\arabic{subsubsection}.}

% Custom content hierarchy behavior
\newcommand{\chapter}[1]{
    \refstepcounter{chapter}
    \subsection*{\Large{\S \thechapter. #1}}
    \addcontentsline{toc}{subsection}{\thechapter. #1}
}
\newcommand{\unit}[1]{
    \refstepcounter{unit}
    \section*{\Huge{\Roman{unit} #1}}
    \addcontentsline{toc}{section}{\Roman{unit} #1}
}

\newcommand{\result}[1]{%
  \subsubsection{#1}%
  \label{result:\thesubsubsection}
}
  
\titleformat{\subsubsection}
    {\normalfont\large\bfseries} % mismo tamaño que \subsection
    {\thesubsubsection}{1em}{}
  
%---- Custom proof commands-----
\newcommand{\dem}{
    \noindent \underline{\textbf{Demostración:}}
}
\newcommand{\nota}{
    \noindent \underline{\textbf{Nota:}}
}
% ----------------------------------------
\hbadness=10000
\vbadness=10000
\hfuzz=100pt
\vfuzz=100pt
% ---------------------------------------
\title{Algoritmia}
\author{Práctica 0}
\date{8/2/2026}


\begin{document}

\maketitle
\hypersetup{linkcolor=black}
\vspace{4mm}
\tableofcontents
\hypersetup{linkcolor=Ivory4}
\newpage

\result{Toma de tiempos Python A1 en dos ordenadores distintos}

\begin{table}[h!]
\centering
\renewcommand{\arraystretch}{1.2}
\begin{tabular}{|r|c|c|c|}
\hline
\rowcolor{LightSteelBlue2}
\textit{n} & $T_1$ (ms) & $T_2$ (ms) & \textit{Cociente }$T_1$/$T_2$ \\
\hline
5.000      & 387    & 1.838   & 0,210554951 \\ \hline
10.000     & 1.464  & 6.496   & 0,225369458 \\ \hline
20.000     & 5.619  & 23.515  & 0,238953859 \\ \hline
40.000     & 22.840 & 92.309  & 0,247429828 \\ \hline
80.000     & 91.608 & FdT     & - \\ \hline
160.000    & FdT    & FdT     & - \\ \hline
\end{tabular}
\end{table}
El cociente de los tiempos se mantiene aun habiendo incrementado $n$ mucho, en línea con el principio de invarianza visto en teoría.

\begin{center}
  \includegraphics[
    width=0.95\textwidth,
    trim=2cm 3cm 2cm 5cm,
    clip
  ]{media/PythonA1.pdf}
  \\[-4ex]
  Comparativa gráfica entre ambos ordenadores
\end{center}
Los tiempos mostrados se corresponden con las siguientes configuraciones de hardware:
\begin{itemize}
  \item $T_1$: CPU 12th Gen Intel(R) Core(TM) i5-12400 @ 2,5GHz + 16GB RAM\\
  \item $T_2$: CPU: Intel Core i7-6700HQ CPU @ 2,60GHz + 16GB RAM
\end{itemize}
El resto de tomas de tiempos de esta memoria se realizarán con la configuración ``$T_2$'' de este apartado.

\newpage
\result{Tiempos en Java con y sin optimización para A2}
\begin{table}[h!]
\centering
\renewcommand{\arraystretch}{1.2}
\begin{tabular}{|r|r|r|}
\hline
\rowcolor{LightSteelBlue2}
\textit{n} & \textit{$T_\text{Optim}(\text{ms})$} & \textit{$T_\text{-Xint}(\text{ms})$} \\
\hline
5.000     & 66    & 68    \\ \hline
10.000    & 219   & 221   \\ \hline
20.000    & 698   & 886   \\ \hline
40.000    & 1.319 & 1.719 \\ \hline
80.000    & 4.267 & 10.860 \\ \hline
160.000   & 26.851 & 57.587 \\ \hline
320.000   & FdT   & FdT   \\ \hline
\end{tabular}
\end{table}

\begin{center}
  \includegraphics[
    width=0.95\textwidth,
    trim=2cm 3cm 2cm 5cm,
    clip
  ]{media/JavaA2.pdf}
  \\[-4ex]
\end{center}
\newpage

\result{Tiempos en Java con y sin optimización para A3}
\begin{table}[h!]
\centering
\renewcommand{\arraystretch}{1.2}
\begin{tabular}{|r|r|r|}
\hline
\rowcolor{LightSteelBlue2}
\textit{n} & \textit{$T_\text{Optim}(\text{ms})$} & \textit{$T_\text{-Xint}(\text{ms})$} \\
\hline
5.000     & 48     & 50      \\ \hline
10.000    & 113    & 169     \\ \hline
20.000    & 663    & 212     \\ \hline
40.000    & 1.734  & 1.406   \\ \hline
80.000    & 8.588  & 8.489   \\ \hline
160.000   & 24.114 & 41.085  \\ \hline
320.000   & 64.550 & 163.845 \\ \hline
640.000   & FdT    & FdT     \\ \hline
\end{tabular}
\end{table} 

\begin{center}
  \includegraphics[
    width=0.95\textwidth,
    trim=2cm 3cm 2cm 5cm,
    clip
  ]{media/JavaA3.pdf}
  \\[-4ex]
\end{center}

\newpage

\result{Tiempos en Java con y sin optimización para A4}
\begin{table}[h!]
\centering
\renewcommand{\arraystretch}{1.2}
\begin{tabular}{|r|r|r|}
\hline
\rowcolor{LightSteelBlue2}
\textit{n} & \textit{$T_\text{Optim}(\text{ms})$} & \textit{$T_\text{-Xint}(\text{ms})$} \\
\hline
5.000       & 43     & 109    \\ \hline
10.000      & 1      & 1      \\ \hline
20.000      & 1      & 1      \\ \hline
40.000      & 4      & 4      \\ \hline
80.000      & 12     & 11     \\ \hline
160.000     & 38     & 28     \\ \hline
320.000     & 75     & 55     \\ \hline
640.000     & 147    & 128    \\ \hline
1.280.000   & 292    & 146    \\ \hline
2.560.000   & 448    & 321    \\ \hline
5.120.000   & 1.064  & 903    \\ \hline
10.240.000  & 2.216  & 1.914  \\ \hline
20.480.000  & 3.113  & 3.934  \\ \hline
40.960.000  & 8.503  & 6.835  \\ \hline
81.920.000  & 14.329 & 15.591 \\ \hline
163.840.000 & 31.754 & 30.892 \\ \hline
\end{tabular}
\end{table}
La optimización no aporta cambios tan grandes o incluso empeora el rendimiento
\begin{center}
  \includegraphics[
    width=0.95\textwidth,
    trim=2cm 3cm 2cm 5cm,
    clip
  ]{media/JavaA4.pdf}
  \\[-4ex]
\end{center}

\newpage
\result{Comparación de tiempos con optimización}
\begin{table}[h!]
\centering
\renewcommand{\arraystretch}{1.2}
\begin{tabular}{|r|r|r|r|r|}
\hline
\rowcolor{LightSteelBlue2}
\textit{n} & \textit{JavaA1 (ms)} & \textit{JavaA2 (ms)} & \textit{JavaA3 (ms)} & \textit{JavaA4 (ms)} \\
\hline
5.000       & 543    & 66     & 48     & 43     \\ \hline
10.000      & 1.083  & 219    & 113    & 1      \\ \hline
20.000      & 3.895  & 698    & 663    & 1      \\ \hline
40.000      & 11.454 & 1.319  & 1.734  & 4      \\ \hline
80.000      & 49.932 & 4.267  & 8.588  & 12     \\ \hline
160.000     & FdT    & 26.851 & 24.114 & 38     \\ \hline
320.000     & FdT    & FdT    & 64.550 & 75     \\ \hline
640.000     & FdT    & FdT    & FdT    & 147    \\ \hline
1.280.000   & FdT    & FdT    & FdT    & 292    \\ \hline
2.560.000   & FdT    & FdT    & FdT    & 448    \\ \hline
5.120.000   & FdT    & FdT    & FdT    & 1.064  \\ \hline
10.240.000  & FdT    & FdT    & FdT    & 2.216  \\ \hline
20.480.000  & FdT    & FdT    & FdT    & 3.113  \\ \hline
40.960.000  & FdT    & FdT    & FdT    & 8.503  \\ \hline
81.920.000  & FdT    & FdT    & FdT    & 14.329 \\ \hline
163.840.000 & FdT    & FdT    & FdT    & 31.754 \\ \hline
\end{tabular}
\end{table}
\begin{center}
  \includegraphics[
    width=0.95\textwidth,
    trim=2cm 8.5cm 2cm 9.5cm,
    clip
  ]{media/Optim.pdf}
  \\[-4ex]
  Se han omitido datos de JavaA4 para evitar la distorsión de la gráfica
\end{center}

\newpage
\result{Comparación entre algoritmos usando -Xint}
\begin{table}[h!]
\centering
\renewcommand{\arraystretch}{1.2}
\begin{tabular}{|r|r|r|r|r|}
\hline
\rowcolor{LightSteelBlue2}
\textit{n} & \textit{JavaA1 (ms)} & \textit{JavaA2 (ms)} & \textit{JavaA3 (ms)} & \textit{JavaA4 (ms)} \\
\hline
5.000       & 278    & 68     & 50      & 109    \\ \hline
10.000      & 1.364  & 221    & 169     & 1      \\ \hline
20.000      & 5.860  & 886    & 212     & 1      \\ \hline
40.000      & 21.258 & 1.719  & 1.406   & 4      \\ \hline
80.000      & 66.728 & 10.860 & 8.489   & 11     \\ \hline
160.000     & FdT    & 57.587 & 41.085  & 28     \\ \hline
320.000     & FdT    & FdT    & FdT     & 55     \\ \hline
640.000     & FdT    & FdT    & FdT     & 128    \\ \hline
1.280.000   & FdT    & FdT    & FdT     & 146    \\ \hline
2.560.000   & FdT    & FdT    & FdT     & 321    \\ \hline
5.120.000   & FdT    & FdT    & FdT     & 903    \\ \hline
10.240.000  & FdT    & FdT    & FdT     & 1.914  \\ \hline
20.480.000  & FdT    & FdT    & FdT     & 3.934  \\ \hline
40.960.000  & FdT    & FdT    & FdT     & 6.835  \\ \hline
81.920.000  & FdT    & FdT    & FdT     & 15.591 \\ \hline
163.840.000 & FdT    & FdT    & FdT     & 30.892 \\ \hline
\end{tabular}
\end{table}

\begin{center}
  \includegraphics[
    width=0.95\textwidth,
    trim=2cm 3cm 2cm 5cm,
    clip
  ]{media/Xint.pdf}
  \\[-4ex]
  Se han omitido datos de JavaA4 para evitar la distorsión de la gráfica
\end{center}

\newpage
\result{Comparativa Java Python}
\begin{table}[h!]
\centering
\renewcommand{\arraystretch}{1.2}
\begin{tabular}{|r|r|r|}
\hline
\rowcolor{LightSteelBlue2}
\textit{n} & \textit{JavaA1 (ms)} & \textit{PythonA1 (ms)} \\
\hline
5.000   & 278    & 1.838  \\ \hline
10.000  & 1.364  & 6.496  \\ \hline
20.000  & 5.860  & 23.515 \\ \hline
40.000  & 21.258 & 92.309 \\ \hline
80.000  & 66.728 & FdT    \\ \hline
160.000 & FdT    & FdT    \\ \hline
\end{tabular}
\end{table}

\begin{table}[h!]
\centering
\renewcommand{\arraystretch}{1.2}
\begin{tabular}{|r|r|r|}
\hline
\rowcolor{LightSteelBlue2}
\textit{n} & \textit{JavaA2 (ms)} & \textit{PythonA2 (ms)} \\
\hline
5.000   & 68     & 148    \\ \hline
10.000  & 221    & 939    \\ \hline
20.000  & 886    & 2.994  \\ \hline
40.000  & 1.719  & 12.533 \\ \hline
80.000  & 10.860 & 44.993 \\ \hline
160.000 & 57.587 & FdT    \\ \hline
\end{tabular}
\end{table}


\begin{table}[h!]
\centering
\renewcommand{\arraystretch}{1.2}
\begin{tabular}{|r|r|r|}
\hline
\rowcolor{LightSteelBlue2}
\textit{n} & \textit{JavaA3 (ms)} & \textit{PythonA3 (ms)} \\
\hline
5.000   & 50     & 68     \\ \hline
10.000  & 169    & 281    \\ \hline
20.000  & 212    & 1.188  \\ \hline
40.000  & 1.406  & 5.828  \\ \hline
80.000  & 8.489  & 22.621 \\ \hline
160.000 & 41.085 & 80.348 \\ \hline
320.000 & FdT    & FdT    \\ \hline
\end{tabular}
\end{table}

\begin{table}[H]
\centering
\renewcommand{\arraystretch}{1.2}
\begin{tabular}{|r|r|r|}
\hline
\rowcolor{LightSteelBlue2}
\textit{n} & \textit{JavaA4 (ms)} & \textit{PythonA4 (ms)} \\
\hline
5.000       & 109    & 1      \\ \hline
10.000      & 1      & 8      \\ \hline
20.000      & 1      & 5      \\ \hline
40.000      & 4      & 24     \\ \hline
80.000      & 11     & 39     \\ \hline
160.000     & 28     & 72     \\ \hline
320.000     & 55     & 182    \\ \hline
640.000     & 128    & 413    \\ \hline
1.280.000   & 146    & 645    \\ \hline
2.560.000   & 321    & 1.416  \\ \hline
5.120.000   & 903    & 3.468  \\ \hline
10.240.000  & 1.914  & 6.657  \\ \hline
20.480.000  & 3.934  & 13.347 \\ \hline
40.960.000  & 6.835  & 30.003 \\ \hline
81.920.000  & 15.591 & 58.415 \\ \hline
163.840.000 & 30.892 & FdT    \\ \hline
\end{tabular}
\end{table}

\vspace{2mm}
\begin{center}
  \includegraphics[
    width=0.95\textwidth,
    trim=2cm 3cm 2cm 5cm,
    clip
  ]{media/PythonConjunto.pdf}
  \\[-4ex]
  Se han omitido datos de A3 y A4 para no distorsionar el eje de abscisas
\end{center}

\newpage
\result{Algortimo A5: Criba de Atkin}
\begin{table}[h!]
\centering
\renewcommand{\arraystretch}{1.2}
\begin{tabular}{|r|r|r|r|}
\hline
\rowcolor{LightSteelBlue2}
\textit{n} & \textit{Python (ms)} & \textit{-Xint (ms)} & \textit{Cociente Py/J} \\
\hline
5.000        & 3      & 99      & 0,0303 \\ \hline
10.000       & 6      & 5       & 1,2000 \\ \hline
20.000       & 13     & 10      & 1,3000 \\ \hline
40.000       & 34     & 18      & 1,8889 \\ \hline
80.000       & 82     & 36      & 2,2778 \\ \hline
160.000      & 153    & 72      & 2,1250 \\ \hline
320.000      & 442    & 140     & 3,1571 \\ \hline
640.000      & 799    & 193     & 4,1399 \\ \hline
1.280.000    & 1.552  & 403     & 3,8511 \\ \hline
2.560.000    & 3.544  & 894     & 3,9642 \\ \hline
5.120.000    & 5.938  & 1.891   & 3,1401 \\ \hline
10.240.000   & 13.359 & 3.736   & 3,5757 \\ \hline
20.480.000   & 27.652 & 7.588   & 3,6442 \\ \hline
40.960.000   & 53.071 & 15.416  & 3,4426 \\ \hline
81.920.000   & FdT    & 27.186  & -      \\ \hline
163.840.000  & FdT    & 53.810  & -      \\ \hline
327.680.000  & FdT    & FdT     & -      \\ \hline
\end{tabular}
\end{table}
Comparativa de tiempos entre Python y Java con \texttt{-Xint}. Podemos observar que los cocientes tienden a una constante no nula, en línea con el principio de invarianza visto en teoría.

\vspace{2mm} Puesto que el algoritmo A4 ya produce tiempos mucho menores que el resto, se ha optado por no mostrar A5 con los demás algoritmos, pues su grafo coincidiría estéticamente con el de A4, provocando ruido en la figura.

\vspace{2mm} Aun teniendo Atkin una complejidad teórica menor, el tamaño de carga $n$ no es lo suficientemente grande como para superar en rendimiento a la criba de Erastótenes en la práctica.
\end{document} 