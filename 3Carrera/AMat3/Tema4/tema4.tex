\documentclass[12pt,a4paper]{article}

% Paquetes básicos
\usepackage[x11names]{xcolor}
\usepackage[utf8]{inputenc}
\usepackage{mathtools, amssymb, amsthm}
\usepackage{changepage}
\usepackage{geometry}
\usepackage[colorlinks=true]{hyperref}
\usepackage{enumitem}
\usepackage{etoolbox}
\usepackage{graphicx}
\usepackage{setspace}
\usepackage{tcolorbox}
\tcbuselibrary{skins, breakable}
\usepackage{titlesec}
\usepackage{tikz} % core TikZ
\usetikzlibrary{matrix}
\usepackage{pgfplots}
\pgfplotsset{compat=newest}
\usepgfplotslibrary{fillbetween}

% Margins
%\geometry{left=3cm,right=3cm,top=2.5cm,bottom=2.5cm}

% Custom operators
\newcommand{\card}{\operatorname{card}}
\newcommand{\muae}{\overset{\mu-a.e.}{=}}

% Implication box setup
\tcbset{Implication-number/.style={
  enhanced,
  boxsep=2pt,
  colback=white,
  frame hidden,
  sharp corners,
  left=2pt, right=2pt, top=1pt, bottom=1pt,
  underlay={
    \draw[line width=0.5pt] (frame.south west) -- ([xshift=-133mm]frame.south east); % línea horizontal
    \draw[line width=0.5pt] ([xshift=-133mm]frame.north east) -- ([xshift=-133mm]frame.south east); % línea vertical
  }
}}

\tcbset{Implication-number-ds/.style={
  enhanced,
  boxsep=2pt,
  colback=white,
  frame hidden,
  sharp corners,
  left=2pt, right=2pt, top=1pt, bottom=1pt,
  underlay={
    \draw[line width=0.5pt] ([yshift=5mm]frame.south west) -- ([xshift=-133mm, yshift=5mm]frame.south east); % línea horizontal
    \draw[line width=0.5pt] ([xshift=-133mm, yshift=-3mm]frame.north east) -- ([xshift=-133mm, yshift=5mm]frame.south east); % línea vertical
  }
}}

\tcbset{Subset-contingency/.style={
  enhanced,
  boxsep=2pt,
  colback=white,
  frame hidden,
  sharp corners,
  left=2pt, right=2pt, top=1pt, bottom=1pt,
  underlay={
    \draw[line width=0.5pt] (frame.south west) -- ([xshift=-140mm]frame.south east); % línea horizontal
    \draw[line width=0.5pt] ([xshift=-140mm]frame.north east) -- ([xshift=-140mm]frame.south east); % línea vertical
  }
}}

\tcbset{Indent-subset-contingency/.style={
  enhanced,
  boxsep=2pt,
  colback=white,
  frame hidden,
  sharp corners,
  left=2pt, right=2pt, top=1pt, bottom=1pt,
  underlay={
    \draw[line width=0.5pt] (frame.south west) -- ([xshift=-140mm+0.07\textwidth]frame.south east); % línea horizontal
    \draw[line width=0.5pt] ([xshift=-140mm+0.07\textwidth]frame.north east) -- ([xshift=-140mm+0.07\textwidth]frame.south east); % línea vertical
  }
}}

% Useful commands
\renewcommand{\contentsname}{Contenidos}

\newcommand{\R}{\mathbb{R}}
\newcommand{\N}{\mathbb{N}}
\newcommand{\Z}{\mathbb{Z}}
\newcommand{\Q}{\mathbb{Q}}
\newcommand{\C}{\mathbb{C}}

\newcommand{\smallcup}{\mathop{\cup}\limits}
\newcommand{\smallcap}{\mathop{\cap}\limits}
\newcommand{\smallsum}{\mathop{\sum}\limits}
\newcommand{\smallprod}{\mathop{\prod}\limits}

\newcommand{\linf}[1]{\displaystyle{\mathop{\underline{\lim}}_{#1}}}
\newcommand{\mlim}[1]{\displaystyle{\lim_{#1}}}

%Integral de Lebesgue con patas
\newcommand{\lbint}{\mathop{\int_{\!\!\!\!\!|\!}^{\!|\!}}}

%Espacios \mathcal{L}_p
\newcommand{\elep}[2]{\mathcal{L}_{#1}(#2)}

% ----- Custom counters and counter commands -----
% Custom counter hierarchy
\newcounter{unit}[section]
\newcounter{chapter}[unit]
\makeatletter
\@addtoreset{subsubsection}{chapter}
\makeatother

\renewcommand{\theunit}{\arabic{unit}}
\renewcommand{\thechapter}{\arabic{chapter}}
\renewcommand{\thesubsubsection}{\theunit.\thechapter.\arabic{subsubsection}}

% Custom content hierarchy behavior
\newcommand{\chapter}[1]{
    \refstepcounter{chapter}
    \subsection*{\Large{\S \thechapter. #1}}
    \addcontentsline{toc}{subsection}{\thechapter. #1}
}
\newcommand{\unit}[1]{
    \refstepcounter{unit}
    \section*{\Huge{\Roman{unit} #1}}
    \addcontentsline{toc}{section}{\Roman{unit} #1}
}

\newcommand{\result}[1]{%
  \subsubsection{#1}%
  \label{result:\thesubsubsection}
}
  
\titleformat{\subsubsection}
    {\normalfont\large\bfseries} % mismo tamaño que \subsection
    {\thesubsubsection}{1em}{}
  
%---- Custom proof commands-----
\newcommand{\dem}{
    \noindent \underline{\textbf{Demostración:}}
}
\newcommand{\nota}{
    \noindent \underline{\textbf{Nota:}}
}
% ----------------------------------------
\hbadness=10000
\vbadness=10000
\hfuzz=100pt
\vfuzz=100pt
% ---------------------------------------
\title{Análisis Matemático III}
\author{Javier Ortín Rodenas}
\date{Curso 2025-2026}


\begin{document}

\maketitle
\newpage
\hypersetup{linkcolor=black}
\tableofcontents
\hypersetup{linkcolor=Ivory4}
\newpage
\setcounter{unit}{3}
\unit{Series de Fourier}
\hspace{3mm} Recordamos que ${L}_2(X,K) \equiv \frac{\mathcal{L}_2(X,K)}{\sim}$ es un espacio de Hilbert; donde $X$ es un $K-$espacio vectorial, y $f \sim g \iff f \muae g$. El producto interior y su norma asociada se definen como:
\begin{align*}
    \langle f,g\rangle = \int_X \bar{f} \cdot g && ||f||_2 = \sqrt{\langle f,f\rangle}
\end{align*}
Nótese que cuando trabajamos con $\C$ como cuerpo el producto interno tiene linealidad directa en una componente, mientras que tiene linealidad por el conjugado en la otra. Asimismo, el producto interno da lugar a escalares del propio cuerpo sobre el que se define, luego podría dar lugar a valores complejos. Además, el producto interno no conmuta (en general) en $\C$.

\vspace{4mm}
La idea de esta unidad es ``aproximar'' una función $f \in L_2(X,K)$ como ``suma'' de funciones más elementales. En particular, a partir de ahora, consideraremos $X = I = [0,2\pi]$, con $K = \R$ ó $\C$. Además, extenderemos las funciones en este conjunto con periodicidad $2\pi$ en $\R$.

\vspace{6mm} 
\chapter{Primeros conceptos}
\result{Definición de sistema ortonormal}
\hspace{3mm} Dado un espacio de Hilbert $V$, un sistema ortonormal en $V$ es un conjunto $\{\varphi_n\}_{n\in\N} \subseteq V$ que verifica:
$$\langle\varphi_n, \varphi_m\rangle = \delta_{n,m} = \begin{cases}
  1 & \text{ si } n = m \\
  0 & \text{ si } n \neq m
\end{cases}$$

\vspace{6mm}
\result{Ejemplos de sistemas ortonormales}
\begin{enumerate}[label=\roman*)]
    \item $K = \R$, $V = L_2([0,2\pi], \R)$. Consideramos el siguiente sistema:
    \begin{align*}
        \varphi_0(t) =  \frac{1}{\sqrt{2\pi}} &&
        \varphi_{2n-1}(t) = \frac{\cos (nt)}{\sqrt{\pi}} &&
        \varphi_{2n}(t) = \frac{\sin (nt)}{\sqrt{\pi}}
    \end{align*}
    Veamos que es ortonormal:
    \begin{flalign*}
        \int_0^{2\pi} \varphi_0^2(t) \,dt = \int_{0}^{2\pi}\frac{1}{2\pi}\,dt = 1&&
    \end{flalign*}
    \begin{flalign*}
        \int_0^{2\pi} \varphi_{2n+1}^2(t) \,dt &= \int_{0}^{2\pi}\frac{\cos^2(nt)}{\pi}\,dt \overset{u = nt}{=} \frac{1}{n\pi}\int_{0}^{2n\pi}\frac{1+\cos(2u)}{2}du \overset{v=2u}{=} &&\\
        &=\frac{1}{4n\pi} \int_{0}^{4n\pi} 1+ cos(v) = \frac{4n\pi}{4n\pi} + 0= 1
    \end{flalign*}
    \begin{flalign*}
        \int_0^{2\pi} \varphi_{2n}^2(t) \,dt &= \int_{0}^{2\pi}\frac{\sin^2(nt)}{\pi}\,dt \overset{u = nt}{=} \frac{1}{n\pi}\int_{0}^{2n\pi}\frac{1-\cos(2u)}{2}du \overset{v=2u}{=}&&\\
        &=\frac{1}{4n\pi} \int_{0}^{4n\pi} 1-(v) = \frac{4n\pi}{4n\pi} + 0 = 1
    \end{flalign*}
    Por ser funciones con periodo $2\pi$, para $n \in \N$ cualquiera, se cumple:
    $$ \int_{0}^{2\pi}\varphi_{2n+1}(t)\,dt=\int_{0}^{2\pi}\varphi_{2n}(t)\,dt= \int_{0}^{2\pi}\varphi_{2n+1}(t)\cdot\varphi_{2n}(t)\,dt= 0$$
    Hemos demostrado que el sistem
    
    \vspace{8mm}
    \item Para $K = \C$, $V = L_2([0,2\pi], \C)$, el siguiente sistema es ortonormal:
    $$ \left\{\varphi_n(t) = \frac{e^{i\cdot n\cdot t}}{\sqrt{\pi}}\right\}_{n\in\N}$$
\end{enumerate}

\vspace{6mm}  
\result{Teorema de óptima aproximación}
\hspace{3mm} Sea $V$ un $K$-espacio pre-Hilbert. Sea $\{\varphi_0, \ldots,\varphi_n\} \subseteq V$ un sistema ortonormal finito en $V$. Sea $f \in V$. Si $W$ es la clausura $K\langle \varphi_0,\ldots,\varphi_n\rangle$. Entonces, el elemento de $W$ que mejor aproxima $f$ (en cuanto a minimizar la norma de su diferencia) es
$$ s_n = \sum_{k=0}^n \underbracket{\langle f, \varphi_k\rangle>}_{c_k}\varphi_k$$
Es decir, dado $t_n = \smallsum_{k=0}^n b_k \cdot \varphi_k \in W$ cualquiera (formado a partir de escalares cualesquiera), se cumple $||f-s_n|| \leq ||f-t_n||$.

\newpage\dem 
\vspace{2mm} \newline \indent Al ser un espacio Pre-Hilbert por hipótesis, tenemos que :
\begin{flalign*}
    ||f &-t_n||^2 = \langle f-t_n, f-t_n\rangle = \langle f,f\rangle - \langle f, t_n\rangle - \langle t_n, f\rangle + \langle t_n, t_n\rangle =&&\\[2ex]
    &= ||f||^2 - \left\langle f, \sum_{k=0}^{n}b_k \varphi_k\right\rangle - \left\langle \sum_{k=0}^n b_k \varphi_k, f\right\rangle + \left\langle \sum_{k=0}^n b_k \varphi_k, \sum_{k=0}^n b_k \varphi_k\right\rangle = \substack{\text{linealidad}\\[1ex] \text{ortonormalidad}} \\[2ex]
    &= ||f||^2 - \sum_{k=0}^n{b_k} \langle f,  \varphi_k\rangle
    - \sum_{k=0}^n\bar{b}_k \langle f,  \varphi_k\rangle + \sum_{k=0}^{n} \bar{b}_k \cdot b_k = \text{definición de }c_k \\
    &= ||f||^2 + \sum_{k=0}^n \Big[-b_k c_k - \bar{b}_k c_k + |b_k|^2\Big] = (*)
\end{flalign*}
Comparamos ahora con la siguiente expresión:
$$|b_k - c_k|^2 = (b_k - c_k)(\bar{b}_k - \bar{c}_k) = b_k \bar{b_k} - b_k \bar{c}_k - \bar{b}_k c_k + c_k \bar{c}_k = |b_k|^2 + |c_k|^2 - b_k \bar{c}_k - \bar{b}_k c_k$$
\vspace{2mm}
Sustituyendo en la igualdad anterior,
$$(*) = ||f||^2 + \sum_{k=0}^{n}\Big[ |b_k-c_k|^2 - |c_k|^2 \Big]$$
En particular, para $b_k = c_k$ tenemos $||f-s_n||^2 \leq ||f-t_n||^2$.

\vspace{4mm}
Motivados por este resultado, buscamos aproximar $f$ por una ``combinación lineal infinita'' de funciones de un sistema ortonormal.

\vspace{6mm}
\result{Definición de coeficientes de Fourier}
\hspace{3mm} Dada una función $f \in L_2([0,2\pi])$, y un sistema ortonormal $\{\varphi_n\}_{n\in\N}$. Se denomina ``coeficiente $n$-ésimo de Fourier de $f$ respecto de $\{\varphi_n\}_{n\in\N}$'' al escalar:
$$ c_n := \langle f, \varphi_n\rangle = \int_{0}^{2\pi} \bar{f}(t)\cdot \varphi_n(t) \,dt$$
Definimos la suma parcial $n$-ésima de Fourier de $f$ respecto de $\{\varphi_n\}_{n\in\N}$ como:
$$s_n(x) = \sum_{k=0}^n c_k \cdot \varphi_k(x)$$
Análogamente, la serie de Fourier de $f$ respecto de $\{\varphi_n\}_{n\in\N}$ viene dada po:
$$s(x) = \sum_{k=0}^\infty c_k \cdot \varphi_k(x)$$

\vspace{6mm}
\result{Unicidad de las Series de Fourier}
\hspace{3mm} Sea $f \in L_2(I,K)$. Sea $\{\varphi_n\}_{n\in\N}$ un sistema ortonormal. La serie de Fourier de $f$ respecto de $\{\varphi_n\}_{n\in\N}$ está bien definida, pues existe un único elemento $s \in L_2(I,K)$ que verifica:
$$||s-s_n|| \xrightarrow{n\to\infty}0$$

\vspace{4mm} \dem
\vspace{2mm} \newline \indent Sean $p,q$ in $\N$ cualesquiera. Supongamos sin pérdida de generalidad que $q > p$. Entonces,
\begin{flalign*}
  ||s_p -& s_q||^2 = \left|\left| \sum_{k=p+1}^q s_k \right|\right|^2 = \left\langle\sum_{k=p+1}^{q}\langle f, \varphi_k\rangle \varphi_k,\sum_{k=p+1}^{q}\langle f, \varphi_k\rangle \varphi_k\right\rangle \overset{\text{ortonormalidad}}{=} &&\\
  &= \sum_{k=p+1}^q \left|\langle f, \varphi_k\rangle\right|^2 = |a_p - a_q|^2 \hspace{4ex} \text{ donde } a_n = \sum_{k=0}^{n} |\langle f, \varphi_k\rangle|^2
\end{flalign*}
\vspace{2mm}
Por el \hyperref[result:4.1.3]{Teorema de óptima aproximación}, para $n \in \N$ cualquiera tenemos que:
$$||f-s_n||^2 = ||f||^2 - \underbracket{\sum_{k=0}^n |\langle f, \varphi_k\rangle|^2}_{a_n} = ||f||^2 - a_n$$
Esto implica que $||f||^2 \geq a_n$. Así, tomando límites en $n$ tenemos que:
$$a := \sum_{k=0}^\infty|\langle f, \varphi_k\rangle|^2 \leq ||f||^2$$
Esta es la conocida como \textbf{Desigualdad de Bessel}.

\vspace{2mm}
Por tanto, tenemos que $(a_n)_n$ es de Cauchy, luego $(s_n)_n$ ha de serlo también. Al estar además en un espacio de Banach, podemos asegurarnos de que $(s_n)_n$ es convergente a $s$.

\vspace{4mm}
\nota \vspace{2mm} \newline \indent
Este resultado garantiza la convergencia en norma de las sumas parciales de Fourier a su límite. No tiene por qué darse $f(x) = s(x)$. Cabe preguntarse si esto es a su vez equivalente a $||f-s_n|| \xrightarrow{n\to\infty}0$. Pasaremos a estudiar escenarios donde puede ocurrir.

\vspace{6mm}
\result{Identidad de Parseval}
\hspace{3mm} Sea $f \in L_2(I,K)$. Sea $\{\varphi_n\}_{n\in\N} \subseteq L_2(I,K)$ un sistema ortonormal. Sea $c_n := \langle f, \varphi_n\rangle$ para cada $n\in\N$. Entonces,
$$||f-s_n|| \xrightarrow{n\to\infty} 0 \iff \sum_{n=0}^\infty |c_n|^2 = ||f||^2$$

\vspace{4mm} \dem
\vspace{2mm} \newline \indent Por el \hyperref[result:4.1.3]{Teorema de óptima aproximación}, sabemos que:
$$||f-s_n||^2 = ||f||^2 - \sum_{k=0}^{n} |c_k|^2 \hspace{2mm}  \forall \hspace{1mm} n \in \N$$
Despejamos y tomamos límites:
$$\lim_n ||f-s_n||^2 = 0 \iff \sum_{k=0}^{\infty} |c_k|^2 = ||f||^2$$

\vspace{6mm}
\result{Definición de sistema ortonormal completo}
\hspace{3mm} Sea $H$ un espacio de Hilbert. Sea $\{\varphi_n\}_{n\in\N}$ un sistema ortonormal. Se dice ``completo'' si satisface la Identidad de Parseval para cualquier $f \in H$. Es decir,
$$\sum_{k=0}^{\infty}\langle f, \varphi_k\rangle = ||f||^2 \hspace{3mm} \forall \hspace{1mm} f \in H$$

\newpage
\nota
\vspace{2mm} \newline \indent Por la \hyperref[result:4.1.5]{Desigualdad de Bessel}, sabemos que $\smallsum_{k=0}^{\infty}|c_k|^2 \leq ||f||^2$. Exigir la igualdad equivale a decir que la norma se resume completamente por el sistema ortonormal: el sistema puede aportar información sobre todo el espacio $H$. Por tanto, toda componente de $f$ queda recogida por cierto $\varphi_n$, luego $f \neq 0$ implica que no es ortogonal a todo el sistema.

\vspace{6mm}
\result{Ejemplo de sistema ortonormal completo}
\hspace{3mm} En $L_2(I,K)$, el siguiente sistema ortonormal es completo:
$$\left\{\frac{1}{\sqrt{2\pi}}, \frac{\cos(t)}{\sqrt{\pi}}, \frac{\sin(t)}{\sqrt{\pi}}, \frac{\cos(2t)}{\sqrt{\pi}}, \frac{\sin(2t)}{\sqrt{\pi}},\ldots\right\} $$
La demostración es demasiado larga, luego no se verá.

\vspace{4mm} Este es el sistema ortonormal más utilizado, y la serie de Fourier de una función $f$ respecto de él, se denomina simplemente ``serie de Fourier de $f$''. Viene representada por:
$$f(t) \sim \frac{a_0}{2} + \sum_{n=1}^{\infty}\Big[a_n\cos(nt) + b_n\sin(nt)\Big]$$
Donde ``$\sim$'' significa convergencia en media cuadrática. Podemos obtener los coeficientes $a_n,b_n$ por identificación al aplicar la ortonormalidad. Por definición, la suma parcial $k$-ésima de Fourier de $f$ viene dada por:
$$s_k(t) = \sum_{n=0}^{k} \langle f, \varphi_n\rangle \hspace{1mm} \varphi_n(t) = \frac{c_0}{\sqrt{2\pi}} + \sum_{n=1}^{k}\left[c_{2n-1}\frac{\cos(nt)}{\sqrt{\pi}}+c_{2n}\frac{\sin(nt)}{\sqrt{\pi}}\right]$$
\vspace{2mm} De este modo, comparando:
\begin{flalign*}
  \hspace{3mm} a_n &= \frac{c_{2n-1}}{\sqrt{\pi}} = \frac{1}{\sqrt{\pi}} \langle f, \varphi_{2n-1}\rangle = \frac{1}{\pi} \int_{0}^{2\pi}\bar{f}(t) \cdot \cos(nt) \,dt &&\\
  b_n &= \frac{c_{2n}}{\sqrt{\pi}} = \frac{1}{\sqrt{\pi}} \langle f, \varphi_{2n}\rangle = \frac{1}{\pi} \int_{0}^{2\pi}\bar{f}(t) \cdot \sin(nt) \,dt     
\end{flalign*}

\vspace{6mm}
\result{Ejemplo de cálculo de serie de Fourier}
\hspace{3mm} Hallemos la serie de Fourier de la siguiente función $f$:
\begin{align*}
    f: [0,2\pi] \longrightarrow \R &&
    x \longmapsto f(x) = \begin{cases}
      0 &\text{ si } x \in [0, \pi) \cup \{2\pi\} \\
      1 &\text{ si } x \in [\pi, 2\pi)
    \end{cases}
\end{align*}
\begin{flalign*}
    a_n &= \frac{1}{\pi}\int_{0}^{\pi}0 \cdot \cos(nt) \,dt + \frac{1}{\pi}\int_{\pi}^{2\pi}1 \cdot \cos(nt) \,dt = \frac{1}{\pi}\left[\frac{1}{n}\sin(nt)\right]_{t=\pi}^{t=2\pi} = 0&& \\[1ex]
    b_n &= \frac{1}{\pi} \left[-\frac{1}{n}\cos(nt)\right]_{t=\pi}^{t=2\pi} = -\frac{2}{n\pi}
\end{flalign*}

\vspace{4mm}
Por tanto, tenemos que:
$$f(t) \sim \frac{1}{2} - \frac{2}{\pi}\sum_{n=1}^\infty \frac{1}{2n+1}\sin\big((2n+1)t\big)$$

\vspace{6ex}
\begin{center}
\begin{tikzpicture}
\def\f(#1){floor(#1/pi)}

\draw[thick,<->] (-1,0) -- (1+2*pi,0) node[anchor=north west] {$t$};
\draw[thick,<->] (0,-1) -- (0,2) node[above] {$f(t)$};
\draw[blue, thick, domain=0:2*pi, samples=100] plot (\x,{\f(\x)});

\end{tikzpicture}
\end{center}

\end{document}