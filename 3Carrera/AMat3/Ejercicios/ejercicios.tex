\documentclass[12pt,a4paper]{article}

% Paquetes básicos
\usepackage[x11names]{xcolor}
\usepackage[utf8]{inputenc}
\usepackage{mathtools, amssymb, amsthm}
\usepackage{changepage}
\usepackage{geometry}
\usepackage{hyperref}
\usepackage{enumitem}
\usepackage{etoolbox}
\usepackage{graphicx}
\usepackage{setspace}
\usepackage{tcolorbox}
\tcbuselibrary{skins, breakable}
\usepackage{titlesec}
\usepackage{tikz} % core TikZ
\usetikzlibrary{matrix}

% Margins
%\geometry{left=3cm,right=3cm,top=2.5cm,bottom=2.5cm}

% Custom operators
\newcommand{\card}{\operatorname{card}}

% Implication box setup
\tcbset{Implication-number/.style={
  enhanced,
  boxsep=2pt,
  colback=white,
  frame hidden,
  sharp corners,
  left=2pt, right=2pt, top=1pt, bottom=1pt,
  underlay={
    \draw[line width=0.5pt] (frame.south west) -- ([xshift=-133mm]frame.south east); % línea horizontal
    \draw[line width=0.5pt] ([xshift=-133mm]frame.north east) -- ([xshift=-133mm]frame.south east); % línea vertical
  }
}}

\tcbset{Implication-number-ds/.style={
  enhanced,
  boxsep=2pt,
  colback=white,
  frame hidden,
  sharp corners,
  left=2pt, right=2pt, top=1pt, bottom=1pt,
  underlay={
    \draw[line width=0.5pt] ([yshift=5mm]frame.south west) -- ([xshift=-133mm, yshift=5mm]frame.south east); % línea horizontal
    \draw[line width=0.5pt] ([xshift=-133mm, yshift=-3mm]frame.north east) -- ([xshift=-133mm, yshift=5mm]frame.south east); % línea vertical
  }
}}

\tcbset{Subset-contingency/.style={
  enhanced,
  boxsep=2pt,
  colback=white,
  frame hidden,
  sharp corners,
  left=2pt, right=2pt, top=1pt, bottom=1pt,
  underlay={
    \draw[line width=0.5pt] (frame.south west) -- ([xshift=-140mm]frame.south east); % línea horizontal
    \draw[line width=0.5pt] ([xshift=-140mm]frame.north east) -- ([xshift=-140mm]frame.south east); % línea vertical
  }
}}

% Useful commands
\renewcommand{\contentsname}{Contenidos}

\newcommand{\R}{\mathbb{R}}
\newcommand{\N}{\mathbb{N}}
\newcommand{\Z}{\mathbb{Z}}
\newcommand{\Q}{\mathbb{Q}}
\newcommand{\C}{\mathbb{C}}

\newcommand{\smallcup}{\mathop{\cup}\limits}
\newcommand{\smallcap}{\mathop{\cap}\limits}
\newcommand{\smallsum}{\mathop{\sum}\limits}
\newcommand{\smallprod}{\mathop{\prod}\limits}

\newcommand{\linf}[1]{\displaystyle{\mathop{\underline{\lim}}_{#1}}}
\newcommand{\mlim}[1]{\displaystyle{\lim_{#1}}}

%Integral de Lebesgue con patas
\newcommand{\lbint}{\mathop{\int_{\!\!\!\!\!|\!}^{\!|\!}}}

% ----- Custom counters and counter commands -----
% Custom counter hierarchy
\newcounter{unit}[section]
\newcounter{chapter}[unit]
\makeatletter
\@addtoreset{subsubsection}{chapter}
\makeatother

\renewcommand{\theunit}{\arabic{unit}}
\renewcommand{\thechapter}{\arabic{chapter}}
\renewcommand{\thesubsubsection}{\theunit.\thechapter.\arabic{subsubsection}}

% Custom content hierarchy behavior
\newcommand{\chapter}[1]{
    \refstepcounter{chapter}
    \subsection*{\Large{\S \thechapter. #1}}
    \addcontentsline{toc}{subsection}{\thechapter. #1}
}
\newcommand{\unit}[1]{
    \refstepcounter{unit}
    \section*{\Huge{\Roman{unit} #1}}
    \addcontentsline{toc}{section}{\Roman{unit} #1}
}

\newcommand{\result}[1]{%
  \subsubsection{#1}%
  \label{result:\thesubsubsection}
}
  
\titleformat{\subsubsection}
    {\normalfont\large\bfseries} % mismo tamaño que \subsection
    {\thesubsubsection}{1em}{}
  
%---- Custom proof commands-----
\newcommand{\dem}{
    \noindent \underline{\textbf{Demostración:}}
}
\newcommand{\nota}{
    \noindent \underline{\textbf{Nota:}}
}
% ----------------------------------------

\title{Análisis Matemático II}
\author{Javier Ortín Rodenas}
\date{Hoja de ejercicios}

\begin{document}
\onehalfspacing

\maketitle
\newpage
\tableofcontents
\newpage
\unit{Teoría de la medida e Integral de Lebesgue}
\chapter{Medida exterior en \texorpdfstring{$\R^N$}{R\^N}}
\result{Ejercicio 5}
a) Sea $A = \Q \cap [0,1]$, demuestra que todo cubrimiento finito de $A$ formado por intervalos abiertos tiene longitud total mayor o igual a 1.

\vspace{2mm}
\dem 
Sea $\{I_i\}_{i=1}^n$ un recubrimiento por intervalos abiertos de $A$, con cada $I_i = (a_i, b_i)$ con $a_i < b_i$. Definimos el siguiente conjunto auxiliar:
$$\mathcal{A} := \{0,1\} \cup \Big([0,1]\cap \big\{a_i,b_i : i \in \{1,\ldots,n\}\big\}\Big) $$
Este conjunto contiene a los extremos de los $I_i$ que se encuentren entre $0$ y $1$. Al haber un número finito de intervalos, tenemos que $\mathcal{A}$ es finito, pudiendo ordenarlo como sigue:
\begin{align*}
    \mathcal{A} = \{x_j\}_{j=0}^m &&
    x_0 = a < x_1 < \ldots < x_m = b &&
    \text{ para cierto } m \in \N
\end{align*}

Al estar en un caso finito, la unión de las clausuras es la clausura de las uniones. Por contención de las clausuras, se tiene:
\begin{flalign*}
    A \subseteq \bigcup_{i=1}^n I_i \Rightarrow \overline{A} = [0, 1] \cap \overline{\Q} = [0,1] \subseteq \overline{\bigcup_{i=1}^n I_i} = \bigcup_{i=1}^n \overline{I_i}
\end{flalign*}
Sea $j \in \{1,\ldots, m\}$ cualquiera, tenemos que $x_j \in [0,1] \subseteq \smallcup_{i=1}^n \overline{I_i}$ luego $\exists k \in \{1,\ldots,n\}$ tal que $x_j \in [a_k, b_k]$. Veamos que $[x_{j-1}, x_j] \subseteq [a_k, b_k]$ también.
Ambos conjuntos son cerrados y conexos con intersección no vacía. Por tanto, su intersección ha de ser conexa, cerrada y no vacía (es decir, un intervalo cerrado). Será de la forma $[u, x_j]$ con $u = \min \{a_k, x_{j-1}\}$. De tenerse $u \neq x_{j-1}$, se cumpliría $x_{j-1} < a_k < x_{j}$, lo que contradice la ordenación establecida para $\mathcal{A}$.

De este modo, como $v_1(I_i) = v_1(\overline{I_i}) \hspace{2mm} \forall i \in \{1,\ldots, n\}$, se cumple:
\begin{flalign*}
    1 = v_1([0,1]) = \sum_{j=1}^m v_1 (x_{j-1}, x_j) \leq \sum_{i=1}^n v_1(\overline{I_i}) = \sum_{i=1}^n v_1(I_i)
\end{flalign*}

\vspace{4mm}
b) Deduce del apartado anterior que $A$ no es compacto.

\vspace{2mm}
\dem Veamos que hay un recubrimiento de $A$ del que no se puede extraer un subrecubrimiento finito. Para ello basta dar un recubrimiento de $A$ por intervalos abiertos tales que la suma de sus medidas sea menor a $1$. Como $\Q$ es numerable, $A$ también lo es. Sea $(a_n)_{n\in\N}$ una enumeración cualquiera de $A$, definimos los siguientes intervalos:
\begin{align*}
    I_n = (a_n - \frac{1}{2^{n+2}}, a_n + \frac{1}{2^{n+2}}) &&
    \text{ luego } A \subseteq \smallcup_{i\in\N}I_i \text{ con } \sum_{i=1}^\infty v_1(I_i) = \sum_{n=1}^{\infty} \frac{1}{2^{n+1}} = \frac{1}{2}
\end{align*}

\hspace{4mm} \noindent
c) Demuestra que la siguiente aplicación $m$ definida sobre $\mathfrak{B}_N$ no es una medida:
$$ m(A) = \inf \left\{\sum{i=1}^n v(I_i) : \smallcup_{i=1}^n I_i \supseteq A \hspace{1mm} \text{ con cada } I_i \text{ cubo abierto en} \R^N \right\}$$

\vspace{2mm}
\dem \newline \noindent En caso de serlo, para $\{E_i\}_{i\in\N} \subseteq \mathfrak{B}_N$ con $i \neq j \Rightarrow E_i \cap E_j = \varnothing$ se cumpliría:
$$ m\left(\smallcup_{i\in\N} E_i\right) = \sum_{i\in\N} \mu(E_i)$$
Todo conjunto unipuntual ${a}$ con $a \in \R^N$ está en $\mathfrak{B}_N$ pues se puede expresar como un intervalo cerrado degenerado. Por tando, todo conjunto numerable está también en $\mathfrak{B}_N$. Consideramos el conjunto $A^N \subseteq [0,1]^N \in \mathfrak{B}_N$. Razonando como en el apartado anterior, todo cubrimiento finito de $A^N$ por cubos ha de tener suma de volúmenes mayor o igual a $1$, luego $m(A^N) = 1$. No obstante, podemos expresar $A^N$ como unión numerable de conjuntos unipuntuales disjuntos, teniendo cada uno de ellos $m(\{b_n\}) = 0$ para $(b_n)_n$ una enumeración de $A^N$.

\vspace{4mm}
\result{Ejercicio 6}
Demuestra que existe un conjunto abierto $O$ en $\R$ y un $\varepsilon > 0$ tal que para cualquier recubrimiento finito de $O$ formado por intervalos abiertos, $\{J_i\}_{i=1}^m$ se cumple $\mu(\smallcup_{i=1}^n J_i \setminus O) > \varepsilon$.
\vspace{2mm}
\dem \newline \indent Para cada $n \in \N$ tomamos $I_n = (n - \frac{1}{3}, n + \frac{1}{3})$. Consideramos el abierto $O = \smallcup_{i\in\N} I_i$. Sea $\{J_i\}_{i=1}^m$ un recubrimiento finito de $O$ por intervalos abiertos. Al ser $O$ no acotado, tiene que haber cierto $k \in \{1, \ldots, m\}$ tal que $J_k$ contiene a infinitos intervalos $I_n$. Dados dos de estos intervalos consecutivos, la distancia entre el extremo derecho del primero y el extremo izquierdo del segundo viene dada por:
$$\left(n+1 - \frac{1}{3}\right) - \left(n + \frac{1}{3}\right) = 1 - \frac{2}{3} = \frac{1}{3}$$
Por tanto, $J_k \setminus O$ ha de contener infinitos intervalos disjuntos de longitud $\frac{1}{3}$ luego tiene medida infinita. Se cumple el resultado para $\varepsilon > 0$ cualquiera.

%CAMBIO DE HOJA
\newpage
\chapter{Conjuntos medibles}
\vspace{2mm}\result{Ejercicio 2}
\hspace{3mm} Sea $A \in \mathfrak{M}_N$ con $\mu(A) < \infty$. Sea $\varepsilon > 0$, demuestra que existe $K \subseteq A$ conjunto compacto tal que $\mu(A \setminus K) < \varepsilon$. ¿Se cumple también si $\mu(A) = \infty$?
\vspace{2mm}
\dem \newline \indent Sea $A_n = A \cap [-n,n]$, tenemos que $(A_n)_n$ es una sucesión creciente de conjuntos medible que tiene como unión el conjunto $A$. Por tanto, se tiene
$$\infty > \mu(A) \mu(\smallcup_{i\in\N} A_i) = \lim_{n} \mu(A_n)$$
Por tanto, existe cierto $n_0 \in \N$ tal que $\mu(A) - \mu(A_{n_0}) = \mu(A \setminus A_{n_0}) < \frac{\varepsilon}{2}$. Hemos podido despejar sin indeterminaciones al estar en un caso de medida finita.

\vspace{2mm} Como $A_{n_0}$ es medible, sabemos que existe un conjunto cerrado $C \subseteq A_{n_0}$ tal que $\mu(A_{n_0} \setminus C) < \frac{\varepsilon}{2}$. Al ser $A_{n_0}$ acotado y estar $C$ contenido en él, tenemos que $C$ es compacto. Por todo lo anterior:
$$\mu(A \setminus C) = \mu(A) - \mu(C) = \mu(A) - \mu(A_{n_0}) + \mu(A_{n_0}) - \mu(C) < \frac{\varepsilon}{2} + \frac{\varepsilon}{2} = \varepsilon$$

\vspace{2mm} En cambio, para el caso infinito, no tiene por qué cumplirse. Sea $A = R^N$, en la topología usual, los conjuntos compactos son aquellos cerrados y acotados. En espacios métricos, un conjunto es acotado sii existe una bola de radio finito que lo contiene. Por el Teorema de Hausdorff, podemos considerar la norma $||\cdot||_1$, que tiene como bolas a los cubos. Por tanto, todo compacto $K$ va a estar contenido en una bola de radio finito; es decir, en un cubo de volumen (medida) finito. Despejando:
$$\mu(\R^N \setminus K) = \mu(\R^N) - \mu(K) = \infty - \mu(K) = \infty > \varepsilon \hspace{2mm} \forall \varepsilon \in (0, +\infty) $$

\newpage
\result{Ejercicio 3}
\hspace{3mm} Sea $A \in \mathfrak{M}_N$, ¿son ciertas las siguientes implicaciones?
\begin{enumerate}[label=\roman*)]
    \item $\mathring{A} \neq \varnothing \Rightarrow \mu(A) = 0$.
    \item $\mathring{A} = \varnothing \Rightarrow \mu(A) = 0$.
    \item $A$ abierto $\Rightarrow \mu(\mathop{Fr.} A) = 0$.
    \item $A$ no numerables $\Rightarrow \mu(A) > 0$.
\end{enumerate}
\vspace{2mm} \dem \newline \noindent $\romannumeral 1)$ Falso: basta considerar $\R^N$.

\vspace{2mm} \noindent $\romannumeral 2)$ Falso: basta considerar $\R \setminus \Q$ (o cualquier variante $N$-dimensional).

\vspace{2mm} \noindent $\romannumeral 3)$ Utilizaremos una variación del conjunto de Cantor. En cada iteración ``eliminamos'' la parte central de longitud $\frac{1}{4^n}$ de cada intervalo. Veamos ejemplos de algunas iteraciones:
\\[-2ex]
\begin{align*}
  &&\Big[0,1\Big]&&
\end{align*}
\begin{align*}
  \left[0, \frac{3}{8}\right] && \left[\frac{5}{8}, 1\right]
\end{align*}
\begin{align*}
  \left[0, \frac{1}{8}\right] && \left[\frac{1}{4}, \frac{3}{8}\right] && \left[\frac{5}{8}, \frac{3}{4}\right] && \left[\frac{7}{8}, 1\right]
\end{align*}
En la $n$-ésima iteración eliminamos $2^{n-1}$ intervalos cada uno de longitud $\frac{1}{4^n}$. Como los intervalos quitados son disjuntos, podemos calcular su medida total como la suma de las longitudes quitadas. Sea $L_n$ la longitud quitada en la $n$-ésima iteración,
\begin{align*}
  L_n = 2^{n-1} \cdot \frac{1}{4^n} = \frac{1}{2^n+1} &&
  \sum_{n=1}^\infty L_n = \frac{1}{2}
\end{align*}
Considerando $O = (0,1)\setminus C$ para $C$ la intersección de todas las iteraciones, tenemos que la frontera de $O$ es $C$, que tiene como medida $1 - \frac{1}{2} = \frac{1}{2}$.

\vspace{2mm}
$\romannumeral 4)$ Falso: basta considerar $\R \setminus \Q$.

\vspace{4mm}
\result{Ejercicio 5}
Demuestra que para cada $A \in \mathfrak{M}_1$ con $\mu(A) < \infty$ y cada $\varepsilon > 0$ existe una colección finita de intervalos disjuntos y abiertos $\{I_i\}_{i=1}^m$ tales que:
\begin{align*}
    \mu\left(A \Delta \smallcup_{i=1}^m I_i\right) < \varepsilon && \text{ con } A \Delta B = (A \setminus B) \cup (B \setminus A)
\end{align*}
\vspace{2mm} \dem \newline \vspace{2mm} \indent Por ser $A$ un conjunto medible, es de la forma $A = (\smallcap_{i\in\N} G_i) \cup M$ para $(G_i)_{i\in\N}$ una sucesión decreciente de abiertos de $\R^N$ y $M$ un conjunto de medida nula.
\vspace{2mm} \newline \indent Como $M$ no interfiere en la medida, tenemos que $\mu(A) = \lim_n \mu(G_n) < \infty$. Por definición de límite, $\exists \hspace{1mm} n_0 \in \N$ tal que $\mu(G_{n_0} \setminus A) = \mu(G_{n_0}) - \mu(A) < \frac{\varepsilon}{2}$. Hemos podido pasar a la medida de la diferencia de conjuntos al estar en el caso de medida finita. Por simplicidad, denotaremos $G = G_{n_0}$.
\vspace{2mm} \newline Al ser $G$ abierto, podemos expresarlo como unión numerable de bolas abiertas $\{J_i\}_{i\in\N}$. Para cada $n \in \N$ tomamos $O_n = \smallcup_{i=1}^n J_i$. Tenemos que $(O_n) _n$ es una sucesión de conjuntos crecientes cuya unión es igual a $G$. Razonando como en el caso anterior:
\begin{align*}
    \mu(G) = \mu(\smallcup_{i\in\N} O_i) = \lim_n \mu(O_n) \Rightarrow \exists \hspace{1mm} n_1 \in \N : \mu(G \setminus O_n) < \frac{\varepsilon}{2}
\end{align*}
\indent Tomamos como $\{I_i\}_{i=1}^n$ las componentes conexas de $\{J_i\}_{i=1}^{n_1}$, que son disjuntas e intervalos (en $\R$ los conjuntos conexos son precisamente los intervalos). Además, hay un número finito de ellas (en el peor de los casos, cada $J_i$ sería una componente conexa). De este modo, aplicando la monotonía de la medida:
\begin{flalign*}
    \mu\left(A \setminus \smallcup_{i=1}^n I_i\right) \leq \mu\left(G \setminus \smallcup_{i=1}^n I_i\right) < \frac{\varepsilon}{2} \\[2ex]
    \mu\left(\smallcup_{i=1}^n I_i \setminus A \right) \leq \mu\left(G \setminus A\right) < \frac{\varepsilon}{2}
\end{flalign*}
Sumando se llega al resultado.

%CAMBIO DE HOJA
\newpage
\chapter{Funciones medibles}
\vspace{2mm}
\result{Ejercicio 5}
\hspace{3mm} Sea $f : \R^N \longrightarrow \R$ uniformemente continua y acotada, demuestra que la siguiente función es medible:
\begin{align*}
    \varphi : \R^N \longrightarrow \R &&
    t \longmapsto \varphi(t) = \sup_{x\in\R^N} |f(x+t) - f(x)|
\end{align*}

\vspace{5mm}
\dem

\vspace{2mm} Por ser $f$ acotada, tenemos que $\exists \hspace{1mm} M \in \R : |f(x)|\leq M \forall x \in \R^N$. De este modo, sean $x, y \in \R^N$, tenemos:
$$|f(x) - f(y) | \leq |f(x)| + |f(y)| \leq M + M = 2M$$
Por tanto, concluimos que $\varphi$ es también acotada. Veamos que es medible por la definición.

\vspace{4mm}
Sea $K := \displaystyle \sup_{t\in\R^N}\varphi(t)$, como $\varphi$ solo toma valores no negativos tenemos que:
\begin{align*}
    \varphi^-1\Big((-\infty, 0]\Big) = \varnothing \in \mathfrak{M}_N &&
    \varphi^-1\Big((-\infty, K]\Big) = \R^N \in \mathfrak{M}_N 
\end{align*}
Fijando ahora $\varepsilon \in (0, K)$ cualquiera, veamos que $\varphi\Big((-\infty, \varepsilon]\Big)$ es medible. Al ser $f$ uniformemente continua por hipótesis, tenemos que para el $\varepsilon$ fijado existe un $\delta > 0$ tal que:
$$|| x - y|| < \delta \Rightarrow |f(x) - f(y)| < \varepsilon \hspace{4mm} \forall x,y \in \R^N$$

Podemos reescribir $y = x + t$ para $t = y - x \in \R^N$. De este modo; sea $t \in \R^N$, se tiene:
$$||t|| < \delta \Rightarrow |f(x+t) - f(x)| < \varepsilon$$
\indent Se cumple la condición para todo $x$ de $\R^N$. Al tomar supremos, la desigualdad pasa a ser no estricta, y el término mayorado por $\varepsilon$ se corresponde con $\varphi(t)$ por definición:
$$||t|| < \delta \Rightarrow \varphi(t) = \sup_{x\in\R^N} |f(x+t) - f(x)| \leq \varepsilon $$
Así, sea $(t_n)_n$ una sucesión que tiende al origen, ha de "atravesar" todos las bolas de radio $\delta$ asociado a valores de $\varepsilon$ arbitrariamente pequeños. Por tanto, podemos concluir que $\phi$ es continua en el origen.

\vspace{4mm}
Sean $s,t \in \R^N$ cualesquiera, se cumple:
\begin{flalign*}
    \varphi(t) &= \sup_{x\in\R^N} |f(x+t) - f(x)| \leq \sup_{x\in\R^N} |f(x+t) - f(x+s)|+ |f(x+s) - f(x)| \leq\\
    &\leq \sup_{x\in\R^N}|f(x + t) - f(x + s)| + \sup_{y\in\R^N}|f(y+s) - f(y)| \overset{z = x + s}{=} \sup_{z\in\R^N}| f\big(z + (t-s) \big)- f(z)| + \varphi(s)
\end{flalign*}
Despejando, obtenemos $0 \leq \varphi(t) - \varphi(s) \leq \varphi(t-s) \xrightarrow{s\to t} \varphi(0) = 0$.

\vspace{4mm} Por todo lo anterior, concluimos que $\varphi$ es continua y acotada, luego es medible en $\mathfrak{B}_N$ y en $\mathfrak{M}_N$
\end{document}