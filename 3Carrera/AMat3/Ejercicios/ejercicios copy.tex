\documentclass[12pt,a4paper]{article}

% Paquetes básicos
\usepackage[x11names]{xcolor}
\usepackage[utf8]{inputenc}
\usepackage{mathtools, amssymb, amsthm}
\usepackage{changepage}
\usepackage{geometry}
\usepackage{hyperref}
\usepackage{enumitem}
\usepackage{etoolbox}
\usepackage{graphicx}
\usepackage{setspace}
\usepackage{tcolorbox}
\tcbuselibrary{skins, breakable}
\usepackage{titlesec}
\usepackage{tikz} % core TikZ
\usetikzlibrary{matrix}

% Margins
%\geometry{left=3cm,right=3cm,top=2.5cm,bottom=2.5cm}

% Custom operators
\newcommand{\card}{\operatorname{card}}

% Implication box setup
\tcbset{Implication-number/.style={
  enhanced,
  boxsep=2pt,
  colback=white,
  frame hidden,
  sharp corners,
  left=2pt, right=2pt, top=1pt, bottom=1pt,
  underlay={
    \draw[line width=0.5pt] (frame.south west) -- ([xshift=-133mm]frame.south east); % línea horizontal
    \draw[line width=0.5pt] ([xshift=-133mm]frame.north east) -- ([xshift=-133mm]frame.south east); % línea vertical
  }
}}

\tcbset{Implication-number-ds/.style={
  enhanced,
  boxsep=2pt,
  colback=white,
  frame hidden,
  sharp corners,
  left=2pt, right=2pt, top=1pt, bottom=1pt,
  underlay={
    \draw[line width=0.5pt] ([yshift=5mm]frame.south west) -- ([xshift=-133mm, yshift=5mm]frame.south east); % línea horizontal
    \draw[line width=0.5pt] ([xshift=-133mm, yshift=-3mm]frame.north east) -- ([xshift=-133mm, yshift=5mm]frame.south east); % línea vertical
  }
}}

\tcbset{Subset-contingency/.style={
  enhanced,
  boxsep=2pt,
  colback=white,
  frame hidden,
  sharp corners,
  left=2pt, right=2pt, top=1pt, bottom=1pt,
  underlay={
    \draw[line width=0.5pt] (frame.south west) -- ([xshift=-140mm]frame.south east); % línea horizontal
    \draw[line width=0.5pt] ([xshift=-140mm]frame.north east) -- ([xshift=-140mm]frame.south east); % línea vertical
  }
}}

% Useful commands
\renewcommand{\contentsname}{Contenidos}

\newcommand{\R}{\mathbb{R}}
\newcommand{\N}{\mathbb{N}}
\newcommand{\Z}{\mathbb{Z}}
\newcommand{\Q}{\mathbb{Q}}
\newcommand{\C}{\mathbb{C}}

\newcommand{\smallcup}{\mathop{\cup}\limits}
\newcommand{\smallcap}{\mathop{\cap}\limits}
\newcommand{\smallsum}{\mathop{\sum}\limits}
\newcommand{\smallprod}{\mathop{\prod}\limits}

\newcommand{\linf}[1]{\displaystyle{\mathop{\underline{\lim}}_{#1}}}
\newcommand{\mlim}[1]{\displaystyle{\lim_{#1}}}

%Integral de Lebesgue con patas
\newcommand{\lbint}{\mathop{\int_{\!\!\!\!\!|\!}^{\!|\!}}}

% ----- Custom counters and counter commands -----
% Custom counter hierarchy
\newcounter{unit}[section]
\newcounter{chapter}[unit]
\makeatletter
\@addtoreset{subsubsection}{chapter}
\makeatother

\renewcommand{\theunit}{\arabic{unit}}
\renewcommand{\thechapter}{\arabic{chapter}}
\renewcommand{\thesubsubsection}{\theunit.\thechapter.\arabic{subsubsection}}

% Custom content hierarchy behavior
\newcommand{\chapter}[1]{
    \refstepcounter{chapter}
    \subsection*{\Large{\S \thechapter. #1}}
    \addcontentsline{toc}{subsection}{\thechapter. #1}
}
\newcommand{\unit}[1]{
    \refstepcounter{unit}
    \section*{\Huge{\Roman{unit} #1}}
    \addcontentsline{toc}{section}{\Roman{unit} #1}
}

\newcommand{\result}[1]{%
  \subsubsection{#1}%
  \label{result:\thesubsubsection}
}
  
\titleformat{\subsubsection}
    {\normalfont\large\bfseries} % mismo tamaño que \subsection
    {\thesubsubsection}{1em}{}
  
%---- Custom proof commands-----
\newcommand{\dem}{
    \noindent \underline{\textbf{Demostración:}}
}
\newcommand{\nota}{
    \noindent \underline{\textbf{Nota:}}
}
% ----------------------------------------

\title{Análisis Matemático II}
\author{Javier Ortín Rodenas}
\date{Hoja de ejercicios}

\begin{document}
\onehalfspacing

\maketitle
\newpage
\tableofcontents
\newpage
\unit{Teoría de la medida e Integral de Lebesgue}
\chapter{Medida exterior en \texorpdfstring{$\R^N$}{R\^N}}
\result{Ejercicio 5}
a) Sea $A = \Q \cap [0,1]$, demuestra que todo cubrimiento finito de $A$ formado por intervalos abiertos tiene longitud total mayor o igual a 1.

\vspace{2mm}
\dem 
Sea $\{I_i\}_{i=1}^n$ un recubrimiento por intervalos abiertos de $A$, con cada $I_i = (a_i, b_i)$ con $a_i < b_i$. Definimos el siguiente conjunto auxiliar:
$$\mathcal{A} := \{0,1\} \cup \Big([0,1]\cap \big\{a_i,b_i : i \in \{1,\ldots,n\}\big\}\Big) $$
Este conjunto contiene a los extremos de los $I_i$ que se encuentren entre $0$ y $1$. Al haber un número finito de intervalos, tenemos que $\mathcal{A}$ es finito, pudiendo ordenarlo como sigue:
\begin{align*}
    \mathcal{A} = \{x_j\}_{j=0}^m &&
    x_0 = a < x_1 < \ldots < x_m = b &&
    \text{ para cierto } m \in \N
\end{align*}

Al estar en un caso finito, la unión de las clausuras es la clausura de las uniones. Por contención de las clausuras, se tiene:
\begin{flalign*}
    A \subseteq \bigcup_{i=1}^n I_i \Rightarrow \overline{A} = [0, 1] \cap \overline{\Q} = [0,1] \subseteq \overline{\bigcup_{i=1}^n I_i} = \bigcup_{i=1}^n \overline{I_i}
\end{flalign*}
Sea $j \in \{1,\ldots, m\}$ cualquiera, tenemos que $x_j \in [0,1] \subseteq \smallcup_{i=1}^n \overline{I_i}$ luego $\exists k \in \{1,\ldots,n\}$ tal que $x_j \in [a_k, b_k]$. Veamos que $[x_{j-1}, x_j] \subseteq [a_k, b_k]$ también.
Ambos conjuntos son cerrados y conexos con intersección no vacía. Por tanto, su intersección ha de ser conexa, cerrada y no vacía (es decir, un intervalo cerrado). Será de la forma $[u, x_j]$ con $u = \min \{a_k, x_{j-1}\}$. De tenerse $u \neq x_{j-1}$, se cumpliría $x_{j-1} < a_k < x_{j}$, lo que contradice la ordenación establecida para $\mathcal{A}$.

De este modo, como $v_1(I_i) = v_1(\overline{I_i}) \hspace{2mm} \forall i \in \{1,\ldots, n\}$, se cumple:
\begin{flalign*}
    1 = v_1([0,1]) = \sum_{j=1}^m v_1 (x_{j-1}, x_j) \leq \sum_{i=1}^n v_1(\overline{I_i}) = \sum_{i=1}^n v_1(I_i)
\end{flalign*}
\newpage
\chapter{Funciones medibles}
\vspace{2mm}
\result{Ejercicio 5}
\hspace{3mm} Sea $f : \R^N \longrightarrow \R$ uniformemente continua y acotada, demuestra que la siguiente función es medible:
\begin{align*}
    \varphi : \R^N \longrightarrow \R &&
    t \longmapsto \varphi(t) = \sup_{x\in\R^N} |f(x+t) - f(x)|
\end{align*}

\vspace{5mm}
\dem

\vspace{2mm} Por ser $f$ acotada, tenemos que $\exists \hspace{1mm} M \in \R : |f(x)|\leq M \forall x \in \R^N$. De este modo, sean $x, y \in \R^N$, tenemos:
$$|f(x) - f(y) | \leq |f(x)| + |f(y)| \leq M + M = 2M$$
Por tanto, concluimos que $\varphi$ es también acotada. Veamos que es medible por la definición.

\vspace{4mm}
Sea $K := \displaystyle \sup_{t\in\R^N}\varphi(t)$, como $\varphi$ solo toma valores no negativos tenemos que:
\begin{align*}
    \varphi^-1\Big((-\infty, 0]\Big) = \varnothing \in \mathfrak{M}_N &&
    \varphi^-1\Big((-\infty, K]\Big) = \R^N \in \mathfrak{M}_N 
\end{align*}
Fijando ahora $\varepsilon \in (0, K)$ cualquiera, veamos que $\varphi\Big((-\infty, \varepsilon]\Big)$ es medible. Al ser $f$ uniformemente continua por hipótesis, tenemos que para el $\varepsilon$ fijado existe un $\delta > 0$ tal que:
$$|| x - y|| < \delta \Rightarrow |f(x) - f(y)| < \varepsilon \hspace{4mm} \forall x,y \in \R^N$$

Podemos reescribir $y = x + t$ para $t = y - x \in \R^N$. De este modo; sea $t \in \R^N$, se tiene:
$$||t|| < \delta \Rightarrow |f(x+t) - f(x)| < \varepsilon$$
\indent Se cumple la condición para todo $x$ de $\R^N$. Al tomar supremos, la desigualdad pasa a ser no estricta, y el término mayorado por $\varepsilon$ se corresponde con $\varphi(t)$ por definición:
$$||t|| < \delta \Rightarrow \varphi(t) = \sup_{x\in\R^N} |f(x+t) - f(x)| \leq \varepsilon $$
Así, sea $(t_n)_n$ una sucesión que tiende al origen, ha de "atravesar" todos las bolas de radio $\delta$ asociado a valores de $\varepsilon$ arbitrariamente pequeños. Por tanto, podemos concluir que $\phi$ es continua en el origen.

\vspace{4mm}
Sean $s,t \in \R^N$ cualesquiera, se cumple:
\begin{flalign*}
    \varphi(t) &= \sup_{x\in\R^N} |f(x+t) - f(x)| \leq \sup_{x\in\R^N} |f(x+t) - f(x+s)|+ |f(x+s) - f(x)| \leq\\
    &\leq \sup_{x\in\R^N}|f(x + t) - f(x + s)| + \sup_{y\in\R^N}|f(y+s) - f(y)| \overset{z = x + s}{=} \sup_{z\in\R^N}| f\big(z + (t-s) \big)- f(z)| + \varphi(s)
\end{flalign*}
Despejando, obtenemos $0 \leq \varphi(t) - \varphi(s) \leq \varphi(t-s) \xrightarrow{s\to t} \varphi(0) = 0$.

\vspace{4mm} Por todo lo anterior, concluimos que $\varphi$ es continua y acotada, luego es medible en $\mathfrak{B}_N$ y en $\mathfrak{M}_N$
\end{document}