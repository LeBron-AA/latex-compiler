\documentclass{article}

\usepackage[utf8]{inputenc}
\usepackage[T1]{fontenc}
\usepackage[spanish]{babel}
\usepackage{amsmath,amssymb,amsthm}
\usepackage{enumitem}

\newtheorem{definicion}{Definición}[section]
\newtheorem{proposicion}[definicion]{Proposición}
\newtheorem{teorema}[definicion]{Teorema}
\newtheorem{lema}[definicion]{Lema}
\newtheorem{observacion}[definicion]{Observación}

\begin{document}

\section*{I Espacios métricos}

En próximas secciones estudiaremos sistemas de distancia que generalizan la distancia habitual en la recta real y en los espacios euclídeos. Un espacio métrico es un conjunto junto con una función que mide la distancia entre pares de puntos y satisface ciertas propiedades básicas (positividad, simetría y desigualdad triangular). A partir de una métrica se define una estructura topológica mediante las bolas abiertas, y se estudian conceptos de convergencia, continuidad, completitud, etc.

\section{Primeras definiciones}

\begin{definicion}[Métrica]\label{def:1.1}
Sea \(X\) un conjunto. Una aplicación \(d:X\times X\to\mathbb{R}\) se llama \emph{métrica} en \(X\) si para todo \(x,y,z\in X\) se cumplen:
\begin{enumerate}[label=(D.\arabic*)]
\item \(d(x,y)\ge 0\) (positividad).
\item \(d(x,y)=0\) si y sólo si \(x=y\).
\item \(d(x,y)=d(y,x)\) (simetría).
\item \(d(x,z)\le d(x,y)+d(y,z)\) (desigualdad triangular).
\end{enumerate}
El par \((X,d)\) se llama \emph{espacio métrico}.
\end{definicion}

\begin{definicion}\label{def:1.2}
Si \((X,d)\) es un espacio métrico y \(Y\subset X\) es un subconjunto, al restringir \(d\) a \(Y\times Y\) obtenemos una métrica en \(Y\), llamada la métrica inducida o submétrica.
\end{definicion}

\begin{definicion}\label{def:1.3}
Ejemplos de métricas:
\begin{enumerate}
\item En \(\mathbb{R}\), la métrica usual \(d(x,y)=|x-y|\).
\item En \(\mathbb{R}^n\), la métrica euclídea \(d(x,y)=\|x-y\|_2\).
\item Métrica discreta: \(d(x,y)=0\) si \(x=y\), \(d(x,y)=1\) si \(x\neq y\).
\item Métricas derivadas de normas en espacios vectoriales (p. ej. \(\ell^p\)-normas y sus métricas asociadas).
\end{enumerate}
\end{definicion}

\subsection{Bolas, abiertos y cerrados}

\begin{definicion}
Sea \((X,d)\) un espacio métrico. Para \(x\in X\) y \(r>0\) se define la \emph{bola abierta} de centro \(x\) y radio \(r\) como
\[
B(x,r)=\{y\in X:\ d(x,y)<r\},
\]
y la \emph{bola cerrada}
\[
\overline{B}(x,r)=\{y\in X:\ d(x,y)\le r\}.
\]
\end{definicion}

\begin{definicion}
Un conjunto \(U\subset X\) es \emph{abierto} si para todo \(x\in U\) existe \(r>0\) tal que \(B(x,r)\subset U\). Un conjunto \(F\subset X\) es \emph{cerrado} si su complemento \(X\setminus F\) es abierto.
\end{definicion}

\begin{proposicion}\label{prop:bolasabiertas}
En un espacio métrico \((X,d)\):
\begin{enumerate}
\item Toda bola abierta \(B(x,r)\) es un conjunto abierto.
\item Una unión arbitraria de conjuntos abiertos es abierta.
\item Una intersección finita de conjuntos abiertos es abierta.
\end{enumerate}
\end{proposicion}

\begin{observacion}
A partir de las bolas abiertas se obtiene una \emph{topología} en \(X\), denominada la topología inducida por la métrica \(d\). Denotaremos por \(\tau_d\) la colección de conjuntos abiertos respecto a \(d\).
\end{observacion}

\begin{definicion}
Dado \(A\subset X\), la \emph{clausura} de \(A\) (denotada \(\overline{A}\)) es la intersección de todos los cerrados que contienen \(A\). El \emph{interior} de \(A\) (denotado \(\operatorname{int}A\)) es la unión de todos los abiertos contenidos en \(A\). El \emph{frontera} o límite \(\partial A\) se define como \(\partial A=\overline{A}\setminus\operatorname{int}A\).
\end{definicion}

\section{Estructura topológica de un espacio métrico}

\begin{definicion}
La topología \(\tau_d\) inducida por la métrica \(d\) es
\[
\tau_d=\{U\subset X:\ \forall x\in U\ \exists r>0\text{ tal que }B(x,r)\subset U\}.
\]
\end{definicion}

\begin{proposicion}
Sea \((X,d)\) un espacio métrico. Entonces:
\begin{enumerate}
\item \((X,\tau_d)\) es un espacio topológico.
\item \((X,\tau_d)\) es Hausdorff: dados \(x\neq y\) existen bolas disjuntas que los separan.
\item \((X,\tau_d)\) es de primera numerabilidad: para cada \(x\in X\) la colección \(\{B(x,1/n):n\in\mathbb{N}\}\) es una base local numerable en \(x\).
\end{enumerate}
\end{proposicion}

\begin{observacion}
La unicidad del límite de sucesiones es consecuencia de la propiedad Hausdorff: si \(x_n\to x\) y \(x_n\to y\) entonces \(x=y\).
\end{observacion}

\section{Convergencia, continuidad y Cauchy}

\begin{definicion}
Una sucesión \((x_n)\subset X\) converge a \(x\in X\) si \(d(x_n,x)\to 0\) cuando \(n\to\infty\). Denotamos \(x_n\to x\).
\end{definicion}

\begin{definicion}
Una función \(f:(X,d_X)\to (Y,d_Y)\) entre espacios métricos es \emph{continua} en \(x\in X\) si para todo \(\varepsilon>0\) existe \(\delta>0\) tal que \(d_X(x,y)<\delta\) implica \(d_Y(f(x),f(y))<\varepsilon\). Equivalente: la preimagen de todo abierto es abierto.
\end{definicion}

\begin{definicion}[Sucesión de Cauchy y completitud]
Una sucesión \((x_n)\) en \((X,d)\) es de Cauchy si para todo \(\varepsilon>0\) existe \(N\) tal que \(m,n\ge N\) implica \(d(x_m,x_n)<\varepsilon\). El espacio métrico \(X\) es \emph{completo} si toda sucesión de Cauchy converge en \(X\).
\end{definicion}

\begin{proposicion}
En un espacio métrico, toda sucesión convergente es de Cauchy. La recíproca no siempre es cierta; cuando sí lo es decimos que el espacio es completo.
\end{proposicion}

\begin{proposicion}
Sea \(Y\) un subespacio de \(X\). Si \(X\) es completo y \(Y\) es cerrado en \(X\), entonces \(Y\) es completo con la métrica inducida. Reversamente, todo subespacio completo de \(X\) es cerrado en \(X\).
\end{proposicion}

\section{Equivalencias de métricas}

\begin{definicion}
Se dice que dos métricas \(d\) y \(d'\) en el mismo conjunto \(X\) son \emph{topológicamente equivalentes} si inducen la misma topología, es decir \(\tau_d=\tau_{d'}\).
\end{definicion}

\begin{definicion}
Las métricas \(d\) y \(d'\) son \emph{fuertemente equivalentes} (o \emph{bi-Lipschitz equivalentes}) si existen constantes positivas \(a,b>0\) tales que para todo \(x,y\in X\)
\[
a\,d(x,y)\le d'(x,y)\le b\,d(x,y).
\]
En tal caso inducen la misma topología y conservan propiedades métricas más rígidas.
\end{definicion}

\section{Espacios métricos acotados y normalizaciones}

\begin{definicion}
Un espacio métrico \((X,d)\) es \emph{acotado} si existe \(M>0\) tal que \(d(x,y)\le M\) para todo \(x,y\in X\). Dada una métrica \(d\), se puede definir una métrica equivalente acotada por ejemplo
\[
d'(x,y)=\frac{d(x,y)}{1+d(x,y)},
\]
que induce la misma topología que \(d\).
\end{definicion}

\section{Observaciones finales}

Las nociones básicas introducidas: bolas, abiertos, clausuras, convergencia, continuidad, completitud y equivalencia de métricas conforman la base del estudio de espacios métricos. A partir de aquí se desarrollan conceptos más avanzados como compacidad, completación, espacios de funciones con métricas inducidas por normas, y aplicaciones en análisis y geometría.

\end{document}