\documentclass{article}

\usepackage[utf8]{inputenc}
\usepackage[T1]{fontenc}
\usepackage[spanish]{babel}
\usepackage{amsmath,amssymb,amsthm}
\usepackage{geometry}
\usepackage{lmodern}
\usepackage{microtype}

\begin{document}
\section*{1. Primeras definiciones}

\subsection*{Definición 1.1}
Sea \(X\) un conjunto tal que \(X\neq\varnothing\). Una aplicación \(d:X\times X\to\mathbb{R}\) es una métrica o distancia definida en \(X\) si cumple las siguientes propiedades:
\begin{enumerate}
  \item[(D.1)] \(d(x,y)\ge 0\quad\forall\,x,y\in X.\)
  \item[(D.2)] \(d(x,y)=0\iff x=y\quad\forall\,x,y\in X.\)
  \item[(D.3)] \(d(x,y)=d(y,x)\quad\forall\,x,y\in X.\)
  \item[(D.4)] \(d(x,z)\le d(x,y)+d(y,z)\quad\forall\,x,y,z\in X.\)
\end{enumerate}
El par \((X,d)\) recibe el nombre de \emph{espacio métrico}. El número real \(d(x,y)\) recibe el nombre de \emph{distancia de \(x\) a \(y\)}. El axioma (D.4) suele llamarse \emph{desigualdad triangular}.

\subsection*{Definición 1.2}
Sean \((X,d)\) un espacio métrico y \(Y\neq\varnothing\) con \(Y\subseteq X\) un subconjunto. Se considera la restricción
\[
d|_{Y\times Y}:Y\times Y\longrightarrow\mathbb{R},\qquad (y,y')\mapsto d(y,y').
\]
La aplicación \(d|_{Y\times Y}\) es una métrica en \(Y\) que recibe el nombre de \emph{métrica inducida} por \(d\). El espacio métrico \((Y,d|_{Y\times Y})\) se llama \emph{subespacio métrico} de \((X,d)\).

\subsection*{Definición 1.3}
Sea \((X,d)\) un espacio métrico, \(x\in X\) y \(\varepsilon\in\mathbb{R}\) con \(\varepsilon>0\). Se definen:
\begin{align*}
B_d(x,\varepsilon)&:=\{y\in X:\ d(x,y)<\varepsilon\} &&\text{(bola abierta de centro \(x\) y radio \(\varepsilon\))},\\[4pt]
D_d(x,\varepsilon)&:=\{y\in X:\ d(x,y)\le\varepsilon\} &&\text{(disco cerrado de centro \(x\) y radio \(\varepsilon\))},\\[4pt]
S_d(x,\varepsilon)&:=\{y\in X:\ d(x,y)=\varepsilon\} &&\text{(esfera de centro \(x\) y radio \(\varepsilon\)).}
\end{align*}

\subsubsection*{Notas}
\begin{description}
  \item[Nota 1.1] Cuando no haya lugar a confusión se omitirá el subíndice que indica la métrica.
  \item[Nota 1.2] Propiedades elementales (para \(\varepsilon,\varepsilon_1,\varepsilon_2>0\)):
  \begin{enumerate}
    \item Si \(0<\varepsilon_1\le\varepsilon_2\) entonces \(B_d(x,\varepsilon_1)\subseteq B_d(x,\varepsilon_2)\).
    \item \(S_d(x,\varepsilon)\cap B_d(x,\varepsilon)=\varnothing\) para todo \(\varepsilon>0\).
    \item \(D_d(x,\varepsilon)=S_d(x,\varepsilon)\cup B_d(x,\varepsilon)\) para todo \(\varepsilon>0\).
  \end{enumerate}
\end{description}

\end{document}