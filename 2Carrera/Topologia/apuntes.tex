\documentclass[12pt]{article}
\usepackage[utf8]{inputenc}
\usepackage{amsmath}
\usepackage{amssymb}
\usepackage{enumerate}
\usepackage{enumitem}
\usepackage{geometry}
\usepackage{graphicx}
\usepackage{hyperref}
\usepackage{setspace}
\usepackage{xcolor}

% ----- Custom counters and commands -----
\newcounter{unit}[section]
\renewcommand{\theunit}{\arabic{unit}}

\newcounter{chapter}[subsection]
\renewcommand{\thechapter}{\arabic{chapter}}

\newcounter{definicion}[subsubsection]
\renewcommand{\thedefinicion}{\theunit.\arabic{definicion}}

\newcounter{nota}[subsubsection]
\renewcommand{\thenota}{\theunit.\arabic{nota}}

\newcounter{lema}[subsubsection]
\renewcommand{\thelema}{\theunit.\arabic{lema}}

\newcounter{teorema}[subsubsection]
\renewcommand{\theteorema}{\theunit.\arabic{teorema}}

\newcommand{\unit}[1]{
    \refstepcounter{unit}
    \section*{\Huge{\Roman{unit} #1}}
    \addcontentsline{toc}{section}{\Roman{unit} #1}
}

\newcommand{\chapter}[1]{
    \refstepcounter{chapter}
    \subsection*{\Large{\hspace{-10mm}\S \thechapter. #1}}
    \addcontentsline{toc}{subsection}{\thechapter. #1}
}

\newcommand{\definicion}{
  \refstepcounter{definicion}
  \subsubsection*{\large{Definición \thedefinicion}}
  \addcontentsline{toc}{subsubsection}{Definición \thedefinicion}
}

\newcommand{\nota}{
  \refstepcounter{nota}
  \subsubsection*{\large{Nota \thenota}}
  \addcontentsline{toc}{subsubsection}{Nota \thenota}
}

\newcommand{\lema}{
  \refstepcounter{lema}
  \subsubsection*{\large{Lema \thelema}}
  \addcontentsline{toc}{subsubsection}{Lema \thelema}
}

\newcommand{\teorema}[1]{
  \refstepcounter{teorema}
  \subsubsection*{\large{Teorema \theteorema \quad \textcolor{red}{#1}}}
  \addcontentsline{toc}{subsubsection}{Teorema \thelema \quad \textcolor{red}{#1}}
}

\newcommand{\dem}{
    \noindent \underline{\textbf{Demostración:}}
}


% ----------------------------------------

\begin{document}

\vfill
\begin{center}
    \Huge{Topología I} \\[3ex] 
    \huge Apuntes curso 2024 - 2025\\[3ex]
    \Large{Javier Ortín Rodenas}
\end{center}
\vspace{2cm}
\onehalfspacing

Estos apuntes tienen como objetivo recopilar los conceptos enseñados en la 
asignatura de Topología I en la Universidad de Oviedo durante el curso 2024-2025.
El índice de contenidos ha sido compartido por el equipo docente, así como algunas
de las demostraciones. Otras, en cambio, han sido realizadas por el autor de estos
apuntes por su cuenta.

\newpage
\tableofcontents
\newpage

\unit{Espacios métricos}
\vspace{3mm}

\chapter{Primeras definiciones}
\vspace{3mm}

\definicion
\vspace{1mm}\hspace{1mm}
Sea $X$ un conjunto no vacío ($X \neq \emptyset$), diremos que una aplicación
$d : X \times X \longrightarrow \mathbb{R}$ es una métrica (o distancia) definida en $X$
si cumple los siguientes axiomas:
\begin{enumerate}[label=D.\arabic*)]
    \item $d(x,y) \geq 0 \hspace{2mm} \forall x,y \in X$
    \item $d(x,y) = 0 \iff x = y \hspace{2mm} \forall x,y \in X$
    \item $d(x,y) = d(y,x) \hspace{2mm} \forall x,y \in X$
    \item $d(x,z) \leq d(x,y) + d(y,z) \hspace{2mm} \forall x,y,z \in X$
\end{enumerate}

\vspace{2mm}
El par $(X, d)$ recibe el nombre de "espacio métrico". El número real $d(x,y)$
recibe el nombre de "distancia de $x$ a $y$". Es común referirse a D.4) como
"desigualdad triangular".

\vspace{4ex}
\definicion
\vspace{1mm}\hspace{1mm}
Sean $(X, d_x)$ un espacio métrico, $Y \neq \emptyset $ con $Y \subseteq X$
un subconjunto. Se considera:
\begin{align*}
    d_{X|Y}:Y\times Y \longrightarrow \mathbb{R} &&
    (y, y') \longmapsto d_{X|Y}(y,y') = d_X(y,y')
\end{align*}

La aplicación $d_{X|Y}$ es una métrica que recibe el nombre de "métrica
inducida por $d_X$". El espacio métrico $(Y, d_{X|Y})$ recibe el nombre
de "subespacio métrico de $(X, d_x)$".

\newpage
\definicion
\vspace{1mm}\hspace{1mm}
Sean $(X, d_x)$ un espacio métrico, $x \in X, \varepsilon \in \mathbb{R}, \varepsilon \geq 0$,
se define:
\begin{itemize}
    \item $B_d(x, \varepsilon) = \{y \in X: d(x,y) < \varepsilon\}$ Bola (abierta) de centro $x$ y radio $\varepsilon$
    \item $D_d(x, \varepsilon) = \{y \in X: d(x,y) \leq \varepsilon\}$ Disco de centro $x$ y radio $\varepsilon$
    \item $S_d(x, \varepsilon) = \{y \in X: d(x,y) = \varepsilon\}$ Esfera de centro $x$ y radio $\varepsilon$
\end{itemize}

\nota
\vspace{1mm}\hspace{1mm}
Cuando no haya lugar a confusión, se omitirá el subíndice correspondiente a la métrica
para denotar a bolas, discos y esferas.

\nota
\vspace{1mm}\hspace{1mm}
Sean $(X, d_x)$ un espacio métrico, $x \in X$. Entonces, se cumple:
\begin{enumerate}[label=-]
    \item Si $\varepsilon_1, \varepsilon_2 \in \mathbb{R}$ tales que
        $0\leq \varepsilon_1 \leq \varepsilon_2$, entonces se tiene
        $B_d(x, \varepsilon_1) \subseteq B_d(x, \varepsilon_2)$
    \item $S_d(x, \varepsilon) \cap B_d(x, \varepsilon) = \emptyset
        \hspace{3mm} \forall \varepsilon \in \mathbb{R}: \varepsilon \geq 0$
    \item $D_d(x, \varepsilon) = S_d(x, \varepsilon) \cup B_d(x, \varepsilon)
        \hspace{3mm} \forall \varepsilon \in \mathbb{R}: \varepsilon \geq 0$
\end{enumerate}

\newpage
\chapter{Estructura topológica de un espacio métrico}
\vspace{3mm}

\lema
\vspace{1mm}\hspace{1mm}
Sea $(X, d)$ un espacio métrico, $x_0 \in X$, $\varepsilon \in \mathbb{R}$
con  $\varepsilon > 0$, $x \in B_d(x_0, \varepsilon)$. Se verifica que
$\exists \mu \in \mathbb{R}$ con $\mu > 0$ tal que
$B_d(x, \mu) \subseteq B_d(x_0, \varepsilon)$. \\[3ex]
\underline{\textbf{Demostración:}} Nótese la importancia de la desigualdad estricta $\varepsilon, \mu > 0$.

\vspace{2mm} \doublespacing
Por hipótesis, $x_0 \in X$, $\varepsilon \in \mathbb{R}$ y $\varepsilon > 0$, luego $B_d(x_0, \varepsilon)$
está bien definida. \\ Además, $x \in B_d(x_0, \varepsilon) \Rightarrow d(x,x_0) < \varepsilon$ por hipótesis
y por definición de bola. \\ De este modo, podemos definir $\mu = \varepsilon - d(x,x_0) > 0$ por el paso anterior. \\
Así, $B_d(x, \mu)$ está bien definida. Queda ver que $B_d(x, \mu) \subseteq B_d(x_0, \varepsilon)$.

\vspace{2mm}
Sea $y \in B_d(x, \mu)$; por definición, $d(x, y) = \mu = \varepsilon - d(x, x_0)$. Por (D.4), \\
$d(x_0, y) \leq d(x_0, x) + d(x, y) < d(x_0, x) + \mu = d(x_0, x) + \varepsilon - d(x_0, x) = \varepsilon$ \\
Así, $y \in B_d(x, \mu) \Rightarrow y \in B_d(x_0, \varepsilon)$, luego $B_d(x, \mu) \subseteq B_d(x_0, \varepsilon)$.

\vspace{3mm}

\definicion
\vspace{1mm}\hspace{1mm}
Sean $(X,d)$ un espacio métrico, $U \subseteq X$. Se dice que $U$ es un d-abierto si: \vspace{-4mm}
$$\forall x \in U \hspace{2mm} \exists \mu \in \mathbb{R}, \hspace{2mm} \mu > 0
\hspace{3mm} \text{tal que} \hspace{3mm} B_d(x, \mu) \subseteq U$$

\nota
\vspace{1mm}\hspace{1mm}
En la definición anterior, el valor de $\mu$ depende de cada $x$, no es un valor único.

\newpage
\teorema{Estructura topológica de un espacio métrico}
Sea $(X,d)$ un espacio métrico, se verifica que:
\begin{enumerate}[label=(\arabic*)]
    \item $X$, $\emptyset$ son d-abiertos
    \item Sean $\{U_i\}_{i \in I}$ todos d-abiertos. Entonces, $U = \bigcup_{i \in I}U_i$ es también un d-abierto.
    \item Si $\{U_1, \dots U_n\}$ son d-abiertos; entonces, $U = \bigcap_{i=1}^nU_i$ es un d-abierto.
\end{enumerate}

\dem \\ \onehalfspacing
(1) Buscamos demostrar que $X$ es un d-abierto; es decir, buscamos ver: \vspace{-2mm}
$$\forall x \in X \hspace{2mm} \exists \mu \in \mathbb{R},
\mu > 0: B_d(x, \mu) \subseteq X$$ \\[-8ex]

Sea $x \in X$ cualquiera y $\mu > 0$ cualquiera,
$B_d(x, \mu) = \{y \in X: d(x,y) < \mu\}$. \\
Por contrarrecíproco, $y \notin X \Rightarrow y \notin B_d(x, \mu)$,
luego $B_d(x, \mu) \subseteq X$. \\
Tenemos que $X$ es un d-abierto, falta ver $\emptyset$. Para ello, reescribiremos
la definición de d-abierto de la siguiente forma: \vspace{-5mm}
$$\emptyset \text{ es un d-abierto} \iff 
\nexists x\in \emptyset : \forall \mu > 0 : B_d(x,\mu) \nsubseteq \emptyset$$ \\[-5ex]
Como $\nexists x \in \emptyset$, se cumple trivialmente. Así, $\emptyset$ es un d-abierto.
\\[6ex]
\noindent
(2) Sea $\{U_i\}_{i \in I}$ una colección arbitraria de d-abiertos, si
$U_i = \emptyset \hspace{2mm} \forall i \in I$, entonces  se tiene $\bigcup_{i \in I}U_i = \emptyset$.
En tal caso, la unión es un d-abierto como consecuendia del apartado anterior. \vspace{4mm}

Supongamos que $\bigcup_{i\in I}U_i \neq \emptyset$. Así, sea $x \in \bigcup_{i\in I}U_i$
 por definición, $\exists j \in I: x \in U_j$. \\
Por hipótesis, $U_j$ es un d-abierto, luego $\exists \mu > 0:$ $B_d(x,\mu) \subseteq U_j \subseteq \bigcup_{i\in I}U_i$.
\\ De este modo, $\bigcup_{i\in I}U_i$ es también un d-abierto.
\newpage
\noindent
(3) Sea $\{U_1, \dots, U_n\}$ una colección finita de d-abiertos, veamos que su intersección
$\bigcap_{i = 1}^n U_i$ es también un d-abierto. En caso de ser vacía la intersección, se cumple
trivialmente.

Supongamos ahora que $\bigcap_{i = 1}^n U_i \neq \emptyset$. Así, podemos tomar $x \in \bigcap_{i = 1}^n U_i$. \\
Por definición, $x \in U_i$ $\forall i \in \{1, \dots, n\}$. Al ser d-abiertos por hipótesis, se cumple: 
$$\forall i \in \{1, \dots, n\} \hspace{2mm} \exists \mu_i > 0 : B_d(x, \mu_i) \subseteq U_i$$
\end{document}
