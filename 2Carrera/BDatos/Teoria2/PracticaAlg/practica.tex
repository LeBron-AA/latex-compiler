\documentclass{article}
\usepackage{enumitem}
\usepackage{amssymb}
\usepackage{amsmath}
\usepackage{enumitem}
\usepackage{multicol}


\title{Relational Algebra Exercises}   
\author{Javier Ortín Rodenas}
\begin{document}
\maketitle
\noindent
\newpage
\section{Database system concepts chapter 6}
\vspace{2ex}
\subsection{Exercise}
Find the name of all instructors in the "Physics" department:
$$\Pi_{\text{name}}\left(\sigma_{\text{dept\_name = "Physics"}}(\text{instructor})\right)$$

\subsection{Exercise}
Consider a query to find the set of all courses taught in the Fall 2009 semester, the
Spring 2010 semester, or both.
$$\Pi_{\text{course\_id}}\left(
    \sigma_{\text{
            (year = 2009} \wedge \text{semester = "Fall")}
        \vee (
            \text{year = 2010} \wedge \text{semester = "Spring"}
        )}(\text{section})\right)$$

\vspace{2ex}
\noindent
Alternatively, we could use the set union as follows:

$$\Pi_{\text{course\_id}}\left(
            \sigma_{\text{year = 2009} \wedge
            \text{semester = "Fall"}}(\text{section})\right) $$
            $$\bigcup $$
            $$\Pi_{\text{course\_id}}\left(
                \sigma_{\text{year = 2010} \wedge \text{semester = "Spring"}}(\text{section})\right)$$

\vspace{2ex}
\subsection{Exercise}
Find all courses of Fall 2009 that were not in Spring 2010:
$$\Pi_{\text{course\_id}}\left(
            \sigma_{\text{year = 2009} \wedge
            \text{semester = "Fall"}}(\text{section})\right) $$
            $$\boldsymbol{-} $$
            $$\Pi_{\text{course\_id}}\left(
                \sigma_{\text{year = 2010} \wedge \text{semester = "Spring"}}(\text{section})\right)$$
\vspace{2mm}
\noindent In this case, we use the we use the ser difference instead of the union

\vspace{2ex}
\subsection{Exercise}
Find the name of all instructors in the Physics department, along with
all the courses they have taught.
\begin{flalign*}
    \Pi_{\text{name, course\_id}}\left(
        \sigma_{\substack{\text{dept\_name = "Physics"} \\
        \wedge \\ \text{instructor.id = teaches.id}}} \left(
            \text{instructor} \times \text{teaches}
        \right)
    \right)
\end{flalign*}

\newpage
\subsection{Exercise}
\noindent Find the highest salary in the university.
    $$\Pi_{\text{salary}}\left(\text{instructor}\right)$$ \\[-7ex]
    $$\boldsymbol{-}$$ \\[-6ex]
    $$\Pi_{i_1.\text{salary}}\left(
        \sigma_{i_1.\text{salary} < i_2.\text{salary}} \left(
            \rho_{i_1}(instructor) \times \rho_{i_2}(instructor)
        \right)
    \right)$$

\vspace{2mm}
We first calculate the salaries that are not the maximum and then
remove them from the total set. Hence, we are left with the highest salary in
the university. 

\vspace{4mm}
\subsection{Exercise}
Find the set of courses taught in both Fall 2009 and Spring 2010
$$\Pi_{\text{course\_id}}\left(
            \sigma_{\text{year = 2009} \wedge
            \text{semester = "Fall"}}(\text{section})\right) $$
$$\bigcap $$
$$\Pi_{\text{course\_id}}\left(
    \sigma_{\text{year = 2010} \wedge \text{semester = "Spring"}}(\text{section})\right)$$

\vspace{2mm}
Even though the intersection of sets is not a fundamental relational algebra
operation, we can obtain it through the use of union and set difference operations
as a result of DeMorgan's laws. The $\cap$ notation can be used to simplify this process.

\vspace{4mm}
\subsection{Exercise}
\noindent Find the name of all the instructors as well as all the courses they have thougt.
$$\Pi_{instructor\_ID, course\_ID}(\text{instructor} \Join \text{teaches}) $$
\vspace{2mm} \noindent Simpler way to solve a previous ecercise using the "natural join" operation.

\vspace{4mm}
\subsection{Exercise}
\noindent Find the names of all the structures in the Comp. Sci. department
together with all the course titles of the courses they have tought.
$$
\Pi_{\text{name, title}}\Big(
    \big(\sigma_{\text{dept\_name = "CompSci"}}(\text{instructor})\big)
    \Join \text{teaches} \Join \text{course}
\Big)
$$

\newpage
\section{Practice exercises by ChatGPT}
\subsection{Schema of reference}
\noindent Following our previous theme, we shall consider the following schema:
\begin{itemize}
    \item Student(SID, Name, Major)
    \item Course(CID, Title, Department)  
    \item Enrolled(SID, CID, Grade)
    \item Professor(PID, Name, Department)
    \item Teaches(PID, CID)
\end{itemize}

\vspace{4mm}
\subsection{Exercise}
\noindent Find the names of students majoring in "Computer Science"
$$ \Pi_{\text{nme}}\left(
    \sigma_{\text{major = "CompSci"}}(
        \text{student})
\right)$$

\vspace{4mm}
\subsection{Exercise}
\noindent List the title of courses taught by professors
from the "Mathematics" department. \vspace{1mm}
$$ \Pi_{\text{title}}\Big(
    \sigma_{\text{department = "Mathematics"}}(
        \text{professor}) \Join
    \text{teaches} \Join
    \text{course}
\Big) $$

\vspace{4mm}
\subsection{Exercise}
\noindent Find the name of students who are enrolled in at least
one course taught by "Dr. Smith".
$$ A \hspace{1mm} := \hspace{1mm}
\Pi_{\text{PID}}\left(
    \sigma_{\text{name = "Smith"}}(
        \text{professor})
\right) $$
$$ B \hspace{1mm} := \hspace{1mm}
\Pi_{\text{CID}}\left(
    \sigma_{\text{PID} \in A}(
        \text{teaches})
\right) $$
$$ \Pi_{\text{name}}\left(
    \sigma_{\text{CID} \in B}(
        \text{Enrolled}) \Join
    \text{Student}
\right)$$

\vspace{2mm}
We use the subquery notation to visually
simplify the query, but this could be done in a single step.
In this context, A represents Smith's ID (there could theoretically
be several of them), and B represents the IDs of the subjects taught
by Smith. 

\newpage
\subsection{Exercise}
\noindent Find the names of professors who teach at least one course
with no enrolled students.

$$A \hspace{1mm} := \hspace{1mm}
\Pi_{\text{CID}}\left(\text{course}\right)
\boldsymbol{-}
\Pi_{\text{CID}}\left(\text{enrolled}\right)$$
$$ \Pi_{\text{name}}\left(
    \sigma_{\text{CID} \in A}(
        \text{teaches}) \Join
    \text{professor}
\right)$$
In a similar fashion to our last exmple, we use the subquery notation.
We check what courses have no students enrolled and save their CIDs in
the set A. We then look which professors teach these subjects.

\vspace{4mm}
\subsection{Exercise}
\noindent Find the names of students who have taken all courses offered by
the Physics department.

$$ \Pi_{\text{name}}\left(
    \sigma_{B \subseteq A_x}(
        \rho_x(\text{student}))
\right)$$
$$ A_x \hspace{1mm} := \hspace{1mm}
\Pi_{\text{CID}}\left(
    \sigma_{\text{SID = x.SID}}(
        \text{enrolled})
\right)$$
$$ B \hspace{1mm} := \hspace{1mm}
\Pi_{\text{CID}}\left(
    \sigma_{\text{department = "Physics"}}(
        \text{course})
\right)$$
\end{document}