\documentclass{article}
\usepackage{enumitem}
\usepackage{amssymb}
\usepackage{amsmath}
\usepackage{enumitem}
\usepackage{multicol}
\usepackage{listings}
\usepackage{xcolor}

\lstdefinelanguage{SQL}{
    morekeywords={SELECT, FROM, WHERE, JOIN, INNER, LEFT, RIGHT, ON,
        AS, AND, OR, NOT, IN, EXISTS, NULL, COUNT, GROUP, BY, ORDER,
        LIMIT, INSERT, INTO, VALUES, UPDATE, SET, DELETE, HAVING,
        NATURAL, EXCEPT},
    sensitive=false,
    morecomment=[l]--,
    morestring=[b]',
}

\lstset{
    language=SQL,
    basicstyle=\ttfamily\small,
    keywordstyle=\color{blue}\bfseries,
    commentstyle=\color{gray},
    stringstyle=\color{teal},
    showstringspaces=false,
    breaklines=true,
    columns=fullflexible,
}

\title{SQL Exercises}   
\author{Javier Ortín Rodenas}
\begin{document}
\maketitle
\noindent
\newpage
\section{Database system concepts chapter 3}
\vspace{2ex}
\subsection{Exercise}
Find the name of all instructors with a higher salary than all the instructors
in the 'Biology' department.
\vspace{2mm}
\begin{lstlisting}
    SELECT DISTINCT name 
    FROM instructor
    WHERE salary > all(
        select distinct salary
        from instructor
        where dept_name = 'Biology');
\end{lstlisting}

\vspace{4mm}
\subsection{Exercise}
\noindent Find all courses taught in both the
Fall 2017 semester and in the Spring 2018 semester.

\begin{lstlisting}
    SELECT course_id
    FROM section s
    WHERE semester = 'Fall' and year = 2017
    and EXISTS (
        select course_id from section
        where semester = 'Spring' and year = 2018
        and course_id = s.course_id);
\end{lstlisting}

\vspace{4mm}
\subsection{Exercise}
\noindent Find all students who have taken all courses offered in the Biology
department.

\begin{lstlisting}
    SELECT * 
    FROM student s
    WHERE NOT EXISTS (
        SELECT course_id
        FROM course c
        WHERE dept_name = 'Biology'
        and NOT EXISTS (
            SELECT *
            FROM takes
            WHERE student_id = s.student_id
            and course_id = c.course_id
        )
    )
\end{lstlisting}

We could write an equivalent query as follows:
\begin{lstlisting}
    SELECT * 
    FROM student s
    WHERE NOT EXISTS (
        (SELECT course_id
        FROM course c
        WHERE dept_name = 'Biology')
        EXCEPT
        (SELECT course_id
        FROM takes
        WHERE student_id = s.student_id)
    )
\end{lstlisting}
\noindent This query will return the same result as the one above it.

\vspace{4mm}
\subsection{Exercise}
\noindent Find all courses that were offered at most once in 2017.
\begin{lstlisting}
    SELECT *
    FROM course c
    WHERE UNIQUE(
        SELECT course_id
        FROM section
        WHERE year = 2017 and course_id = c.course_id
    );
\end{lstlisting}

\vspace{4mm}
\subsection{Exercise}
\noindent Find the average instructors’ salaries of those departments where the
average salary is greater than \$42,000.
\begin{lstlisting}
    SELECT AVG(salary) as avg_sal
    FROM instructor
    WHERE dept_name in (
        SELECT distinct dept_name
        FROM instructor
        GROUP BY dept_name
        HAVING AVG(salary) > 42000
    );
\end{lstlisting}
We must find out which departments have an average salary greater than
\$42.000 and then calculate the average salary of the instructors from
those departments.
\newpage
\subsection{Exercise}
\noindent  Find all departments where the total salary is greater than the average of
the total salary at all departments.
\begin{lstlisting}
    SELECT dept_name, SUM(salary) AS total_salary
    FROM instructor
    GROUP BY dept_name
    HAVING SUM(salary) > (
        SELECT AVG(dept_total)
        FROM (
            SELECT SUM(salary) AS dept_total
            FROM instructor
            GROUP BY dept_name
        ) AS dept_totals
    );    
\end{lstlisting}

\vspace{4mm}
\subsection{Exercise}
\noindent List all departments along with the number of instructors in each
department
\begin{lstlisting}
    SELECT dept_name, COUNT(*) as n_instructors
    FROM instructor
    GROUP BY dept_name;
\end{lstlisting}

\newpage
\section{SQL 2019 Exam}
\subsection{Exercise}
\noindent Cambiar todas las medidas con estado "inscorpleta" de los senssres del
módulo "MA" tomadas en la calle Uria der pués de lan 17:00, para que estado sea ok.
\begin{lstlisting}
    UPDATE MEDIDAS
    SET estado = 'ok'
    WHERE id_ubicacion in (select id_ubicacion
        from ubicacion where calle = 'Uria')
    and id_modulo in (select id_modulo
        from modulo where nombre_modulo = 'MA')
    and estado = 'incompleta'
    and hora > '17:00';
\end{lstlisting}

\vspace{4mm}
\subsection{Exercise}
\noindent Identificación y descripción de los sensores que tengan el mayor número de
medidas realisades dentiro de cada tipo de sensores (imprimir también el tipo).
\begin{lstlisting}
    SELECT id_modulo, id_sensor, descripcion_sensor, tipo
    FROM medida as m NATURAL JOIN sensor as s
    GROUP BY m.id_modulo, m.id_sensor, s.descripcion_sensor, s.tipo
    HAVING COUNT(*) >= ALL (
        SELECT COUNT(*)
        FROM medida NATURAL JOIN sensor as s2
        WHERE s2.tipo = s.tipo
        GROUP BY id_modulo, id_sensor
    );
\end{lstlisting}

\vspace{4mm}
\subsection{Exercise}
\noindent Calles en las que nunca se ha realizado una medida de tipo"temperatura"
con resultads "fallo". También hacer en álgebre relacional
\begin{lstlisting}
    (SELECT calle
    FROM ubicacion)
    EXCEPT
    (SELECT calle
    FROM ubicacion
    WHERE id_ubicacion in(
        SELECT id_ubicacion
        FROM medida as m NATURAL JOIN sensor as s
        WHERE m.estado = "fallo"
        and s.tipo = "temperatura"
    ));
\end{lstlisting}
\noindent Veamos cómoo hacer la consulta en álgebra relacional:
$$ \Pi_{\text{calle}}\left(\text{ubicacion}\right) $$
$$\boldsymbol{-}$$
$$ \Pi_{\text{calle}}\Big(
    \sigma_{\text{estado = "fallo"}}(
        \text{medida}) \Join
    \sigma_{\text{tipo = "temperatura"}}(
        \text{sensor}) \Join
    \text{ubicacion}
\Big) $$

\vspace{4mm}
\subsection{Exercise}
\noindent Identificación del módulo y nombre del responsable de aquellos
módulos tales que todas las medidas que han realizado sus sensores del tipo
"humedad" han tenido valor mayor a 100.
\begin{lstlisting}
    SELECT id_modulo, nombre_responsable
    FROM modulo mod
    WHERE NOT EXISTS (
        SELECT *
        FROM sensor as s NATURAL JOIN medida as med
        WHERE s.id_modulo = mod.id_modulo
        and s.tipo = 'humedad'
        and med.valor <= 100
    );
\end{lstlisting}
\end{document}