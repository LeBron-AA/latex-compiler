\documentclass{article}
\usepackage{enumitem}
\usepackage{amssymb}
\usepackage{amsmath}
\usepackage{enumitem}
\usepackage{multicol}

\newcommand{\sech}[1]{\operatorname{sech} #1}


\title{TG IV}   
\author{
    \large{Grupo S} \\\\
    Javier Ortín Rodenas \\
    Bruno Martín Rivera \\
    Jorge Gota Ortín \\
    Alejandra Sánchez Mayo
}
\date{}
\begin{document}
\maketitle
\noindent
\section{Enunciado:}
\begin{enumerate}
    \item Dado $C$ el cilindro $x^2 + y^2 = 1$, considere el difeomorfismo siguiente:
        \begin{align*}
            \phi_1 : C \longrightarrow \mathbb{S}^2\backslash\{S, N\} &&
            (x,y,z) \longmapsto \phi_1(x,y,z) = \left(x \sech{z}, y \sech{z}, \tanh{z}\right)
        \end{align*}
        ¿Preserva ángulos? ¿Es una isometría local? ¿Es isoareal?
        \\
    \item Dado $C_t$ el cilindro truncado $\{(x,y,z) \in \mathbb{R}^3 : x^2 + y^2 + z^2 = 1, -1<z<1\}$,
        considere el difeomorfismo $\phi_2$ definido como sigue:
        \begin{align*}
            \phi_2 : C_t \longrightarrow \mathbb{S}^2\backslash\{S, N\} &&
            (x,y,z) \longmapsto \phi_2(x,y,z) = \left(x \sqrt{1-z^2}, y\sqrt{1-z^2} ,z\right)
        \end{align*}           
        ¿Preserva áreas? ¿Es una isometría local? ¿Preserva ángulos?
\end{enumerate}

\section{Solución}
\subsection{Primer apartado}
\hspace{1mm} En primer lugar, debemos de parametrizar el cilindro $C$.
Podemos hacerlo a partir de la siguiente función:
\begin{align*}
    X : [0, 2\pi) \times \mathbb{R} \longrightarrow C&&
    (\theta, z) \longmapsto X(\theta, z) = \left(\cos\theta, \sin\theta, z\right)
\end{align*}
\vspace{1mm}
Veamos qué expresión toma $\bar{X} = (\phi \circ X)$ al componerlas:
\begin{align*}
    \bar{X} : [0, 2\pi) \times \mathbb{R} \longrightarrow \mathbb{S}^2 \backslash \{S,N\}&&
    (\theta, z) \longmapsto \bar{X}(\theta, z) =
        \left(\frac{\cos \theta}{\cosh z}, \frac{\sin \theta}{\cosh z}, \tanh z\right)
\end{align*}

\newpage
Para comprobar si la aplicación $\phi$ preserva ángulos, tenemos que comprobar si es o no conforme.
Por tanto, hemos de hallar los coeficientes de la Primera Forma Fundamental de $X$ y de $\bar{X}$
para intentar definir así una aplicación $\lambda$ adecuada.

\vspace{3mm}
\noindent Hallemos las derivadas parciales de $X$:
\begin{align*}
    X_\theta (\theta, z) = (-\sin \theta, \cos \theta, 0) &&
    X_z (\theta, z) = (0,0,1)
\end{align*}
\noindent De este modo, los coeficientes de la PFF de $X$ vienen dados por:
\begin{flalign*}
    E(\theta, x) & = X_\theta (\theta, z) \boldsymbol{\cdot} X_\theta (\theta, z)
     = (-\sin \theta, \cos \theta, 0) \boldsymbol{\cdot}  (-\sin \theta, \cos \theta, 0)
     = \sin^2\theta + \cos^2\theta = 1 \\
    F(\theta, x) &= X_\theta (\theta, z) \boldsymbol{\cdot} X_z (\theta, z)
     = (-\sin \theta, \cos \theta, 0) \boldsymbol{\cdot}  (0,0,1) = 0 && \\
    G(\theta, x) &= X_z (\theta, z) \boldsymbol{\cdot} X_z (\theta, z)
     = (0,0,1) \boldsymbol{\cdot} (0,0,1) = 1
\end{flalign*}

\vspace{3mm}
\noindent Veamos qué ocurre con $\bar{X}$. Cabe recordar que $\frac{d}{dx}\cosh x = \sinh x$.
\begin{align*}
    \bar{X}_\theta (\theta, x) = \left(\frac{-\sin \theta}{\cosh z}, \frac{\cos \theta}{\cosh z}, 0\right) &&
    \bar{X}_z (\theta, x) = \left(\frac{-\cos \theta \cdot \sinh z}{\cosh^2z}, \frac{-\sin \theta \sinh z}{\cosh^2 z}, \frac{1}{\cosh^2 z}\right)
\end{align*}

\vspace{2mm}
\noindent Calculamos ahora los coeficientes de la PFF de $\bar{X}$:
\begin{flalign*}
    \bar{E}(\theta, x) & = \bar{X}_\theta (\theta, z) \boldsymbol{\cdot} \bar{X}_\theta (\theta, z)
     = \frac{1}{\cosh^2 z} (\sin^2 \theta + \cos^2 \theta)
     = \frac{1}{\cosh^2 z}\\[2ex]
    \bar{F}(\theta, x) &= \bar{X}_\theta (\theta, z) \boldsymbol{\cdot} \bar{X}_z (\theta, z)
     = \frac{\cos \theta \cdot \sin \theta}{\cosh^2 z}(1-1) = 0 && \\[2ex]
    \bar{G}(\theta, x) &= \bar{X}_z (\theta, z) \boldsymbol{\cdot} \bar{X}_z (\theta, z)
     = \frac{1}{\cosh^4z} \left(\cos^2\theta \cdot \sinh^2z + \sin^2\theta \cdot \sinh^2z + 1\right) = \\
     &= \frac{1}{\cosh^4z} \left(\sinh^2z + 1\right) = \frac{1}{\cosh^4z} \left(\cosh^2z\right) = \frac{1}{\cosh^2z}
\end{flalign*}

\vspace{3mm}
En vista de lo anterior, es claro que $\phi_1$ preserva ángulos, pues es una aplicación conforme.
Basta tomar la aplicación $\lambda$ definida como sigue:
\begin{align*}
    \lambda : C \longrightarrow \mathbb{R} &&
    p = X(\theta, z) \longmapsto \lambda(p) = \frac{1}{\cosh z}
    = \frac{2}{e^x + e^{-x}} \neq 0 \hspace{2ex} \forall z \in \mathbb{R}
\end{align*}

Tenemos que $\lambda$ no se anula y es diferenciable al ser composición
de funciones diferenciables. Además, verifica las siguientes tres ecuaciones:
\begin{align*}
    \bar{E} = \lambda(p)^2 E &&
    \bar{E} = \lambda(p)^2 E &&
    \bar{E} = \lambda(p)^2 E
\end{align*}
\newpage
Por un resultado teórico, sabemos que se cumple lo siguiente:
$$ \phi_1 \text{ isoareal} \iff EG - F^2 = \bar{E}\bar{G} - \bar{F}$$
\noindent Veamos si se cumple o no la igualdad:
\begin{flalign*}
    &&&EG - F^2 = 1 \cdot 1 - 0^2 = 1 &&&&&&&&&&&&&&\\
    &&&\bar{E}\bar{G} - \bar{F} = \frac{1}{\cosh^2 z} \cdot \frac{1}{\cosh^2 z} - 0
    = \frac{1}{\cosh^4 z} = \frac{16}{(e^z + e^{-z})^4}
\end{flalign*}

\vspace{2mm}
La igualdad no se cumple para $z \neq 1$ luego $\phi_2$ no es isoareal.
Además, podemos concluir que no es una isometría gracias al siguiente resultado:
$$ \phi_1 \text{ isometría} \iff \phi_1 \text{ conforme e isoareal}$$
Al no ser isoareal, no puede ser una isometría como consecuencia de esta caracterización.

\newpage
\subsection{Segundo apartado}
El cilindro $C_t$ es un subconjunto del cilindro $C$ del apartado anterior.
Por tanto, podemos utilizar la misma parametrización restringiendo los valores
que puede tomar la tercera componente.
\begin{align*}
    X : [0, 2\pi) \times (-1,1) \longrightarrow C&&
    (\theta, z) \longmapsto X(\theta, z) = \left(\cos\theta, \sin\theta, z\right)
\end{align*}
Al no haber modificado la expresión de $X$ nos sirve el razonamiento del apartado anterior para deducir:
\begin{align*}
    E(\theta, z) = 1 &&
    F(\theta, z) = 0 &&
    G(\theta, z) = 1
\end{align*}

\vspace{2mm} \noindent
Veamos qué expresión tiene ahora $\bar{X} = (\phi_2 \circ X)$:
\begin{align*}
    \bar{X}: [0,2\pi) \times (-1,1) \longrightarrow \mathbb{S}^2 \backslash \{S,N\} &&
    (\theta, z) \longmapsto \bar{X}(\theta, z) = \left(\cos\theta \sqrt{1-z^2},\sin\theta \sqrt{1-z^2},z\right)
\end{align*}

\vspace{2mm}
\noindent Hallemos ahora sus derivadas parciales:
\begin{align*}
    \bar{X}_\theta(\theta, z) = \left(-\sin\theta \sqrt{1-z^2},\cos\theta \sqrt{1-z^2},0\right) &&
    \bar{X}_z(\theta, z) = \left(\frac{-z\cos\theta}{1-z^2},\frac{-z\sin\theta}{1-z^2},1\right)
\end{align*}

\vspace{2mm}
\noindent
De este modo, los coeficientes de la PFF de $\bar{X}$ vienen dados por:
\begin{flalign*}
    \bar{E}(\theta, x) & = \bar{X}_\theta (\theta, z) \boldsymbol{\cdot} \bar{X}_\theta (\theta, z)
     = (1-z^2)(\sin^2 \theta + \cos^2 \theta) = 1-z^2\\[2ex]
    \bar{F}(\theta, x) &= \bar{X}_\theta (\theta, z) \boldsymbol{\cdot} \bar{X}_z (\theta, z)
     = \frac{z\cos\theta \cdot \sin\theta}{\sqrt{1-z^2}}(1-1+ 0) = 0 &&\\[1ex]
    \bar{G}(\theta, x) &= \bar{X}_z (\theta, z) \boldsymbol{\cdot} \bar{X}_z (\theta, z)
     = \frac{z^2}{1-z^2} \left(\cos^2\theta + \sin^2\theta\right) + 1 = \frac{1}{1-z^2}
\end{flalign*}

\vspace{2mm}
Veamos que $\phi_2$ no es conforme por reducción al absurdo. Supongamos que sí lo es; en tal caso,
existiría una función $\lambda : S_1 \longrightarrow \mathbb{R}$ tal que:
\begin{align*}
    \bar{E} = \lambda(p)^2 E &&
    \bar{F} = \lambda(p)^2 F &&
    \bar{G} = \lambda(p)^2 G &&
\end{align*}


De este modo, como $\bar{E} = 1 - z^2$ y $E = 1$, necesariamente ha de cumplirse
$\lambda(p)^2 = 1-z^2$ para satisfacer la condición $\bar{E} = \lambda(p)^2 E$. No
obstante; esto nos genera un problema, pues al imponer esta restricción puede no cumplirse
$\bar{G} = \lambda(p)^2 G$, pues $G = 1$ y sabemos $\bar{G} = \frac{1}{1-z^2}$. Por todo lo anterior:
\begin{flalign*}
    \bar{E} = &\lambda(p)^2 E \Rightarrow \lambda(p)^2 = 1 - z^2
    \Rightarrow \lambda(p)^2 G = (1-z^2) 1 =
     1 - z^2 \neq \bar{G} = \frac{1}{1-z^2}
\end{flalign*} 
Por todo lo anterior, ninguna aplicación $\lambda$ puede satisfacer simultáneamente
$\bar{E} = \lambda(p)^2 E$ y $\bar{G} = \lambda(p)^2 G$. Hemos llegado a una contradicción,
luego $\phi_2$ no puede ser conforme.

\vspace{2mm}
Como consecuencia, al no ser conforme $\phi_2$ no puede ser una isometría (pues ser isometría equivale
a ser simultáneamente conforme e isoareal). No obstante, sí
podría preservar áreas. Para comprobarlo basta ver si se cumple la igualdad $EG-F^2 = \bar{E}\bar{G} - \bar{F}^2$.
\begin{flalign*}
&&&EG-F^2=  1 \cdot 1 - 0^2 =  1 &&&&&&&&&&&&&&&&\\
&&&\bar{E}\bar{G} - \bar{F}^2 = (1-z^2) \cdot \frac{1}{1-z^2} - 0^2 = 1
\end{flalign*}

\vspace{2mm} 
\noindent Por todo lo anterior, $\phi_2$ preserva áreas pero no ángulos, luego no es una isometría.
\end{document}