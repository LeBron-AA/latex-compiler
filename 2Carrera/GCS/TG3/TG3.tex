\documentclass{article}
\usepackage{enumitem}
\usepackage{amssymb}
\usepackage{amsmath}
\usepackage{enumitem}
\usepackage{multicol}

\title{TG III}   
\author{
    \large{Grupo M} \\\\
    Javier Ortín Rodenas \\
    Bruno Martín Rivera \\
    Jorge Gota Ortín \\
    Alejandra Sánchez Mayo
}
\date{}
\begin{document}
\maketitle
\noindent
\section{Enunciado:}
\begin{enumerate}
    \item Dado el paraboloide hiperbólico $z = y^2 - x^2$, halle su imagen esférica.
        Para el punto $P = (0,0,0)$, halle las curvaturas normales en las direcciones
        (principales) $v_1 = (1,0,0)$ y $v_2 = (0,1,0)$. ¿El punto $P$ es umbilical?
        Halle la curvatura de $P$ en la dirección $v_3 = \left(\frac{1}{\sqrt{2}}, \frac{1}{\sqrt{2}}, 0 \right)$.
    \item Dada la superficie $z = x^2 + y^3$, clasifique sus puntos en elípticos, hiperbólicos, parabólicos o planos.
\end{enumerate}

\section{Solución}
\subsection{Primer apartado}
\hspace{1mm} En primer lugar, debemos parametrizar el paraboloide hiperbólico en cuestión.
Nótese que es la gráfica de la siguiente función $f$:
\begin{align*}
    f : \mathbb{R}^2 \longrightarrow \mathbb{R} \hspace{3ex}
    & (x,y) \longmapsto f(x,y) = y^2 - x^2
\end{align*}
\noindent
Por tanto, podemos definir la siguiente parametrización:
\begin{align*}
    X : \mathcal{U} = \mathbb{R}^2 \longrightarrow \mathbb{R}^3 \hspace{3ex}
    & (u,v) \longmapsto X(u,v) = (u,v,v^2 - u^2)
\end{align*}

\vspace{2mm}
Para hallar la imagen esférica del paraboloide, es necesario calcular el vector normal a esta
superficie en cada punto de la misma. Para ello debemos conocer las derivadas parciales de la
parametrización. Derivando en la expresión de $X$ obtenemos:
\vspace{2mm}
$$X_u(u,v) = (1,0,-2u) \hspace{3ex} X_v(u,v) = (0,1,2v)$$
\newpage
\noindent Calculemos el producto vectorial de las derivadas según el siguiente falso determinante:
\vspace{1mm}
\begin{align*}
    X_u(u,v) \times X_v(u,v) &= 
    \begin{vmatrix}
    e_1 & e_2 & e_3 \\
    1 & 0 & -2u \\
    0 & 1 & 2v
    \end{vmatrix}
    = (2u, -2v, 1) \\[3ex]
    ||X_u(u,v) \times X_v(u,v)|| &= \sqrt{(2u)^2 + (-2v)^2 + 1^2} = \sqrt{4u^2+4v^2+1}
\end{align*} \\
\noindent Normalizando para que sea un vector unitario:
$$N(u,v) = \frac{X_u(u,v) \times X_v(u,v)}{||X_u(u,v) \times X_v(u,v)||} 
= \frac{1}{\sqrt{4u^2+4v^2+1}}\Big(2u,-2v,1\Big)$$

\vspace{3mm}
La imagen esférica del paraboloide se corresponde con la imagen del conjunto abierto $\mathcal{U}$ que
lo parametriza por medio de su vector normal. En este caso particular, $\mathcal{U} = \mathbb{R}^2$, luego
los parámetros $u,v$ pueden tomar cualquier valor real. De este modo, su imagen esférica viene dada por el siguiente conjunto:
\vspace{1mm}
$$ \left\{\left(\frac{2u}{\sqrt{4u^2+4v^2+1}}, \frac{-2v}{\sqrt{4u^2+4v^2+1}},
\frac{1}{\sqrt{4u^2+4v^2+1}}\right) \in \mathbb{R}^3 : u,v \in \mathbb{R}\right\}$$
\noindent
Veamos que este conjunto se corresponde con el hemisferio norte de la esfera unidad en $\mathbb{R}^3$:
$$ \left\{(x,y,z) \in \mathbb{R}^3 : x^2+y^2+z^2 = 1,  z > 0\right\}$$
Es evidente que la imagen esférica de la superficie está contenido en el hemisferio norte de la esfera
unidad, pues todos sus puntos son unitarios y tienen tercera componente positiva. Queda ver el otro contenido.
Para ello, basta ver que todo $(x,y,z)$ del hemisferio norte tiene un $(u,v)$ que lo genere.
Igualando las ecuaciones, para $D = 4u^2 + 4v^2 + 1$ se tiene:
\begin{align*}
x = \frac{2u}{\sqrt{D}} &&
y = \frac{-2v}{\sqrt{D}} &&
z = \frac{1}{\sqrt{D}}
\end{align*}
\\[2ex]
Despejando, obtenemos:
\begin{align*}
    x = \frac{2u}{\sqrt{D}} = 2uz \Rightarrow u =  \frac{x}{2z} &&
    y = \frac{-2v}{\sqrt{D}} = -2vz \Rightarrow v =  \frac{-y}{2z} &&
\end{align*}
\\[2ex]
\noindent En efecto, los dos conjuntos coinciden.

\newpage
Atendiendo ahora a la siguiente pregunta del primer apartado, consideramos el punto $P = (0,0,0)$
En efecto, se trata de un punto de la superficie, pues cumple su ecuación: $0 = 0^2 - 0^2$.
En particular, $P = X(0,0)$. Evaluando sus derivadas parciales:
\begin{align*}
    X_u(0,0) = (1,0,0) = v_1 &&
    X_v(0,0) = (0,1,0) = v_2
\end{align*}
\noindent Los $v_i$ pertenecen a $T_PS$ y son unitarios, luego tiene sentido calcular $\kappa_n(v_i, p)$.
En primer lugar, calculemos la curvatura normal respecto de $v_1$. Para ello, utilizaremos una curva $\alpha$
que cumpla $\alpha(0) = P, \hspace{1.5mm} \alpha'(0) = v_1$. Tomaremos la siguiente curva:
\begin{align*}
    \alpha: (-\varepsilon, \varepsilon) \longrightarrow \mathbb{R}^3 &&
    t \longmapsto X(0 + t, 0) = (t, 0, -t^2)
\end{align*}
\noindent De este modo, $\alpha(0) = P = (0,0,0)$, $\alpha'(0) = X_u(0,0) = (1,0,0) = v_1$.
Además, como $\mathcal{U} = \mathbb{R}^2$, $\alpha$ está bien definida para cualquier $\varepsilon > 0$.
Veamos qué valor tiene la curvatura normal: \vspace{2mm}
\begin{align*}
    &\alpha''(t) =  (0,0,-2) \\
    &N(0,0) = \frac{1}{\sqrt{4\cdot0^2 + 4\cdot 0^2 + 1}} \Big(2\cdot 0, -2 \cdot 0, 1\Big)
     = (0,0,1) \\[2ex]
    & \kappa_n(v_1, P) = \alpha''(0) \boldsymbol{\cdot} N(P) = -2&
\end{align*}

\vspace{2mm}
Realicemos el mismo procedimiento para $v_2$ ahora. Para ello, definiremos la siguiente curva para
un $\varepsilon > 0$ arbitrario.
\begin{align*}
    \beta: (-\varepsilon, \varepsilon) \longrightarrow \mathbb{R}^3 &&
    t \longmapsto X(0, 0 + t) = (0, t, t^2)
\end{align*}
\noindent De manera análoga al caso anterior, se cumple
$\beta(0) = X(0,0) = (0,0,0) = P$,  $\beta'(0) = X_v(0,0) = (0,1,0) = v_2$.
Veamos qué valor tiene la curvatura normal asociada a $v_2$:
\begin{align*}
    \beta''(0) = (0,0,2) &&
    \kappa_n(v_2, P) = \beta''(0) \cdot N(P) = 2
\end{align*}
Por todo lo anterior, podemos afirmar que el punto $P$ no es umbilical, pues
las curvaturas normales respecto de sus direcciones principales no coinciden.

\vspace{4mm}
Finalmente, hallemos la curvatura normal en $P$ respecto de la dirección $v_3$.
Tiene sentido, pues $||v_3|| = 1$,  y  $v_3 \in T_PS$ al ser combinación lineal
de $v_1$ y $v_2$. En este caso, utilizaremos la segunda forma fundamental.
Calculando las segundas derivadas:
\begin{align*}
    X_{uu}(u,v) = (0,0,-2)&&
    X_{uv}(u,v) = (0,0,0)&&
    X_{vv}(u,v) = (0,0,2)
\end{align*}
\newpage
Los coeficientes de la expresión matricial de la segunda forma fundamental
vienen dados por el producto escalar de las segundas derivadas:
\vspace{-5mm}
\begin{multicols}{2}
    \begin{flalign*}
        &e = X_{uu}(0,0) \boldsymbol{\cdot} N(0,0)
        = (0,0,-2) \boldsymbol{\cdot} (0,0,1) = -2 \\
        &f = X_{uv}(0,0) \boldsymbol{\cdot} N(0,0)
        = (0,0,0) \boldsymbol{\cdot} (0,0,1) = 0 \\
        &g = X_{vv}(0,0) \boldsymbol{\cdot} N(0,0)
        = (0,0,2) \boldsymbol{\cdot} (0,0,1) = 2 \\     
    \end{flalign*}
    \columnbreak
    \\\\
    \begin{flalign*}
        \hspace{25mm}
        \mathrm{II}_P =
        \begin{pmatrix}
            e & f \\
            f & g
        \end{pmatrix}
        =
        \begin{pmatrix}
            -2 & 0 \\
            0 & 2
        \end{pmatrix}
    \end{flalign*}
\end{multicols}
\vspace{-5mm}
Esta matriz representa $\mathrm{II}_P$ en $T_PS$ según la base
 $\{X_u(0,0), X_v(0,0)\}$. Por tanto, es claro que $v_1, v_2$
son direcciones principales al ser vectores propios. 
Expresemos $v_3$ en función de esta base:
$$v_3 =  \left(\frac{1}{\sqrt{2}}, \frac{1}{\sqrt{2}}, 0 \right)
 = \frac{1}{\sqrt{2}}\big(1,0,0\big) + \frac{1}{\sqrt{2}}\big(0,1,0\big)
 = \frac{1}{\sqrt{2}} X_u(0,0) + \frac{1}{\sqrt{2}} X_v(0,0)$$
\vspace{1mm}
Por tanto, evaluando la segunda forma fundamental en este vector:
\begin{flalign*}
    \kappa_n(v_3, P) =& \mathrm{II}_P(v_3) =
    \begin{pmatrix}
        \frac{1}{\sqrt{2}} & \frac{1}{\sqrt{2}}
    \end{pmatrix}
    \begin{pmatrix}
        -2 & 0 \\
        0 & 2
    \end{pmatrix}
    \begin{pmatrix}
        \frac{1}{\sqrt{2}} \\
        \frac{1}{\sqrt{2}} 
    \end{pmatrix}
    =
    \begin{pmatrix}
        \frac{-2}{\sqrt{2}} &
        \frac{2}{\sqrt{2}}
    \end{pmatrix}
    \begin{pmatrix}
        \frac{1}{\sqrt{2}} \\
        \frac{1}{\sqrt{2}}
    \end{pmatrix}
    = \\
    &= \frac{-2}{\sqrt{2}} \cdot\frac{1}{\sqrt{2}}
     + \frac{2}{\sqrt{2}} \cdot \frac{1}{\sqrt{2}}
    = -1 + 1 = 0
\end{flalign*}

\newpage
\subsection{Segundo apartado}
\hspace{1mm}
Al igual que en el primer apartado, la superficie es la gráfica de una función.
Por tanto, podemos parametrizarla como sigue:
\begin{align*}
    X : \mathcal{U} = \mathbb{R}^2 
     \longrightarrow \mathbb{R}^3 &&
    (u,v) \longmapsto X(u,v) = (u,v, u^2 + v^3)
\end{align*}
Estudiemos qué aspecto tiene la matriz de la segunda forma fundamental
para cada punto de la superficie. En primer lugar, hallemos las derivadas
parciales:
\begin{align*}
    X_u(u,v) = (1,0,2u) &&
    X_v(u,,v) = (0,1,3v^2)
\end{align*}
Hallemos el vector normal unitario:
\begin{align*}
    &X_u(u,v) \times X_v(u,v) = 
    \begin{vmatrix}
    e_1 & e_2 & e_3 \\
    1 & 0 & 2u \\
    0 & 1 & 3v^2
    \end{vmatrix}
    = (-2u, -3v^2, 1) \\[3ex]
    &||X_u(u,v) \times X_v(u,v)||
     = \sqrt{(-2u)^2 + (-3v^2)^2 + 1^2}
     = \sqrt{4u^2 + 9v^4 + 1}&
\end{align*}\\[0.5mm]
$$N(u,v) = \left(
    \frac{-2u}{\sqrt{4u^2 + 9v^4 + 1}},
    \frac{-3v^2}{\sqrt{4u^2 + 9v^4 + 1}},
    \frac{1}{\sqrt{4u^2 + 9v^4 + 1}},
\right)$$
\\[2ex]

Vamos a expresar la segunda forma fundamental en función del
producto escalar entre el vector normal y las segundas derivadas
de la parametrización. Veamos cuáles son las derivadas de orden
dos de $X$ en un punto arbitrario:
\begin{align*}
    X_{uu}(u,v) = (0,0,2) &&
    X_{uv}(u,v) = (0,0,0) &&
    X_{vv}(u,v) = (0,0,6v)
\end{align*}

\vspace{2mm}
Para clasificar los puntos en elípticos, hiperbólicos, parabólicos
o planos, solo es necesario conocer el signo de sus curvaturas normales
principales. Por tanto, podemos multiplicar las segundas derivadas por
$X_u \times X_v$ en lugar de por el vector normal, pues no afectará al signo
del resultado. Por tanto, basta estudiar la matriz dada por:
\vspace{2mm}
\begin{flalign*}
    &\hat{e} = X_{uu}(u,v) \boldsymbol{\cdot} \big(X_u(u,v) \times X_v(u,v)\big)
    = (0,0,2) \boldsymbol{\cdot} (-2u,-3v^2,1) = 2 \\
    &\hat{f} = X_{uv}(u,v) \boldsymbol{\cdot} \big(X_u(u,v) \times X_v(u,v)\big)
    = (0,0,0) \boldsymbol{\cdot} (-2u,-3v^2,1) = 0 \\
    &\hat{g} = X_{vv}(u,v) \boldsymbol{\cdot} \big(X_u(u,v) \times X_v(u,v)\big)
    = (0,0,6v) \boldsymbol{\cdot} (-2u,-3v^2,1) = 6v \\     
\end{flalign*}
\vspace{-10mm}
\begin{flalign*}
    \hat{\mathrm{II}}_P =
    \begin{pmatrix}
        \hat{e} & \hat{f} \\
        \hat{f} & \hat{g}
    \end{pmatrix}
    =
    \begin{pmatrix}
        2 & 0 \\
        0 & 6v
    \end{pmatrix}
\end{flalign*}
\newpage
Esta matriz ya es diagonal, y sus valores propios son
2 y $6v$ (las curvaturas normales principales).
Para estudiar el tipo de punto basta estudiar el signo del
producto de las mismas.
Nótese que al menos uno de los dos valores es
siempre distinto de cero, luego la superficie no tiene puntos planos.
El carácter de $X(u,p)$ viene determinado por $6v$. Comparando el signo de $6v$
con el signo de 2 se tiene:

\begin{itemize}
    \item $P = X(u,v)$ es elíptico $\iff 6v > 0 \iff v > 0$
    \item $P = X(u,v)$ es hiperbólico $\iff 6v < 0 \iff v < 0$
    \item $P = X(u,v)$ es parabólico $\iff 6v = 0 \iff v = 0$
\end{itemize}
Sustituyendo en la parametrización, obtenemos:
\begin{itemize}
    \item Puntos elípticos: $\{(u,v,u^2+v^3) \in \mathbb{R}^3 : u \in \mathbb{R}, v \in (0, +\infty)\}$
    \item Puntos hiperbólicos: $\{(u,v,u^2+v^3) \in \mathbb{R}^3 : u \in \mathbb{R}, v \in (-\infty, 0)\}$
    \item Puntos parabólicos: $\{(u,0,u^2) \in \mathbb{R}^3 : u \in \mathbb{R}\}$
\end{itemize}
\end{document}