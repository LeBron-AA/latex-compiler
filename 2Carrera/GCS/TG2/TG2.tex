\documentclass{article}
\usepackage{enumitem}
\usepackage{amssymb}
\usepackage{amsmath}
\usepackage{enumitem}
\usepackage{tikz}

\title{TG II}   
\author{
    Javier Ortín Rodenas \\
    Bruno Martín Rivera \\
    Jorge Gota Ortín \\
    Alejandra Sánchez Mayo
}
\date{}
\begin{document}
\maketitle
\noindent
\section{Enunciado:}
Un toro es la superficie de revolución generada por una circunferencia al girar alrededor de un eje situado en su mismo plano y que no la corta.\\
Sean $a, b \in \mathbb{R}$ con $0<a<b$.\\
Consideremos el toro cuya curva generatriz es la circunferencia de centro
en el punto $(a, 0,0)$ y radio $b$ situada en el plano $y=0$.
Dicha circunferencia gira alrededor del eje $z$.\\[2ex]

\begin{enumerate}[label=(\roman*)]
    \item Usando como parámetros $u \in(0,2 \pi)$, ángulo de giro del punto inicial respecto del eje $x, y$ $v \in(0,2 \pi)$, ángulo de giro de dicho punto respecto del eje z, obtener una parametrización del toro, como superficie de revolución, de la forma
    $$
    \begin{aligned}
        X: U=(0,2 \pi) \times(0,2 \pi) \subseteq \mathbb{R}^{2} & \longrightarrow \mathbb{R}^{3} \\
        (u, v) & \longmapsto X(u, v)=(x(u, v), y(u, v), z(u, v))
    \end{aligned}
    $$
    
    \item Con ayuda de dicha parametrización, obtener una ecuación cartesiana de la superficie generada, es decir, de modo que el toro se pueda expresar en la forma
    
    $$
    \mathbb{T}=\left\{(x, y, z) \in \mathbb{R}^{3}: F(x, y, z)=0\right\}
    $$
    
    \noindent donde $F$ es una función que hay que determinar.\\

\item Determinar si $X$ es una carta del toro de clase $C^{k}(k \geq 1)$.\\
\item Determinar el plano tangente al toro $y$ el vector unitario normal en cada punto regular $p=X(q), \hspace{1mm} q \in U$ del soporte $X(U)$

\end{enumerate}

\section{Solución}
\subsection{Primer apartado}
\hspace{1mm} Para parametrizar el toro, podemos centrarnos primero en parametrizar la curva
generatriz para posteriormente modelar el proceso de revolución. En este caso,
la curva generatriz es una circunferencia de centro $(a,0,0)$ y radio $b$
contenida en el plano $y = 0$.

\begin{center}
    \begin{tikzpicture}[scale=1.5]
        % Axes
        \draw[->] (0,0) -- (7,0) node[right] {$x$}; % X-axis
        \draw[->] (0,0) -- (0,3) node[left] {$z$}; % Z-axis
        \draw[->] (0,0) -- (0,-3)
        
        % Circle with center at (a,0)
        \draw[thick] (3,0) circle(1.5); % Circumference
        \filldraw (3,0) circle(2pt);  % Center point
        
        % Labeling the center
        \node[below] at (3,0) {$(a,0,0)$};
    \end{tikzpicture}
    
\end{center}

\end{document}