\documentclass{article}
\usepackage{enumitem}
\usepackage{amssymb}
\usepackage{amsmath}
\usepackage{enumitem}
\usepackage{hyperref}
\usepackage{tikz}
\usetikzlibrary{3d, perspective}

%Cambiar "Figure" a "Figura"
\renewcommand{\figurename}{Figura}
\renewcommand{\figureautorefname}{Figura}

\title{TG II}   
\author{
    Javier Ortín Rodenas \\
    Bruno Martín Rivera \\
    Jorge Gota Ortín \\
    Alejandra Sánchez Mayo
}
\date{}
\begin{document}
\maketitle
\noindent
\section{Enunciado:}
Un toro es la superficie de revolución generada por una circunferencia al girar alrededor de un eje situado en su mismo plano y que no la corta.\\
Sean $a, b \in \mathbb{R}$ con $0<a<b$.\\
Consideremos el toro cuya curva generatriz es la circunferencia de centro
en el punto $(a, 0,0)$ y radio $b$ situada en el plano $y=0$.
Dicha circunferencia gira alrededor del eje $z$.\\[2ex]

\begin{enumerate}[label=(\roman*)]
    \item Usando como parámetros $u \in(0,2 \pi)$, ángulo de giro del punto inicial respecto del eje $x, y$ $v \in(0,2 \pi)$, ángulo de giro de dicho punto respecto del eje z, obtener una parametrización del toro, como superficie de revolución, de la forma
    $$
    \begin{aligned}
        X: U=(0,2 \pi) \times(0,2 \pi) \subseteq \mathbb{R}^{2} & \longrightarrow \mathbb{R}^{3} \\
        (u, v) & \longmapsto X(u, v)=(x(u, v), y(u, v), z(u, v))
    \end{aligned}
    $$
    
    \item Con ayuda de dicha parametrización, obtener una ecuación cartesiana de la superficie generada, es decir, de modo que el toro se pueda expresar en la forma
    
    $$
    \mathbb{T}=\left\{(x, y, z) \in \mathbb{R}^{3}: F(x, y, z)=0\right\}
    $$
    
    \noindent donde $F$ es una función que hay que determinar.\\

\item Determinar si $X$ es una carta del toro de clase $C^{k}(k \geq 1)$.\\
\item Determinar el plano tangente al toro $y$ el vector unitario normal en cada punto regular $p=X(q), \hspace{1mm} q \in U$ del soporte $X(U)$

\end{enumerate}

\section{Solución}
\subsection{Primer apartado}
\hspace{1mm} Para parametrizar el toro, podemos centrarnos primero en parametrizar la curva
generatriz para posteriormente modelar el proceso de revolución. En este caso,
la curva generatriz es una circunferencia de centro $(a,0,0)$ y radio $b$
contenida en el plano $y = 0$.

\begin{figure}[htbp]
\begin{center}
    \begin{tikzpicture}[scale=1.2]
        % Axes
        \draw[->] (0,0) -- (7,0) node[right] {$x$}; % X-axis
        \draw[->] (0,0) -- (0,3) node[left] {$z$}; % Z-axis
        \draw[->] (0,0) -- (0,-3);
        
        % Circle with center at (a,0)
        \draw[thick] (4,0) circle(1); % Circumference
        \node[below] at (4,0) {\small{$(a,0,0)$}}; %Labeling the center
        
        %Drawing a point of the circumference
        \draw[red, thick] (4,0) -- ++(45:1); %Radius
        \filldraw (4,0) circle(2pt);  % Center point
        \node[left, red] at (4.3, 0.5) {$b$}; %b text
        \filldraw[red] (4.7071,0.7071) circle(1.5pt); %Dotting the point
        \node[anchor=south west] at (4.7071,0.7071) {\small{$\big(a + b \cos(u),0,b \sin(u)\big)$}}; %Labeling the point

        %Drawing the angle arch
        \draw[thick, blue] (4.4,0) arc[start angle=0, end angle=45, radius=0.4];
        \node[blue, anchor = south west] at (4.4,0) {$u$};

        %Radius
        \draw[green, <->] (0,0.7071) -- (4.7071,0.7071);
        \node[green, above] at (2.35355, 0.7071) {$a +b\cos(u)$};

        %Height
        \draw[cyan,thick, <->] (0.7,0) -- (0.7,0.7071);
        \node[cyan, right] at (0.7, 0.35) {$b \sin(u)$};

    \end{tikzpicture}
\end{center}
\caption{Representación de la circunferencia generatriz en el plano $y = 0$}
\label{fig1}
\end{figure}

Hemos utilizado coordenadas polares para parametrizar la circunferencia generatriz.
En este caso, el radio de la circunferencia es $b$, luego es el coeficiente que
multiplica al seno y al coseno del ángulo que forma cada punto con el eje $x$. Como la circunferencia
se encuentra en el plano $y = 0$, la segunda componente tiene valor constante $0$. Finalmente,
sumamos las coordenadas del centro $(a, 0, 0)$ para obtener la parametrización final de la 
curva generatriz. \vspace{1mm}

Queda parametrizar ahora el proceso de revolución. Como el eje de giro es el eje $z$, el radio
de revolución de un punto de la circunferencia será su distancia hasta él; es decir, su componente $x$.
El resultado final del proceso de revolución tendrá  la misma coordenada $z$ que el punto que lo generó.
No obstante, las componentes $x$ e $y$ se verán modificadas según el ángulo de giro.
\vspace{1mm}

Sea $(x_0, 0, z_0)$ un punto cualquiera de la circunferencia generatriz, los puntos del toro resultantes
de su revolución son los de la circunferencia de centro $(0,0,z_0)$ y radio $x_0$ en el plano $z = z_0$.
Razonando así para cada punto de la generatriz llegamos a la parametrización final del toro.

Veamos un ejemplo gráfico del proceso de revolución para un punto dado.
Fijado el punto de la \autoref{fig1}, veamos qué sucede al girarlo sobre el eje $z$.
La siguiente representación coincide con el escenario supuesto, pues $0 < b < a$ por hipótesis,
lo que garantiza que la generatriz no corte al eje de giro.

\newcommand{\mycircle}[3] % Círculo tangente al vector x, y, z
{%
  \pgfmathsetmacro\vtheta{atan2(#2,#1)}                     % spherical coordinate theta
  \pgfmathsetmacro\vphi  {acos(#3/sqrt(#1*#1+#2*#2+#3*#3))} % spherical coordinate phi
  \begin{scope}[rotate around z=\vtheta,rotate around y=\vphi,canvas is xy plane at z=0]
    \draw (-0.5,-0.5) rectangle (0.5,0.5);
    \draw[fill=gray,fill opacity=0.2] (0,0) circle (0.5);   % the plane, probably not needed
  \end{scope}
}


\begin{figure}[h!]
    \begin{center}
    \begin{tikzpicture}[scale=2]
        % Axes
        \draw[->] (0,0,0) -- (4,0,0) node[right] {$x$};  % x-axis
        \draw[->] (0,0,0) -- (0,0,4) node[below right] {$y$};  % y-axis
        \draw[->] (0,0,0) -- (0,2.5,0) node[above] {$z$};  % z-axis
        \draw[->, dotted] (0,0,0) -- (0,-2.5,0);

        %Generatriz 1
        \begin{scope}[canvas is xy plane at z=0]
            \draw (2,0) circle (0.75cm);
            \filldraw (2,0) circle (1pt);
          \end{scope}

        %Generatriz 2
        \begin{scope}[canvas is zy plane at x=0]
            \draw (2,0) circle (0.75cm);
            \filldraw (2,0) circle (1pt);
          \end{scope}

        %Revolución
        \begin{scope}[canvas is xz plane at y=0]
            \draw[thick, blue] (2,0) arc[start angle=0, end angle=90, radius=2];
        \end{scope}

        \begin{scope}[canvas is xz plane at y=0.75]
        \draw[thick, red] (2,0) arc[start angle=0, end angle=90, radius=2];
        \end{scope}

        \begin{scope}[canvas is xz plane at y=-0.75]
        \draw[thick, red] (2,0) arc[start angle=0, end angle=90, radius=2];
        \end{scope}

        \begin{scope}[canvas is xz plane at y=0]
        \draw[green] (2.75,0) arc[start angle=0, end angle=90, radius=2.75];
        \end{scope}

        \begin{scope}[canvas is xz plane at y=0]
        \draw[dotted,green] (1.25,0) arc[start angle=0, end angle=90, radius=1.25];
        \end{scope}


        %Radios
        \draw[thick, dotted] (0,0.75,0) -- (2,0.75,0);
        \draw[thick, dotted] (0,0.75,0) -- (0,0.75,2);
        \node[above] at (0.9,0.75,0) {$a + b\cos (u)$};

        %Parámetro ángulo
        \begin{scope}[canvas is xz plane at y=0.75]
            \draw[dotted, thick] (0.5,0) arc[start angle=0, end angle=90, radius=0.5];
            \node at (0.5, 0.5) {$v$};
        \end{scope}
    \end{tikzpicture}
\end{center}
    \caption{Giro sobre el eje $z$ de la generatriz en el primer cuadrante}
\end{figure}

\vspace{4ex}

\noindent Por todo lo anterior, la parametrización del toro es la siguiente:
$$
    \begin{aligned}
        X: U&= (0,2 \pi) \times(0,2 \pi) \subseteq \mathbb{R}^{2}  \longrightarrow \mathbb{R}^{3} \\
        (u,& v)  \longmapsto  X(u, v)= \Big(\big(a + b \cos(u)\big)\cos(v),
                \hspace{1mm} \big(a + b \cos(u)\big)\sin(v),
                \hspace{1mm}b \sin(u)\Big)
    \end{aligned}
$$

\newpage

\subsection{Segundo apartado}
\hspace{1mm} Veamos cómo definir el toro parametrizado en el apartado anterior con una ecuación cartesiana.
Es decir, buscamos que no aparezcan los parámetros $u$ y $v$, sino que solo
intervengan las variables $x,y,z$.
Planteamos el siquiente sistema para después obtener una única ecuación:

\[
\hspace{-7cm} % Adjust this value as needed
\left\{
\begin{aligned}
x &= \Big(a + b \cos(u)\Big) \cos(v) \\
y &= \Big(a + b \cos(u)\Big) \sin(v) \\
z &= b \sin(u)
\end{aligned}
\right.
\]
\\
\noindent Relacionamos las ecuaciones del sistema mediante identidades trigonométricas:
\begin{flalign*}
    x^2 + y^2 &= \Big(a + b \cos(u)\Big)^2 \cos^2(v)
    + \Big(a + b \cos(u)\Big)^2 \sin^2(v) =\\
    = & \Big(a + b \cos(u)\Big)^2 = a^2 + 2ab \cos(u) + b^2 \cos^2(u)
\end{flalign*}

$$ x^2 + y^2 + z^2 = a^2 + 2ab \cos(u) + b^2 \cos^2(u) + b^2 \sin^2(u) = a^2 + 2ab \cos(u) + b^2$$
\\
\noindent Notamos que el coseno sigue presente, de manera similar a cómo ocurría en la expresión de $x^2 + y^2$.
Buscamos definir el toro a partir de esta correlación.

$$(x^2+y^2+z^2+a^2-b^2)^2 = \left(2a^2 + 2ab \cos(u) \right)^2 = 4a^4 + 4a^3b \cos(u) + 4a^2b^2 \cos^2(u)$$
\\
\noindent Finalmente, sacando factor común obtenemos la ecuación cartesiana del toro:

$$ (x^2+y^2+z^2+a^2-b^2)^2 = 4a^2(a^2 + 2ab \cos(u) + b^2\cos^2(u)) = 4a(x^2 + y^2)$$
\\ \noindent
Por todo lo anterior, si definimos la siguiente función, podremos escribir el toro como su conjunto de ceros.

$$
    \begin{aligned}
        F: \mathbb{R}^3& \longrightarrow \mathbb{R} \\
        (x,& y, z)  \longmapsto  F(x, y, z)= (x^2+y^2+z^2+a^2-b^2)^2 - 4a(x^2 + y^2)
    \end{aligned}
$$
$$\mathbb{T} = \left\{(x,y,z) \in \mathbb{R}^3 :F(x,y,z) = 0\right\}$$
    

\newpage

\subsection{Tercer apartado}
\hspace{1mm} En vista de la proposición 1.18, para comprobar que $X$ es una carta de clase $\mathcal{C}^k (k\geq1)$,
podemos comprobar que se verifican simultáneamente las siguientes condiciones:
\begin{itemize}
    \item $X$ es de clase $\mathcal{C}^k$
    \item $X$ es inyectiva
    \item $d(X)(u,v) $ tiene rango 2 $ \hspace{1.5mm} \forall (u,v) \in U$
\end{itemize}

\vspace{2mm}
En primer lugar, tenemos que $X$ es de clase $\mathcal{C}^\infty$ al
ser composición y suma de funciones  de clase $\mathcal{C}^\infty$.

\vspace{2mm}
Veamos que $X$ es una carta al comprobar que su matriz jacobiana (y por tanto su diferencial) siempre
tiene rango máximo. Para ello, mostraremos que no puede haber un contraejemplo donde la diferencial
de $X$ no tiene rango 2 por reducción al absurdo. Estudiemos la matriz jacobiana:
\vspace{1mm}
\[
J_X(u,v) =
\begin{pmatrix}
-b \sin(u) \cos(v) & -\big(a + b\cos(u) \big) \sin(v)\\[2ex]
-b \sin(u) \sin(v) & \big(a + b \cos(u) \big) \cos(v)\\[2ex]
b \cos(u) & 0
\end{pmatrix}
\]

\vspace{2mm{}}
De haber un caso $(u, v) \in U$ donde la diferencial de $X$ no tuviese rango
máximo, tendría que cumplirse necesariamente que las columnas de la matriz
$J_X(u,v)$ sean proporcionales. En particular, igualando la tercera componente:
$$X_u(u,v) = \lambda X_v(u,v) \Rightarrow b \cos(u) = 0 \Rightarrow \cos(u) = 0 $$
\noindent La última igualdad se deduce de que $b > 0$ por hipótesis. \\

Además, el rango de $J_X(u,v)$ es el el mayor rango entre sus submatrices cuadradas
de orden dos. En caso de no ser máximo, el determinante de todas ellas ha de ser
necesariamente nulo. Estudiemos el caso de su submatriz superior,
a la que denotaremos $A$. Como sabemos que $\cos(u) = 0$; para este caso,
$A$ viene dada por: \\
\[
A =
\begin{pmatrix}
-b \sin(u) \cos(v) & -a \sin(v)\\[2ex]
-b \sin(u) \sin(v) & a \cos(v)
\end{pmatrix}
\]
\\
$$
\det{A} = -ab \sin(u) \cos^2(v) - a b \sin(u)\sin^2(v) = 
 -a b \sin(u)
$$

\vspace{2mm}
En estas condiciones, el determinante de $A$ se anula si y solo
si $\sin(u) 0$, pero como ha de cumplirse simultáneamente
$\cos(u) = 0$ llegamos a una contradicción. Hemos demostrado que
el rango de $J_X(u, v)$ es siempre máximo para todo $(u,v) \in U$.

Queda ver ahora que $X$ es inyectiva. Para ello, supongamos que
existen $(u_1, v_1), (u_2, v_2) \in U$ tales que $X(u_1, v_1) = X(u_2, v_2)$
para ver que se cumple necesariamente $(u_1, v_1) = (u_2, v_2)$. Por
definición de $X$, obtenemos el siguiente sistema de ecuaciones:
\\
\[
\hspace{-2cm} % Adjust this value as needed
\left\{
\begin{aligned}
\big(a + b\cos&(u_1)\big)\cos(v_1) = \big(a + b\cos(u_2)\big)\cos(v_2) \\
\big(a + b\cos&(u_1)\big)\sin(v_1) = \Big(a + b \cos(u_2)\Big) \sin(v_2) \\
b \sin(u_1) &= b \sin(u_2)
\end{aligned}
\right.
\]
\\
\noindent De la última ecuación se deduce lo siguiente:
$$ \sin(u_1) = \sin(u_2) \Rightarrow \cos^2(u_1) = \sqrt{1 - \sin^2(u_1)}
= \sqrt{1 - \sin^2(u_2)} = \cos^2(u_2)$$
\noindent Si elevamos al cuadrado las dos primeras ecuaciones y las sumamos:
\begin{flalign*}
    \big(a +  b\cos(u_1)\big)^2 &=  \big(a + b\cos(u_2)\big)^2 
    \Rightarrow a^2 + 2ab \cos(u_1) + \cos^2(u_1) = a^2
     + 2ab \cos(u_2) + \cos^2(u_2) \\
    \Rightarrow & 2ab\cos(u_1) = 2ab\cos(u_2)
     \Rightarrow \cos(u_1)= \cos(u_2)
\end{flalign*}
Como $X$ está definida sobre $U = (0, 2\pi) \times (0,2\pi)$, que se cumplan
simultáneamente $\cos(u_1) = \cos(u_2)$ y $\sin(u_1) = \sin(u_2)$ implica $u_1 = u_2$.
Así, al dividir en las dos primerdas ecuaciones entre 
$\big(a + b\cos(u_i)\big)$ obtenemos el sistema:
\[
\hspace{-8cm} % Adjust this value as needed
\left\{
\begin{aligned}
\cos(v_1) = \cos(v_2) \\
\sin(v_1) = \sin(v_2) \\
\end{aligned}
\right.
\]
Razonando análogamente al caso anterior, como $v_1, v_2 \in (0, 2\pi)$,
tenemos que $v_1 = v_2$. Hemos demostrado
$(u_1, v_1) = (u_2, v_2)$, luego $X$ es inyectiva. Por todo lo anterior,
concluimos que $X$ sí es una carta.

\newpage
\subsection{Cuarto apartado}
El espacio tangente al toro en un determinado punto $(u,v) \in U$ es un
espacio vectorial que tiene a las derivadas parciales de $X$ como base.
Por tanto, el espacio tangente al toro $\mathbb{T}$ en un punto $X(u,v)$
viene dado por la clausura:
$$T_{X(u,v)}\mathbb{T} = \mathbb{R} \langle X_u(u,v), X_v(u,v)\rangle$$

Sabemos además por el apartado anterior cuál es la expresión de $J_X(u,v)$,
que tiene a las derivadas parciales de $X$ en $(u,v)$ como vectores columna.
Por tanto, el espacio tangente a $\mathbb{T}$ para cada $(u,v)$ es el espacio
vectorial generado como clausura de los siguientes vectores:
$$X_u(u,v)= \Big(-b \sin(u) \cos(v), -b \sin(u) \sin(v), b \cos(u)\Big)$$
$$X_v(u,v) = \Big(\big(-a - b\cos(u)\big) \sin(v), \big(a + b \cos(u) \big) \cos(v), 0\Big)$$
\end{document}