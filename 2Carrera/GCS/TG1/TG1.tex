\documentclass{article}
\usepackage{enumitem}
\begin{document}

\title{TG I}
\date{21/02/2025}
\maketitle
\begin{center}
    
    \author{
    GOTA ORTÍN, JORGE \\
    \and MARTÍN RIVERA, BRUNO \\
    \and ORTÍN RODENAS, JAVIER \\
    \and SÁNCHEZ MAYO, ALEJANDRA
    }   
\end{center}
\vspace{5mm}

\noindent
\textbf{Problema:} \\[1mm]

Sea $\alpha : I \to R^3$ una p.p.a birregular de clase
$\mathcal{C}^k$ $(k \geq 3)$ tal que $\tau(t) \neq 0$, \\
para todo $t \in I$, entonces son equivalentes:

\begin{enumerate}[label=(\roman*)]
    \item La curva parametrizada por $\alpha$ está contenida en una esfera.
    \item Existe una constante $r > 0$ tal que $r^2 = R(t)^2 + \left(\frac{R^\prime(t)}{\tau(t)}\right)^2$ \hspace{2mm} $\forall t \in I$
    \item $\frac{\tau(t)}{\kappa(t)} = \left( \frac{\kappa^\prime(t)}{\tau(t) \kappa^2(t)} \right)^\prime$
\end{enumerate}
\noindent
En tal caso, la curva tiene como centro
$c = \alpha(t) + R(t)N(t) + \frac{R'(t)}{\tau(t)}B(t)$
para todo $t \in I$, y como radio
$r = \sqrt{R(t)^2 + \left( \frac{R'(t)}{\tau(t)}\right)^2} \hspace{2mm} > 0$.
\\\\[0,6cm]
\textbf{Demostración:} \\[0,2cm]
\indent En primer lugar...
\end{document}