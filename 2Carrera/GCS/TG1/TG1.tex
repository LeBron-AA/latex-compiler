\documentclass{article}
\usepackage{enumitem}
\usepackage{amssymb}
\usepackage{amsmath}

\setlength{\jot}{3mm}

\begin{document}

\title{TG I}
\date{21/02/2025}
\maketitle
\begin{center}
    
    \author{
    Javier Ortín Rodenas \\
    \and Manuel Mateos Suárez \\
    \and Jimena Rubín Sánchez \\
    \and Alejandra García Fernández
    }   
\end{center}
\vspace{5mm}
\noindent
\textbf{Problema:} \\[1mm]

Sea $\alpha : I \to \mathbb{R} ^3$ una p.p.a birregular de clase
$\mathcal{C}^k$ $(k \geq 3)$ tal que $\tau(t) \neq 0$, \\
para todo $t \in I$, entonces son equivalentes:

\begin{enumerate}[label=(\roman*)]
    \item La curva parametrizada por $\alpha$ está contenida en una esfera.
    \item Existe una constante $r > 0$ tal que $r^2 = R(t)^2 + \left(\frac{R^\prime(t)}{\tau(t)}\right)^2$ \hspace{2mm} $\forall t \in I$
    \item $\frac{\tau(t)}{\kappa(t)} = \left( \frac{\kappa^\prime(t)}{\tau(t) \kappa^2(t)} \right)^\prime$
\end{enumerate}
\noindent
En tal caso, la curva tiene como centro
$c = \alpha(t) + R(t)N(t) + \frac{R'(t)}{\tau(t)}B(t)$
para todo $t \in I$, y como radio
$r = \sqrt{R(t)^2 + \left( \frac{R'(t)}{\tau(t)}\right)^2} \hspace{2mm} > 0$.
\\\\[0,6cm]
\textbf{Demostración:} \\[0,2cm]
\underline{$(i) \Rightarrow (ii)$} \vspace{2mm}

Supongamos que $\alpha$ está contenida en una esfera.
De este modo, existen $r \in \mathbb{R}, r > 0$, $c \in \mathbb{R}^3$
constantes tales que para todo $t \in I$ se verifica:
$$(\alpha(t) - c) \cdot (\alpha(t) - c) = r^2$$
Por hipótesis, sabemos que $\alpha$ es birregular,
luego existe el Triedro de Frenet para todo $t \in I$.
Veamos cómo expresar el vector $(\alpha(t) - c)$ según esta base:

\newpage
En general, sea $\{v_1, \dots, v_n\}$ una base ortonormal
de $\mathbb{R}^n $. Sea $v \in \mathbb{R}^n$, podemos expresar
$v$ en función de los vectores de la base:
$v = \lambda_1 v_1 + \dots + \lambda_n v_n$
Aplicando la ortonormalidad, sea $i \in \{1,\dots n\}$:
$$
v \cdot v_i = (\lambda_1 v_1 + \dots + \lambda_n) \cdot v_i
= \left(\sum_{j=1}^{n}\lambda_j v_j\right) \cdot v_i
= \sum_{j=1}^{n}\lambda_j \delta_{ij} = \lambda_i
$$
\noindent
Por tanto, para expresar  el vector $(\alpha(t) - c)$ en función de
$\{T(t), N(t), B(t)\}$ basta ver el producto escalar entre
dicho vector y los componentes de la base.

\vspace{4mm} \noindent
Derivando la expresión
$r^2 = (\alpha(t) - c) \cdot (\alpha(t) - c)$
obtentemos:
\begin{equation*}
    0 = 2\alpha '(t) \cdot \big{(}\alpha(t) - c\big{)}
    \Rightarrow T(t) \cdot \big{(}\alpha(t) - c\big{)} = 0
\end{equation*}

\vspace{2mm} \noindent
Derivando de nuevo:
\begin{flalign*}
    0 &= T'(t) \cdot \big(\alpha(t) - c \big) + T(t) \cdot \alpha'(t)
    = K(t) \cdot \big(\alpha(t) - c \big) + ||T(t)||^2 \Rightarrow &&
    \\
    \Rightarrow & -1 = K(t) \cdot \big(\alpha(t) - c\big) = (\kappa(t) N(t)) \cdot \big(\alpha(t) - c \big) &&
    \\
    \Rightarrow & N(t) \cdot \big(\alpha(t) - c \big) = -R(t)
\end{flalign*}

\vspace{2mm} \noindent
Derivando una vez más, por las ecuaciones de Frenet-Serret:
\begin{flalign*}
    -R'&(t) = N'(t) \cdot \big(\alpha(t) - c \big) + N(t) \cdot \alpha'(t)
    = N'(t) \cdot \big(\alpha(t) - c \big) + N(t) \cdot T(t) =
    &&\\
    = & \Big{(} -\kappa(t)T(t) + \tau(t)B(t) \Big{)} \cdot \big(\alpha(t) - c\big)  + 0
    = -\kappa(t) T(t) \cdot \big(\alpha(t) - c \big)
    + \tau(t) B(t) \cdot \big{(}\alpha(t) - c \big{)} =
    && \\
    = &\hspace{1mm}  0 +  \tau(t)B(t)\big(\alpha(t) - c \big) = -R'(t)
    \Rightarrow B(t) \cdot \big(\alpha(t) - c \big) = \frac{-R'(t)}{\tau(t)}
\end{flalign*}
\\[3mm]

Por todo lo anterior, concluimos que la expresión del vector respecto de la base es:
$\big(\alpha(t) - c \big) = 0T(t) -R(t)N(t) - \frac{R'(t)}{\tau(t)}B(t)$.
Multiplicando el vector consigo mismo; en efecto,
$r^2 = R(t)^2 + \left(\frac{R'(t)}{\tau(t)}\right)^2 > 0$ pues $\alpha$ es birregular
por hipótesis. Finalmente, despejando concluimos:

\begin{flalign*}
    r &=  \sqrt{R(t)^2 + \left(\frac{R'(t)}{\tau(t)}\right)^2}
    &&\\
    c &=  \alpha(t) + R(t)N(t) + \frac{R'(t)}{\tau(t)}B(t)
\end{flalign*}

\newpage \noindent
\underline{$(ii) \Rightarrow (iii)$} 
\vspace{2mm} \\
\noindent
Supongamos que existe $r > 0$ tal que para todo $t \in I$ se cumple:
$$r^2 = R(t)^2 + \left(\frac{R'(t)}{\tau(t)} \right)^2$$

\noindent
Derivando en la definición de radio de curvatura, obtenemos: 
$$R(t) = \frac{1}{\kappa(t)} \Rightarrow R'(t) = \frac{-\kappa'(t)}{\kappa(t)^2}$$

\vspace{3mm} \noindent
Derivamos ahora la expresión del radio que nos da la hipótesis y sustituimos $R'(t)$ por la expresión anterior:
\vspace{-3mm}
\begin{center}
    \begin{flalign*}
        2 & R(t) R'(t) + 2 \frac{R'(t)}{\tau(t)} \left(\frac{R'(t)}{\tau(t)} \right)' = 0
        \Rightarrow  \frac{1}{\kappa(t)} \left(\frac{-\kappa'(t)}{\kappa(t)^2} \right)
         + \left(\frac{-\kappa'(t)}{\tau(t)\kappa(t)^2} \right) \left(\frac{-\kappa'(t)}{\tau(t)\kappa(t)^2} \right)' = 0\Rightarrow
        &\\
        \Rightarrow &\frac{-\kappa'(t)}{\kappa(t)^3}
         = \left(\frac{-\kappa'(t)}{\tau(t)\kappa(t)^2} \right) \left(\frac{\kappa'(t)}{\tau(t)\kappa(t)^2} \right)'
        \Rightarrow \frac{-\kappa'(t)}{\kappa(t)^3}  \left(\frac{-\tau(t)\kappa(t)^2}{\kappa'(t)}\right)
         = \left(\frac{\kappa'(t)}{\tau(t)\kappa(t)^2}\right)' \Rightarrow
        &\\
        \Rightarrow & \frac{\tau(t)}{\kappa(t)} = \left(\frac{\kappa'(t)}{\tau(t)\kappa(t)^2}\right)'
    \end{flalign*}
\end{center}
\vspace{3mm}
\noindent Hemos llegado a la expresión deseada.



\vspace{2cm}

\noindent \underline{$(iii) \Rightarrow (i)$} \vspace{2mm}

Supongamos que para todo $t \in I$ se verifica la siguiente expresión:
$$\frac{\tau(t)}{\kappa(t)} = \left(\frac{\kappa'(t)}{\tau(t)\kappa(t)^2}\right)'$$

Buscamos comprobar que la parametrización $\alpha$ se encuentra contenida en una esfera.
Para ello, debemos demostrar en primer lugar que la expresión de su centro es constante;
es decir, que la derivada de su función es siempre nula.

$$ c(t) = \alpha(t) + R(t)N(t) + \frac{R'(t)}{\tau(t)}B(t)$$
$$ c'(t) = T(t) + R'(t)N(t) + R(t)N'(t) + \left(\frac{R'(t)}{\tau(t)}\right)'B(t) + \left(\frac{R'(t)}{\tau(t)}\right)B'(t)$$

\newpage
\noindent Recordando las ecuaciones de Frenet-Serret:
$$N'(t) = -\kappa(t) T(t) + \tau(t) B(t)
\hspace{10mm}
B'(t) = -\tau(t)N(t)$$

\vspace{6mm}
\noindent Sustituyendo en la expresión de $c'(t)$ obtenemos:
\vspace{-6mm}
\begin{center}
    \begin{flalign*}
        c'(t) &= T(t) + R'(t)N(t) + R(t)\big{(}-\kappa(t)T(t) + \tau(t)B(t)\big{)}
         + \left(\frac{R'(t)}{\tau(t)}\right)'B(t)
         + \left(\frac{R'(t)}{\tau(t)}\right)\big{(}-\tau(t)N(t)\big{)} =
        &\\
        &= T(t) + R'(t)N(t) - \frac{\kappa(t)}{\kappa(t)}T(t)
         + \frac{\tau(t)}{\kappa(t)} B(t)
         + \left(\frac{R'(t)}{\tau(t)}\right)'B(t)
         - R'(t)N(t) =
        &\\
        &= \frac{\tau(t)}{\kappa(t)} B(t)
         + \left(\frac{R'(t)}{\tau(t)}\right)'B(t)
    \end{flalign*}
\end{center}

\vspace{6mm}
Por hipótesis, se cumple
$\frac{\tau(t)}{\kappa(t)} = \left(\frac{\kappa'(t)}{\tau(t)\kappa(t)^2}\right)'$,
pero, tal y como vimos en el apartado anterior, sabemos que 
$\left(\frac{\kappa'(t)}{\tau(t)\kappa(t)^2}\right)'
= \left(\frac{-R'(t)}{\tau(t)}\right)'$.
Juntando ambas expresiones:
$\left(\frac{R`(t)'}{\tau(t)}\right)' = \frac{-\tau(t)}{\kappa(t)}$.
Sustituyendo en la derivada del centro:
$$ c'(t) = \frac{\tau(t)}{\kappa(t)} B(t) + \left(\frac{R'(t)}{\tau(t)}\right)'B(t)
= \frac{\tau(t)}{\kappa(t)} B(t) - \frac{\tau(t)}{\kappa(t)} B(t) = 0$$

\vspace{2mm} \noindent
Concluimos que la expresión del centro es constante.
Veamos qué ocurre con la expresión del radio.\\

Por cómo está definido, el radio es positivo para todo $t \in I$.
Por tanto, basta estudiar si $r^2$ es constante. Podemos manejar así
una expresión más sencilla de derivar.
$$ r^2 = \big{(}\alpha(t) - c\big{)} \boldsymbol{\cdot}\big{(}\alpha(t) - c\big{)}
= \left(-R(t)N(t) - \frac{R'(t)}{\tau(t)}B(t) \right)
\boldsymbol{\cdot} \left(-R(t)N(t) - \frac{R'(t)}{\tau(t)}B(t) \right)$$

\vspace{2mm} \noindent
Como el Triedro de Frenet es una base ortonormal en $\mathbb{R}^3$, obtenemos:
$$r(t)^2 = R(t)^2 + \left(\frac{R'(t)}{\tau(t)}\right)^2$$
$$r'(t)r(t) = R(t)R'(t) + \left(\frac{R'(t)}{\tau(t)}\right)\left(\frac{R'(t)}{\tau(t)}\right)'$$

\newpage \noindent
Una vez más, aplicando $R(t) = \frac{1}{\kappa(t)}$ , y  
$\left(\frac{R'(t)}{\tau(t)}\right)' =\frac{-\tau(t)}{\kappa(t)}$
obtenemos:
$$r'(t)r(t) = \frac{-\kappa'(t)}{\kappa(t)^3}
 + \left(\frac{-\kappa'(t)}{\tau(t)\kappa(t)^2}\right)
    \left(\frac{-\tau(t)}{\kappa(t)}\right)
= \frac{-\kappa'(t)}{\kappa(t)^3} + \frac{\kappa'(t)}{\kappa(t)^3} = 0$$

\vspace{4mm}
\noindent Finalmente $r(t) > 0  \hspace{2mm} \forall t \in I$
luego $r'(t)r(t) = 0 \Leftrightarrow r'(t) = 0$.

\vspace{10mm}
Por todo lo anterior, concluimos que tanto la expresión del centro
como la del radio son constantes. Además, dichas expresiones se corresponden
con las vistas para la esfera que contiene a $\alpha$ en el apartado $(i)$.
Se cumple la implicación.
\end{document}