\documentclass{article}
\begin{document}
\thispagestyle{empty}
\vfill
\begin{center}
    {\Huge \bfseries \sffamily Trabajo en grupo} \\[5ex]
    {\LARGE \bfseries \sffamily Fase I} \\[10ex]
    \Large {
        Andrés Fernández-Junquera Fernández UO302806\\[3ex]
        Bruno Martín Rivera UO302144\\[3ex]
        Javier Ortín Rodenas UO299855\\[3ex]
        Mateo Rama García UO300710
    }
\end{center}
\vfill 

\newpage

\tableofcontents

\newpage


\section{Primera parte}
\subsection{PasswordControl()}
Parte de Andrés
\subsection{CountActiveBits()}
Esta función debe pedir dos números enteros sin singo. Luego, debe contar el número de bits activos
que hay en cada número entre la posición 5 y la 8, ambos inclusive. Finalmente, en caso de que el 
número de bits activos de los dos números entre las posiciones 5 y 8  no sea igual, la función 
imprimirá 'No coinciden' y llamará a la función exit().
\subsection{AsmBasedControl()}
\hspace{1mm}
Esta función debe leer tres enteros y pasárselos a IsValidAssembly.
Según la parametrización original del enunciado, esta segunda función debe comprobar que se
cumplan las dos condiciones siguientes:
\begin{itemize}
    \item El bit 8 del segundo número es igual al bit 5 del tercer número
    \item El valor de los 2 bits más bajos del primer número interpretados como 
        binario natural es mayor que 11
\end{itemize}
Como la codificación de 11 en binario natural es 1011b, la segunda condición no puede
ocurrir nunca en estas condiciones. Por tanto, escribimos la función para que tome los 4 bits
más bajos del primer entero, los interprete como natural, y lo compare con 11.
\\[3ex]
\noindent Entrada válida: $a = -3, \hspace{1.5mm} b = 403, \hspace{1.5mm} c = 56$ \\[2ex]
\noindent Entrada no válida: $a = 1, \hspace{1.5mm} b = 7, \hspace{1.5mm} c = 5$

\subsection{ArrayMinMax()}
Parte de Bruno

\newpage
\section{Segunda parte}
\subsection{Dirección de memoria IsValidAssembly}
[PLACEHOLDER, CAPTURA DE PANTALLA]

\subsection{Dirección de memoria PasswordControl}
[PLACEHOLDER, CAPTURA DE PANTALLA]

\subsection{Marco de pila ArrayMinMax}
[PLACEHOLDER, CAPTURA DE PANTALLA]

\subsection{Acceso de lectura ArrayMinMax}
[PLACEHOLDER, CAPTURA DE PANTALLA]

\newpage

\section{División del trabajo}
\hspace{1mm} En primer lugar, dedicamos tiempo a leer el enunciado y repartir las tareas.
Cada alumno eligió una función para programarla y responder las preguntas sobre ella
de la segunda fase.
[PLACEHOLDER ACABAR]
\end{document}