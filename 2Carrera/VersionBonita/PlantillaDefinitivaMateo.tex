\documentclass[11pt,a4paper]{article}

\usepackage[spanish]{babel}
\usepackage{amsmath,amsfonts, amssymb, amsthm} % Podemos añadir amssymb, amsthm o bm
\usepackage{graphicx, tikz, xparse, array}
\usepackage[top=2cm,bottom=2cm,left=3cm,right=3cm,marginparwidth=1.75cm]{geometry} % Este paquete permite modificar los márgenes del documento
\usepackage[colorlinks=true, allcolors=blue]{hyperref} % Se indica que los hipervínculos van todos en azul
\usepackage{setspace}
\usepackage{caption}
\usepackage{xcolor}
\usepackage{graphicx} % Paquete para incluir imágenes
\usepackage{fancyhdr} % Paquete para cabeceras y pies de página
\usepackage{lipsum}   % Para generar texto de ejemplo
\usepackage[most]{tcolorbox}

\tcbuselibrary{breakable}


%Colores
\definecolor{verdeSuave}{HTML}{6AD58A}
\definecolor{verdeFuerte}{HTML}{206936}
\definecolor{blanco}{HTML}{FFFFFF}
\definecolor{negro}{HTML}{000000}
\definecolor{azulSuave}{HTML}{6ac9d5}
\definecolor{naranjaSuave}{HTML}{d5c06a}
\definecolor{rojoSuave}{HTML}{EF4949}
\definecolor{amarilloSuave}{HTML}{D8E058}


\setstretch{1.2}
\decimalpoint


%\graphicspath{ {images/}}

\title{\textbf{Tema 1: } Teoría de curvas}
\author{Mateo Rama García}


%\date{Fecha}

\begin{document}
  
\begin{titlepage}
  \centering  
  \vspace*{1cm}  % Espacio opcional antes de la imagen
  \includegraphics[width=0.3\textwidth]{uniovi.jpg} \hspace{2cm}
  \includegraphics[width=0.3\textwidth]{descarga.jpeg} \\[1cm] 
  \vspace{\fill}% Imagen centrada arriba
  \hrule
  \vspace{0.5cm}
  {\Huge \bfseries Trabajo en grupo\par}
  \vspace{0.5cm}
  {\Large \bfseries Fase I\par}
  \vspace{0.5cm}
  \hrule
  \vspace{1cm}
  {\bfseries Andrés Fernández-Junquera Fernández UO302806\\[3ex]
  Bruno Martín Rivera UO302144\\[3ex]
  Javier Ortín Rodenas UO299855\\[3ex]
  Mateo Rama García UO300710\par} % Nombres centrados
  \vspace{\fill}  % Espacio después de los nombres
  {\Large \textbf{Fundamentos de computadores y redes}\par}
\end{titlepage}



\newpage

\tableofcontents

\newpage


\section{Primera parte}
\subsection{PasswordControl()}
Parte de Andrés
\subsection{CountActiveBits()}
Esta función debe pedir dos números enteros sin singo. Posteriormente, debe contar el número de bits activos
que hay en cada número entre la posición 5 y la 8, ambos inclusive. Finalmente, en caso de que el 
número de bits activos entre las posiciones 5 y 8 de los dos números   no sea igual, la función 
imprimirá 'No coinciden' y llamará a la función exit().
\begin{itemize}
  \item Entrada válida: \(a = 352 \text{ y } b = 448\)
  \item Entrada no válida: \(a = 352 \text{ y } b = 0\)
\end{itemize}
\subsection{AsmBasedControl()}
\hspace{1mm}
Esta función debe leer tres enteros y pasárselos a IsValidAssembly.
Según la parametrización original del enunciado, esta segunda función debe comprobar que se
cumplan las dos condiciones siguientes:
\begin{itemize}
    \item El bit 8 del segundo número es igual al bit 5 del tercer número
    \item El valor de los 2 bits más bajos del primer número interpretados como 
        binario natural es mayor que 11
\end{itemize}
Como la codificación de 11 en binario natural es 1011b, la segunda condición no puede
ocurrir nunca en estas condiciones. Por tanto, escribimos la función para que tome los 4 bits
más bajos del primer entero, los interprete como natural, y lo compare con 11.
\\[3ex]
\noindent Entrada válida: $a = -3, \hspace{1.5mm} b = 403, \hspace{1.5mm} c = 56$ \\[2ex]
\noindent Entrada no válida: $a = 1, \hspace{1.5mm} b = 7, \hspace{1.5mm} c = 5$

\subsection{ArrayMinMax()}
Parte de Bruno

\newpage
\section{Segunda parte}
\subsection{Dirección de memoria IsValidAssembly}
[PLACEHOLDER, CAPTURA DE PANTALLA]

\subsection{Dirección de memoria PasswordControl}
[PLACEHOLDER, CAPTURA DE PANTALLA]

\subsection{Marco de pila ArrayMinMax}
[PLACEHOLDER, CAPTURA DE PANTALLA]

\subsection{Acceso de lectura ArrayMinMax}
[PLACEHOLDER, CAPTURA DE PANTALLA]

\newpage

\section{División del trabajo}
\hspace{1mm} En primer lugar, dedicamos tiempo a leer el enunciado y repartir las tareas.
Cada alumno eligió una función para programarla y responder las preguntas sobre ella
de la segunda fase.
[PLACEHOLDER ACABAR]

\end{document}
