\documentclass{article}
\usepackage{enumitem}
\usepackage{amssymb}
\usepackage{amsmath}
\usepackage{enumitem}
\usepackage{multicol}

\title{Pregunta corta}   
\author{Control 2 PyE}
\date{}
\begin{document}
\maketitle
\noindent
\section{Enunciado}
Consideramos la siguiente sucesión de variables aleatorias $X_n$ definida sobre
el espacio de probabilidad $(\mathbb{R}, \mathcal{B}_\mathbb{R}, \mathcal{P})$,
donde $P$ es sigue una distribución exponencial de parámetro $\lambda = 1$.
$$ X_n (\omega) =
\begin{cases}
  n & \text{si } 0 \leq x \leq \omega \\
  0   & \text{en otro caso }
\end{cases}$$
¿Tal sucesión converge? De hacerlo, ¿a qué variable aleatoria?

\section{Solución}
\hspace{3mm}
Nótese que la pregunta simplemente hace referencia a si la sucesión
converge o no, no al tipo de convergencia; luego no es necesario estudiar
todos los tipos de convergencia (confirmado por Montenegro).

\vspace{2mm}
Veamos qué función de distribución tienen las $X_n$ para estudiar la convergencia
débil, pues es el caso más sencillo. Para ello, comencemos por hallar la probabilidad
de un conjunto en $(\mathbb{R}, \mathcal{B}_\mathbb{R}, P)$ al usar la función de densidad de $P$:
$$P \sim \varepsilon(1) \text{ luego tenemos }
f_P(\omega) =
\begin{cases}
    e^{-\omega}& \text{ si } \omega > 0 \\
    0 &\text{ si } \omega \leq 0
\end{cases}$$

\noindent
Estudiemos ahora el comportamiento de las $X_n$ según $P$ al integrar
los conjuntos de $\Omega = \mathbb{R}$ correspondientes:
$$P(X_n \leq x) = P(\omega \in \mathbb{R} : X_n(\omega) \leq x)$$
\begin{flalign*}
    F_{X_n}(x) =
    \begin{cases}
        P(\emptyset) & \text{ si } x < 0 \\
        P((-\infty, 0)\cup(n, +\infty)) & \text{ si } 0 \leq x < n \\
        P(\mathbb{R}) & \text{ si } x \geq n
    \end{cases}
\end{flalign*}
\newpage
Por los axiomas de probabilidad sabemos que $P(\emptyset) = 0$ y que
$P(\mathbb{R}) = P(\Omega) = 1$, queda ver $P((-\infty, 0)\cup(n, +\infty))$.
Este es el conjunto a integrar, pues los $\omega$ de este conjunto son
tales que $X_n(\omega) = 0 \in [0, n)$.

\vspace{2mm}
Como $(-\infty, 0)\cup(n, +\infty) = \mathbb{R}\backslash[0,n]$, pordemos
aplicar las propiedades del complementario de la probabilidad:
\begin{flalign*}
    P([0,n]) = \int_0^n f_P(t) dt
    = \int_0^n e^{-t} dt = [-e^-t]_0^n
    = e^0 - e^{-n} = 1-e^{-n}
\end{flalign*}
\noindent Así,
$P((-\infty, 0)\cup(n, +\infty)) = 1 - P([0,n])
 = 1 - (1 - e^{-n}) = e^{-n}$

 \vspace{2mm}
Volviendo a la función de distribución:
\begin{flalign*}
    F_{X_n}(x) =
    \begin{cases}
        0 & \text{ si } x < 0 \\
        e^{-n} & \text{ si } 0 \leq x < n \\
        1 & \text{ si } x \geq n
    \end{cases}
\end{flalign*}

\vspace{2mm}
Por la propiedad arquimediana, 
$\lim_{n \to \infty} F_{X_n}(x) = 0 \hspace{2mm}
\forall x \in \mathbb{R}$, luego converge a la función nula.
Al ser continua en $\mathbb{R}$, no la podemos "arreglar" para
que converga débilmente. Como el resto de convergencias implican la
convergencia en distribución, $(X_n)_{n \in \mathbb{N}}$ no converge
a ninguna variable aleatoria.
\end{document}